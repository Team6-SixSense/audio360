\documentclass{article}

\usepackage{tabularx}
\usepackage{booktabs}
\usepackage{cite}

\title{Problem Statement and Goals\\\progname}

\author{\authname}

\date{}

\input{../Comments}
%% Common Parts

\newcommand{\progname}{Audio360} % PUT YOUR PROGRAM NAME HERE
\newcommand{\authname}{Team 6, SixSense
\\ Omar Alam
\\ Sathurshan Arulmohan
\\ Nirmal Chaudhari
\\ Kalp Shah
\\ Jay Sharma
} % AUTHOR NAMES                  

\usepackage{hyperref}
    \hypersetup{colorlinks=true, linkcolor=blue, citecolor=blue, filecolor=blue,
                urlcolor=blue, unicode=false}
    \urlstyle{same}
                                


\begin{document}

\maketitle

\begin{table}[hp]
\caption{Revision History} \label{TblRevisionHistory}
\begin{tabularx}{\textwidth}{llX}
\toprule
\textbf{Date} & \textbf{Developer(s)} & \textbf{Change}\\
\midrule
Date1 & Name(s) & Description of changes\\
Date2 & Name(s) & Description of changes\\
... & ... & ...\\
\bottomrule
\end{tabularx}
\end{table}

\section{Problem Statement}

\subsection{Problem}

Individuals who are deaf or hard of hearing tend to have difficulties with staying situationally aware, generally leading to increased risk of injury.
Many safety cues such as "the sounds of a tea kettle, the warning beep as a fork lift backs up, and the engine of an oncoming car may be missed". \cite{Masterson2016}
General sound cues such as someone calling their name, or a phone ringing, may also be missed, often leading to miscommunication and elevated frustration.
With 1 in 10 Canadians being impacted by hearing loss \cite{Healthing2025}, there are over 4 million individuals in Canada dealing with these struggles every day.


\subsection{Inputs and Outputs}

The high level input of the system is the audio from the surrounding environments, with the output
being a visual indication of the direction of audio sources and their respective classifications.

\subsection{Stakeholders}

The primary stakeholders are individuals who are deaf or hard of hearing.

\subsection{Environment}

The software environment is a real-time operating system (RTOS, i.e. FreeRTOS) running C/C++ code.
The hardware environment is a 4 dimentional microphone array, with each microphone connected to a STM32F767 microcontroller for processing, as well as a pair of smart glasses (Rokid Smart Glasses) for the visual display.

\section{Goals}

\section{Stretch Goals}

\section{Extras}

% TODO: link to the actual documentation when we create them.
\begin{enumerate}
    \item Price + Hardware Selection Report
    \item Usability Report
\end{enumerate} 

\newpage{}

\section*{Appendix --- Reflection}

\input{../Reflection.tex}

\begin{enumerate}
    \item What went well while writing this deliverable? 
    
    \textbf{Omar Alam:} I think all members of our team were proactive and genuinely interested in the project presented which made it easier
    to delegate and expect high quality work. 
    \item What pain points did you experience during this deliverable, and how
    did you resolve them?
    \textbf{Omar Alam:} Since the project idea incorporates glasses with displays that are visible to the user, we had to do a significant amount
    of research to figure out if it was feasible in the time that we have. We resolved this by developing a contingency plan that would allow us to 
    still allow us to develop the core algorithms without the display glasses.
    \item How did you and your team adjust the scope of your goals to ensure
    they are suitable for a Capstone project (not overly ambitious but also of
    appropriate complexity for a senior design project)?
    \textbf{Omar Alam:} My team and I spent a significant amount of time researching the feasibility of the project. Since the team does not have much
    experience with signal processing, we decided to consult with Dr. Mohrenschildt to get his opinion on the project. He provided us with valuable feedback on how
    to constraint our project goals to ensure that we can complete the project in the time we have.
\end{enumerate}  

\bibliographystyle{IEEEtran}
\bibliography{../../refs/References}

\end{document}