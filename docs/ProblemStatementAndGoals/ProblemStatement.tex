\documentclass{article}

\usepackage{tabularx}
\usepackage{booktabs}

\title{Problem Statement and Goals\\\progname}

\author{\authname}

\date{}

\input{../Comments}
%% Common Parts

\newcommand{\progname}{Audio360} % PUT YOUR PROGRAM NAME HERE
\newcommand{\authname}{Team 6, SixSense
\\ Omar Alam
\\ Sathurshan Arulmohan
\\ Nirmal Chaudhari
\\ Kalp Shah
\\ Jay Sharma
} % AUTHOR NAMES                  

\usepackage{hyperref}
    \hypersetup{colorlinks=true, linkcolor=blue, citecolor=blue, filecolor=blue,
                urlcolor=blue, unicode=false}
    \urlstyle{same}
                                


\begin{document}

\maketitle

\begin{table}[hp]
\caption{Revision History} \label{TblRevisionHistory}
\begin{tabularx}{\textwidth}{llX}
\toprule
\textbf{Date} & \textbf{Developer(s)} & \textbf{Change}\\
\midrule
Date1 & Name(s) & Description of changes\\
Date2 & Name(s) & Description of changes\\
... & ... & ...\\
\bottomrule
\end{tabularx}
\end{table}

\section{Problem Statement}

\wss{You should check your problem statement with the
\href{https://github.com/smiths/capTemplate/blob/main/docs/Checklists/ProbState-Checklist.pdf}
{problem statement checklist}.} 

\wss{You can change the section headings, as long as you include the required
information.}

\subsection{Problem}

\subsection{Inputs and Outputs}

\wss{Characterize the problem in terms of ``high level'' inputs and outputs.  
Use abstraction so that you can avoid details.}

\subsection{Stakeholders}

\subsection{Environment}

\wss{Hardware and Software Environment}

\section{Goals}

Our goal is to help deaf and hard-of-hearing users stay aware of what is happening around them. The system will listen to the environment using a four-microphone setup and, in real time, show simple cues on Rokid smart glasses. At a basic level, it will point out where important sounds are coming from (front, left, right, or behind), turn speech into short text (like when someone calls your name), and recognize everyday sounds such as a tea kettle, a reversing beep, an engine, or a phone ringing. The on-glasses display will be fast and easy to understand, with safety-related alerts shown first. We aim for it to work reliably in typical indoor and outdoor spaces, and we will protect privacy by avoiding unnecessary recording and keeping processing on the device or a trusted nearby system when possible.
\section{Stretch Goals}

If time allows, we will make the system more precise and helpful. We want to show exactly where a sound comes from, not just the general direction, and estimate distance when possible. We also plan to handle several sounds at once, and make the system work better in noisy or echoey places. The system will also get smarter about which alerts to show first (for example, safety-related sounds or user-selected sounds), and let users customize sensitivity, and how alerts look and feel. For critical moments, we may add optional haptic feedback through a wearable. We also plan a short “recent events” view so users can review what just happened, and we will work on power savings so it can run comfortably through the day.
\section{Extras}

% TODO: link to the actual documentation when we create them.
\begin{enumerate}
    \item Price + Hardware Selection Report
    \item Usability Report
\end{enumerate} 

\newpage{}

\section*{Appendix --- Reflection}

\wss{Not required for CAS 741}

\input{../Reflection.tex}

\begin{enumerate}
    \item What went well while writing this deliverable? 
    
    \textbf{Omar Alam:} I think all members of our team were proactive and genuinely interested in the project presented which made it easier
    to delegate and expect high quality work. 
    \item What pain points did you experience during this deliverable, and how
    did you resolve them?
    \textbf{Omar Alam:} Since the project idea incorporates glasses with displays that are visible to the user, we had to do a significant amount
    of research to figure out if it was feasible in the time that we have. We resolved this by developing a contingency plan that would allow us to 
    still allow us to develop the core algorithms without the display glasses.
    \item How did you and your team adjust the scope of your goals to ensure
    they are suitable for a Capstone project (not overly ambitious but also of
    appropriate complexity for a senior design project)?
    \textbf{Omar Alam:} My team and I spent a significant amount of time researching the feasibility of the project. Since the team does not have much
    experience with signal processing, we decided to consult with Dr. Mohrenschildt to get his opinion on the project. He provided us with valuable feedback on how
    to constraint our project goals to ensure that we can complete the project in the time we have.
\end{enumerate}  

\end{document}