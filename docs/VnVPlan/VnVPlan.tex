\documentclass[12pt, titlepage]{article}

\usepackage{amssymb}
\usepackage{booktabs}
\usepackage{float}
\usepackage{tabularx}
\usepackage{xltabular}
\usepackage[x11names]{xcolor}
\usepackage{framed}
\usepackage{quoting}
\usepackage{hyperref}
\usepackage{xr}
\usepackage{cite}
\hypersetup{
    colorlinks,
    citecolor=blue,
    filecolor=black,
    linkcolor=red,
    urlcolor=blue
}
\usepackage[numbers]{natbib}
\usepackage{amsmath}
\usepackage{float}

\colorlet{shadecolor}{gray!15}

% Environment for quotes
\newenvironment{shadedquotation}
    {\begin{shaded*}
     \quoting[leftmargin=0pt, vskip=0pt]}
    {\endquoting
     \end{shaded*}}

\input{../Comments}
\input{../Common}

\setlength{\parindent}{0pt} % Set no indent to entire document.

\begin{document}

\title{System Verification and Validation Plan for \progname{}} 
\author{\authname}
\date{\today}
	
\maketitle

\pagenumbering{roman}

\externaldocument[SRS-]{../SRS/SRS}

\section*{Revision History}

\begin{tabularx}{\textwidth}{p{3cm}p{2cm}X}
\toprule {\bf Date} & {\bf Version} & {\bf Notes}\\
\midrule
2025-10-27 & 1.0 & Initial write-up\\
\bottomrule
\end{tabularx}

\newpage

\tableofcontents

\listoftables

\listoffigures
\wss{Remove this section if it isn't needed}

\newpage

\section{Symbols, Abbreviations, and Acronyms}

\renewcommand{\arraystretch}{1.2}
\begin{tabular}{l l} 
  \toprule		
  \textbf{symbol} & \textbf{description}\\
  \midrule 
  T & Test\\
  \bottomrule
\end{tabular}\\

\wss{symbols, abbreviations, or acronyms --- you can simply reference the SRS
  SRS tables, if appropriate}

\wss{Remove this section if it isn't needed}

\newpage

\pagenumbering{arabic}

This document ... \wss{provide an introductory blurb and roadmap of the
  Verification and Validation plan}

\section{General Information}

\subsection{Summary}

\wss{Say what software is being tested.  Give its name and a brief overview of
  its general functions.}

\subsection{Objectives}

\wss{State what is intended to be accomplished.  The objective will be around
  the qualities that are most important for your project.  You might have
  something like: ``build confidence in the software correctness,''
  ``demonstrate adequate usability.'' etc.  You won't list all of the qualities,
  just those that are most important.}

\wss{You should also list the objectives that are out of scope.  You don't have 
the resources to do everything, so what will you be leaving out.  For instance, 
if you are not going to verify the quality of usability, state this.  It is also 
worthwhile to justify why the objectives are left out.}

\wss{The objectives are important because they highlight that you are aware of 
limitations in your resources for verification and validation.  You can't do everything, 
so what are you going to prioritize?  As an example, if your system depends on an 
external library, you can explicitly state that you will assume that external library 
has already been verified by its implementation team.}

\subsection{Challenge Level and Extras}

\wss{State the challenge level (advanced, general, basic) for your project.
Your challenge level should exactly match what is included in your problem
statement.  This should be the challenge level agreed on between you and the
course instructor.  You can use a pull request to update your challenge level
(in TeamComposition.csv or Repos.csv) if your plan changes as a result of the
VnV planning exercise.}

\wss{Summarize the extras (if any) that were tackled by this project.  Extras
can include usability testing, code walkthroughs, user documentation, formal
proof, GenderMag personas, Design Thinking, etc.  Extras should have already
been approved by the course instructor as included in your problem statement.
You can use a pull request to update your extras (in TeamComposition.csv or
Repos.csv) if your plan changes as a result of the VnV planning exercise.}

\subsection{Relevant Documentation}

\wss{Reference relevant documentation.  This will definitely include your SRS
  and your other project documents (design documents, like MG, MIS, etc).  You
  can include these even before they are written, since by the time the project
  is done, they will be written.  You can create BibTeX entries for your
  documents and within those entries include a hyperlink to the documents.}

\wss{Don't just list the other documents.  You should explain why they are relevant and 
how they relate to your VnV efforts.}

\section{Plan}

This section will go over the \hyperref[sec:vnv_team]{verification and
validation team}. It will then be followed by the plan to verify the
\hyperref[sec:srs_verification]{SRS},
\hyperref[sec:design_verification]{design}, 
\hyperref[sec:vnv_plan_verification]{verification and validation plan}, and
\hyperref[sec:implementation_verification]{implementation}. Finally the section
will end off with 
\hyperref[sec:testing_tools]{automated testing and verification tools} and
\hyperref[sec:software_validation]{software validation}.

\subsection{Verification and Validation Team}\label{sec:vnv_team}

Figure \ref{table:vnv_team} outlines the roles and responsibilities of each
team member involved in the verification and validation process. Roles were
intentionally assigned to individuals not directly responsible for the
corresponding implementation components, ensuring an unbiased evaluation of 
system functionality. The supervisor also contributes by providing technical
oversight and expert validation for signal processing.

\begin{xltabular}{\textwidth}{|X|X|c|}

  \caption{Verification and validation team breakdown.}
  \label{table:vnv_team} \\
  \toprule
  \textbf{Role} & \textbf{Description} & \textbf{Assignee} \\
  \midrule
  \endfirsthead

  \toprule
  \textbf{Role} & \textbf{Description} & \textbf{Assignee} \\
  \midrule
  \endhead

  \bottomrule
  \multicolumn{3}{r}{\textit{Continued on next page}} \\
  \endfoot

  \bottomrule
  \endlastfoot


  Firmware Verification \label{role:firmware_verfication} &
  Develops and executes tests to confirm that the firmware implementation
  conforms to the requirements outlined in the software specification. &
  Jay \\
  \hline

  Visualization Verification + Validation \label{role:visual_vnv}&
  Develops and executes tests to confirm that the visualization implementation
  conforms to the requirements outlined in the software specification.
  Also responsible for engaging with users to validate the usability of the
  product specific to the visualization. &
  Nirmal \\
  \hline

  Audio Classification Verification \label{role:classification_verfication} &
  Develops and executes tests to confirm that the audio classification module
  conforms to the requirements outlined in the software specification.
  &
  Sathurshan \\
  \hline
  
  Directional Analysis Verification \label{role:directional_verfication}&
  Develops and executes tests to confirm that the directional analysis
  component conforms to the requirements outlined in the software
  specification. &
  Omar \\
  \hline

  Product Validation \label{role:product_validation} &
  Responsible for engaging with individuals who are hard of hearing to
  validate that the system effectively addresses their pain points related to
  situational awareness. &
  Kalp \\
  \hline

  Audio Processing Verification \label{role:audio_processing_verification}  &
  Reviews and assesses the audio processing methodologies implemented. The
  team will demonstrate and explain these techniques in a supervisor meeting
  and receive verbal or written feedback. &
  MVM (Supervisor) \\

\end{xltabular}

\subsection{SRS Verification}\label{sec:srs_verification}

The verification of the SRS will follow a structured and systematic process.
Each software requirement will be associated with at least one corresponding
test case, which will verify whether the
implementation satisfies the intended specification. Both unit and integration
testing will be conducted to confirm functionality at the component and system
levels. To avoid bias, test cases will be developed and executed by a team
member who was not directly involved in the implementation of the component.
These fall under the
\hyperref[role:firmware_verfication]{firmware verification},
\hyperref[role:visual_vnv]{visual verification},
\hyperref[role:classification_verfication]{audio classification verification},
\hyperref[role:directional_verfication]{directional analysis verification}
roles. \newline

The team will adopt a new GitHub peer review process to SRS verification. A 
comment will be added by bot to the PR. The comment will contain a list of
reminders for the reviewers to confirm that the
software implementation and/or written test cases comply with the requirements
defined in the SRS. If a requirement cannot be met, the reviewer will be
instructed to request an update in a separate PR linked with a rationale for the
change. Below is the text that is included in the bot's comment addressing this
topic. \newline

\begin{shadedquotation}
Reviewer's Note

- Ensure that all implemented features and/or test cases comply with the SRS.
If any requirement cannot be met, link a separate PR updating the SRS and
explaining the rationale for the change.
\end{shadedquotation}

The 
\hyperref[role:audio_processing_verification]{supervisor verification process}
will follow a formal meeting based review approach. During these meetings,
the team will present core system elements, mainly audio processing
methodologies, using mathematical descriptions, prototype demonstrations,
and graphed data. The supervisor will be provided with targeted review
questions and asked to identify potential weaknesses or missing test cases.
Feedback will be documented, and resulting action items will be tracked
and resolved through the project's issue tracker. \newline


For validation, the team will engage users who are hard of hearing in structured
sessions. These sessions will include observation of product use and
semi-structured interviews. The observation aspect aims to allow the team
understand the usability of the product while the interview serves as a method 
to identify validation issues in addressing user's needs related to situational
awareness. This will fall under the 
\hyperref[role:visual_vnv]{visual validation} and
\hyperref[role:product_validation]{product validation} roles.

\subsection{Design Verification}\label{sec:design_verification}

The design verification process will include structured peer reviews conducted
by the team. Below is the checklist to verify the design.\\
\newline
\textbf{Software Core Architecture} \\
$\square$ Does the selected software architecture appropriately support the
system's requirements and intended functionality? \\
$\square$ Is the software design portable, allowing the software to be easily
integrated with different hardware or simulation layer. \\
\newline
\newline
\textbf{Software Design} \\
$\square$ Is the system decomposed into small, modular components that can be
individually tested? \\
$\square$ Are encapsulation principles followed, ensuring that data and
functions that should be private are private? \\
$\square$ Are design assumptions, dependencies, and interfaces clearly defined
and documented? \\
$\square$ Are software design principles being followed. Check box should fail
if there is an another design principle that will be better fitted.
\newline
\newline
\textbf{General} \\
$\square$ Is there a corresponding UML diagram of the design being tracked on
git. \\

Reviewers will document feedback on any checklist criteria that are not
satisfied and provide recommendations for improvement. The team will track all 
feedback using the project's issue tracker.

\subsection{Verification and Validation Plan Verification}
\label{sec:vnv_plan_verification}

The verification and validation plan verification process will include
structured peer reviews conducted by the Teaching Assistant and Team 13.
They will use the checklist from \texttt{Checklists/VnV-Checklist.pdf}.

\subsection{Implementation Verification}\label{sec:implementation_verification}

As outlined in the development plan, the primary source code implementation
will be developed in C/C++. The compiler used to build the source code
will provide warnings of potential bugs. The team will resolve all
warnings that is under the team's control, this excludes warnings from
imported libraries. \newline

The team will also employ Clang Static Analyzer \cite{clangStaticAnalyzer} as
a static analyzer tool. The static analyzer will be employed to detect bugs
without running the source code on the hardware, and will be ran prior to
merging PRs. It will block the PR from merging until all issues identified by
the static analyzer has been resolved.

The team will also develop test that verify requirements. These tests are
outline in the \hyperref[sec:system_tests]{System Tests section}.

\subsection{Automated Testing and Verification Tools}
\label{sec:testing_tools}

Automated testing and verification tools are defined in the following sections
from the Development Plan document.

\begin{itemize}
  \item 10.3: Linter, Static Analyzer and Formatting Tools
  \item 10.4: Testing Frameworks (section also includes code coverage)
  \item 10.5: CI/CD (contains automated testing plan in CI)
\end{itemize}

As the software will be deployed on an embedded device, running unit tests on
hardware is infeasible. As a result, all unit tests will be done on developer's
local machine without any hardware in loop.

\subsection{Software Validation}\label{sec:software_validation}

The \hyperref[role:product_validation]{Product Validation} role, defined in
section \hyperref[sec:vnv_team]{3.1: Verification and Validation Team}, is
responsible for validating the product with the primary stakeholder. The
product is composed of both software and hardware with more focus on the
software. The validation will be conducted primarily with members of the
McMaster Sign Language Club, who may not have the technical expertise to
evaluate the requirements from the SRS. To address this, the Product Validation
team member will conduct semi-structured interviews as described in section
\hyperref[sec:srs_verification]{3.2: SRS Verification}. Rev 0 demo will include
the results of the user validation, providing an opportunity to gather feedback
and improve the software.

\section{System Tests} \label{sec:system_tests}

This section outlines the tests for verifying and validating the functional and 
nonfunctional requirements outlined in the SRS \citep{SRS}. When done correctly 
this ensures the system meets the user expectations and performs reliably. 

\subsection{Tests for Functional Requirements}

The sections below outline the tests that will be used to verify the functional 
requirements in section \hyperref[SRS-sec:S.2]{S.2} of the SRS. Each subsection 
will focus on how the functional requirements for a specific component will be 
verified through testing. These components include the Embedded Firmware, 
Driver Layer, Audio Filtering, Audio360 Engine, Frequency Analysis, 
Visualization Controller, Microphone, Output Display and Microcontroller. 

\subsubsection{Audio Filtering Tests}

This section covers the tests for ensuring the system processes audio into a 
form that can be analyzed by internal components of the system. Each test is 
associated with a functional requirement defined under section 
\hyperref[SRS-sec:FR3]{3.2.3} of the SRS. As such, each test will verify whether
 the system meets the associated functional requirement. 

\begin{enumerate}

\item{\textbf{test-FR-3.1} Converting time-domain audio signals to 
frequency-domain \\}

\textbf{Control:} Automatic
					
\textbf{Initial State:} 
The audio filtering module is initialized and ready to process audio input 
retrieved from an audio file. 
					
\textbf{Input:}
A 3 second audio clip represented in an audio file containing pre-recorded audio 
data in the time domain sampled at 16 kHz. The audio clip contains 3 sine waves 
at low (100 Hz), mid (1 kHz), and high (8 kHz) frequency ranges. No filtering 
or frequency transformation have been applied to the audio data initially.
					
\textbf{Output:}
The audio filtering module accepts the file with no errors. The resulting 
frequency domain representation should display 3 spectral peaks at 
approximately 100 Hz, 1 kHz, and 8 kHz, corresponding to the sine waves.

\textbf{Test Case Derivation:} 
The Fourier Transform converts time-domain signals into frequency domain by 
independently extracting the frequency of various waves in the signals and 
plotting the peaks at those frequencies after the transformation. In the 
original audio clip, there are 3 sine waves at 100 Hz, 1 kHz, and 8 kHz. After
applying the Fourier Transform, the resulting frequency domain representation
should display peaks at those frequencies.
					
\textbf{How test will be performed:}
The test file will be uploaded as an artifact in the automated testing 
framework. This test will trigger when a commit is made to any branch in the 
repository. The audio filtering module will return the frequency domain 
representation automatically on the input of the audio file. The 
frequency-domain output will be inspected to verify the presence of peaks at 
100 Hz, 1 kHz and 8 kHz. The test passes if all 3 peaks are present with no 
unexpected frequencies showing up.  
					
\item{\textbf{test-FR-3.2} Normalize amplitude of signals\\}

\textbf{Control:} Automatic
					
\textbf{Initial State:} 
The audio filtering module is initialized and ready to process audio input 
retrieved from a audio file. 
					
\textbf{Input:}
A 2 second digital audio signal sampled at 16 kHz that alternates between a 
low-amplitude sine wave and a high-amplitude sine wave with the same frequency. 
These sine waves will be decimal multiples of a defined max amplitude value. 
Where the low-sine wave will be 0.2 * max amplitude, and the high sine wave will
 be 0.8 * max amplitude. 
					
\textbf{Output:}
A normalized output signal that still has both the low amplitude and high 
amplitude sine waves, but both waves have been scaled to a consistent target 
amplitude, having a maximum absolute value of 1.0. Note, the frequency of the 
sine wave should remain unchanged. 

\textbf{Test Case Derivation:} 
Amplitude normalization scales the amplitude of a signal so its maximum 
ampltiude is between 0 and 1. If one section is quiet (0.2 * max), and 
another section is louder (0.8 * max), normalization should scale both sections 
so their peak amplitudes are between the range 0 and 1. 
					
\textbf{How test will be performed:}

The test file will be uploaded as an artifact in the automated testing 
framework. This test will trigger when a commit is made to any branch in the 
repository. The audio filtering module will return normalized time-domain 
signal automatically on the input of the audio file. The normalized time 
domain output will be inspected to verify the amplitude across both sections of 
the file are the same now.

\item{\textbf{test-FR-3.3} Reduced spectral leakage\\}

\textbf{Control:} Automatic
					
\textbf{Initial State:} 
The audio filtering module is initialized and ready to process audio input 
retrieved from an audio file. 
					
\textbf{Input:}
A 1 second sine wave at an arbitary frequency sampled at 16 kHz, whose duration 
does not contain an integer number of cycle (the cycle is cut off before 1 
period is complete). This intentionally causes spectral leakage. This will be 
passed in with a parameter of whether to apply a windowing function or not. 
In one case, a window function will be passed in, in the other no function
will be passed in.
					
\textbf{Output:}
The windowed output audio should have reduced spectral leakage. This is 
represented by a sharper and more defined peak at the sine wave's frequency,
with reduced side-lobes in the frequency spectrum compared to the output of the 
non-windowed case. 

\textbf{Test Case Derivation:} 
Spectral leakage occurs when a signal is truncated without windowing, causing 
discontinuities at the edges of the truncated signal. Applying a windowing 
function tapers the edges of the signal, reducing the discontinuities, and 
confining the energy to the main frequency band, preventing leakage into other 
frequencies from occuring. As such, in the windowed case, the frequency 
spectrum should show a sharper peak at the sine wave's frequency, with redcued 
side-lobes compared to the non-windowed case. The leakage will be measured by 
first computing the peak amplitude $K_{\text{peak}}$, then applying the 
leakage function. Where M represents the mainlobe half-width in bins, based on 
the windowing function used.  

\[
\text{Leakage} = 1 - \frac{\displaystyle\sum_{k = k_{\text{peak}} - M}^{k_{\text{peak}} + M} |X[k]|^2}{\displaystyle\sum_{k} |X[k]|^2}
\]

\textbf{How test will be performed:}
The test file will be uploaded as an artifact in the automated testing 
framework. This test will trigger when a commit is made to any branch in the 
repository. The audio filtering module will return 2 frequency-domain spectrums.
 One spectrum will be generated without windowing, and the other will be applied
 with a windowing function. For each spectrum, the amplitude of the main-lobe 
will be compared with the largest side-lobe amplitude. The test passes if the 
side-lobe in the filtered case is lower than the unfiltered case, which 
indicates reduced spectral leakage. 

\item{\textbf{test-FR-3.4} Hardware acceleration\\}

\textbf{Control:} Manual
					
\textbf{Initial State:} 
The audio filtering module is deployed on the microcontroller. A local computer 
without hardware acceleration is available to the team to test the audio 
filtering component on.
					
\textbf{Input:}
A 10 second digital audio signal sampled at 16 kHz containing waves with 
mixed frequencies and amplitudes. The same input will be processed once with 
the microcontroller, and once on a local development machine without hardware 
acceleration. Each test will be timed to measure the processing speed. 
					
\textbf{Output:}
Both processing modes should produce equivalent spectrograms for the given 
audio input. This means for each frequency in the spectrogram, the amplitude 
defined in the hardware-accelerated mode should match the amplitude in the 
non-accelerated mode within a defined tolerance of 0.1\%. The hardware 
accelerated run should complete in less time than the non-accelerated run.

\textbf{Test Case Derivation:} 
Hardware acceleration uses specialized processing uits to perform expensive 
operations, like FFT or convolutions more efficiently than general-purpose. 
Verifying the reduced runtime and equivalent outputs confirms the module 
deployed on the hardware is functioning correctly. 

\textbf{How test will be performed:}
Manually running one confirguration on the microcontroller, and another on the 
local computer. Execution time will be measured with performance logs. Test 
function will be written to measure the numerical equivalence of both outputs 
after processing is completed. Logs will be manully inspected to verify the 
response time of the hardware accelerated mode is less than the non-accelerated 
mode.

\item{\textbf{test-FR-3.5} Flagging anomalies\\}

\textbf{Control:} Automatic
					
\textbf{Initial State:} 
The audio filtering module is initialized and ready to process audio input 
retrieved from an audio file. 
					
\textbf{Input:}
Three separate audio clips represented in audio files. One will have a 1 second 
sine wave with an amplitude that exceeds 1.0. Since amplitudes above 1.0 will 
become clipped. Another clip will have a 1 second sine wave, that is replaced 
by zeros halfway. This will test the lost signal case. The last clip will just 
have 2 secnds of zero amplitude, measuring the silence case. 
					
\textbf{Output:}
For each test case, the component should output the correct anomaly flag. 
In this case, for the first audio clip, it should output a clipping flag. 
For the second clip, it should output a lost signal flag. For the last clip, 
it should output a silence flag.

\textbf{Test Case Derivation:} 
Clipping occurs when the amplitude of a signal exceeds the maximum 
representable value (-1.0 to 1.0 for normalized audio). As such, for sine wave 
with an amplitude above 1.0, clipping will occur. A lost signal is detected 
when a section of the audio suddenly dropped to zero amplitude, which is the 
case in the second clip. Silence is detected when the entire audio clip has 
zero amplitude, which is the case in the last clip.

\textbf{How test will be performed:}
Each test file will be uploaded as an artifact in the automated testing 
framework. There will be a test case for each test file, measuring each of the 
anomalies mentioned above. THe test cases will trigger when a commit is made to 
any branch in the repository. The audio filtering module will return 1 output 
for each test case. Test will be verified by asserting whether the correct 
anomly is displayed for each audio file in each test case. 

\end{enumerate}

\subsubsection{Visualization Controller Tests}

This section covers the tests for ensuring the correct output is being created 
and sent from the visualization controller to the output display. Each test is 
associated with a functional requirement defined under section 
\hyperref[SRS-sec:FR6]{3.2.6} of the SRS. As such, each test will verify whether
 the system meets the associated functional requirement. 

\begin{enumerate}

\item{\textbf{test-FR-6.1} Notify direction of audio source \\}

\textbf{Control:} Manual
					
\textbf{Initial State:} 
The Visualization Controller module is deployed on the microcontroller and 
initialized. Drivers for output display are installed in microcontroller, 
and the microcontroller is connected to the output display. 
					
\textbf{Input:}
A mock audio source direction input, represented as the object taken by the 
Visualization Controller module. The object will include an angle parameter in 
degrees (0 to 360°), indicating the direction of the audio source relative to 
the user. This can be an arbitary angle, such as 0°, 90°, 180°, 270°. 
					
\textbf{Output:}
Corresponding visual indicator appears on the output display pointing in 
the same direction as the input angle. The visualization 
appears within 1 second of inputting the direction 
(\hyperref[SRS-sec:VC-3.2]{VC-3.2}) 

\textbf{Test Case Derivation:} 
When the audio system detects an incoming sound and reports its direction, 
the visualization controller must translate that information into a user-facing 
cue so that it may displayed on the output display. In this case, by sending an 
object that outlines the direction of audio, that direction must be formatted 
by the Visualization Controller so that it can be rendered on the output display.
 This confirms that signals are being correctly translated.
					
\textbf{How test will be performed:}
Simulate a directional events by mocking the Visualization Controller's input 
object with directions at 0°, 90°, 180°, 270°. Capture the output display 
output by visually seeing if the correct direction is visualized. The test 
passes if all simulated directions match the expected visual outputs and 
response time thresholds (read using microcontroller logs) are met. 
					
\item{\textbf{test-FR-6.2} Notify direction or classification failure\\}

\textbf{Control:} Manual
					
\textbf{Initial State:}
The Visualization Controller module is deployed on the microcontroller and 
initialized. Drivers for output display are installed in microcontroller, 
and the microcontroller is connected to the output display. 
					
\textbf{Input:}
2 mock audio source direction input, represented as the object taken by the 
Visualization Controller module. The first object's metadata will include a 
failure flag indicating that the direction of the audio source could not be
determined. The second object's metadata will include a failure flag indicating 
that the classification of the audio source could not be determined.
					
\textbf{Output:}
For the first input object, a visual indicator appears on the output display 
signifying that the direction of an audio source could not be determined. 
The second input object should produce a different visual indicator on the 
output display signifying that the classification of the audio source could not
be determined. 

\textbf{Test Case Derivation:} 
When the audio system fails to determine either the direction or classification,
 it will report that failure in the input object to the Visualization Controller.
 The Visualization Controller must then translate that failure information into 
a user-facing cue so that it may be displayed on the output display. This 
confirms that errors are correctly being processed and presented to the user. 
					
\textbf{How test will be performed:}
Simulate failure events by mocking the Visualization Controller's input object 
with 2 failure flags, one for direction failure, and one for classification in 
the object metadata. Capture the output display output by visually seeing if 
the correct failure indicators are visualized on the output display. The test 
passes if all simulated failure events match the expected visual outputs.

\end{enumerate}


\subsection{Tests for Nonfunctional Requirements}

This section covers system tests for the non-functional requirements (NFR) 
listed under section \hyperref[SRS-sec:S.2]{S.2} of the SRS. Each subsection 
will be focused on the NFR for a specific component will be verified through 
testing.

\subsubsection{Audio Filtering}
		

\begin{enumerate}

\item{\textbf{test-NFR3.1} Accurate frequency-domain translation\\}

\textbf{Type:} Non-Functional, Dynamic, Automatic
					
\textbf{Initial State:} 
The audio filtering module is initialized and ready to process audio input 
retrieved from an audio file. Reference implementation for true 
frequency-domain representation is available for comparison. 
					
\textbf{Input/Condition:} 
A 1-second sine wave with arbitrary frequency sampled at 16 kHz. Additional 
composite signals (white noise segments) may be used for robustness testing. 
					
\textbf{Output/Result:} 
The computed frequnecy-domain representation from the component should differ 
from the true spectrum by less than 10\% error across all frequency bins. 
					
\textbf{How test will be performed:} 
Upload the audio file and high-precision FFT reference file to the automated 
testing framework. Confgiure the test to run every time a commit is made to 
Git. When a commit is made, the test suite will feed the audio file into the 
audio filtering component. After retrieving the frequency-domain output, 
calculate the mean relative error between component's output and the reference 
spectrum using the following formula across all bins. If the mean is less than 
10\%, the test passes. 

\[
\text{Error} = \frac{\left|A_{\text{component}} - A_{\text{true}}\right|}{A_{\text{true}}} \times 100\%, 
\quad \forall A \in \text{Spectrum}
\]
					
\item{\textbf{test-NFR3.2} Handle different input signal sizes\\}

\textbf{Type:} Non-Functional, Dynamic, Manual
					
\textbf{Initial State:} 
The audio filtering component is deployed on the microcontroller, and ready to 
process audio input retrieved from an audio file. Logging has been 
implemented on the microcontroller to capture time taken for processing.
					
\textbf{Input/Condition:} 
Digital audio signals of varying sizes: 512, 1024, 2048 and 4096 frames, all 
sampled at 16 kHz. Each input contains an arbitary test signal (sine wave with 
arbitrary frequency). 
					
\textbf{Output/Result:} 
For each input size, the Audio Filtering component should process all frames 
without exceeding time constraints defined in \hyperref[SRS-NFR1_2]{NFR1.2}. 
					
\textbf{How test will be performed:} 
Manually upload each audio file to the microcontroller and trigger processing. 
Execution time will be measured using microcontroller logs. After processing
 is complete, logs will be manually inspected to verify the processing time
 for each input size meets the time constraints defined in the SRS.

\item{\textbf{test-NFR3.3} Accuracy of FFT calculation exceeds 90\%\\}

\textbf{Type:} Non-Functional, Dynamic, Manual
					
\textbf{Initial State:} 
The audio filtering component is deployed on the microcontroller, and ready to 
process continous audio retrieved from the environment using attatched 
microphones. Mechanism to output spectrogram data from microcontroller is 
available for future analysis.
					
\textbf{Input/Condition:} 
60 second continous audio from the environment sampled at 16 kHz. The audio 
should contain a mix of frequencies and amplitudes to simulate real-world 
conditions. This same audio will be processed simulatenously by a high-precision 
FFT reference implementation on a seperate laptop.
					
\textbf{Output/Result:} 
The spectrogram output from the microcontroller should match the accuracy of 
the reference implementation with at most 10\% relative error across all 
frequency bins. The following formula can be used to calculate the relative 
error is shown below. 

\[
\text{Error} = \frac{\left|A_{\text{component}} - A_{\text{true}}\right|}{A_{\text{true}}} \times 100\%, 
\quad \forall A \in \text{Spectrum}
\]

\textbf{How test will be performed:} 
Manually record 60 seconds of audio from the environment using microphones 
attatched to microcontroller. The same audio will be recorded on a seperate 
laptop for reference processing. After recording, both the microcontroller and 
laptop will output their respective spectrograms. The spectrograms will be
compared by calculating the mean relative error across all frequency bins
using the formula above. If the mean error is less than 10\%, the test passes

\end{enumerate}

\subsubsection{Visualization Controller}
		

\begin{enumerate}

\item{\textbf{test-NFR6.1} Display safety critical information first\\}

\textbf{Type:} Non-Functional, Dynamic, Manual
					
\textbf{Initial State:} 
The Visualization Controller module is deployed on the microcontroller and 
initialized. Drivers for output display are installed in microcontroller, 
and the microcontroller is connected to the output display. 
					
\textbf{Input/Condition:} 
3 mock audio sources, represented as the object taken by the Visualization 
Controller module. These sources will be sent simulatenously to the module. The 
object meta will have a parameter that outlines the priority of the audio 
sourcs. The first object will have the highest priority, the second object will 
have medium priority and the third object will have the lowest priority.

\textbf{Output/Result:} 
The output display should only visualize the highest priority audio source 
first. So in this case, the direction of the first object should be visualized 
on the output display, and the rest should be ignored. 
					
\textbf{How test will be performed:} 
Simulate multiple audio sources by mocking the Visualization Controller's input 
objects with different priority levels. Capture the output display output by 
visually seeing if only the highest priority direction is visualized on the 
output display. The test passes if the highest priority direction is the only
 one visualized.

					
\item{\textbf{test-NFR6.2} Present information in a non-intrusive manner\\}

\textbf{Type:} Non-Functional, Dynamic, Manual
					
\textbf{Initial State:} 
The Visualization Controller module is deployed on the microcontroller and 
initialized. Drivers for output display are installed in microcontroller, 
and the microcontroller is connected to the output display. 
					
\textbf{Input/Condition:} 
A series of mock audio source direction inputs, represented as the object taken 
by the Visualization Controller module. The object will include an angle 
parameter in degrees (0 to 360°), indicating the direction of the audio source 
relative to the user. These can be an arbitary angles.
					
\textbf{Output/Result:} 
Stakeholders verifies the non-obtrusive nature of the visualizations 
on the output display. The stakeholder should report that the visualizations do 
not obstruct their view or cause discomfort during typical usage scenarios.

\textbf{How test will be performed:} 
Conduct a controlled usability session with at least 5 stakeholders. Record 
quantiative feedback from stakeholders, each rating the non-obtrusiveness on a 
scale of 1 to 5 (1 being very obtrusive, 5 being very non-obtrusive). The test 
passes if the average rating across all stakeholders is at least 4.

\end{enumerate}

\subsection{Traceability Between Test Cases and Requirements}

\begin{table}[H]
\centering
\caption{Functional Requirements and Corresponding Test Sections}
\begin{tabular}{|l|l|}
\hline
\textbf{Test Section} & \textbf{Supported Requirement(s)} \\ \hline
Audio Filtering & FR-3.1, FR-3.2, FR-3.3, FR-3.4 \\ \hline
Visualization Controller & FR-6.1, FR-6.2 \\ \hline
\end{tabular}
\end{table}

\begin{table}[H]
\centering
\caption{Non-Functional Requirements and Corresponding Test Sections}
\begin{tabular}{|l|l|}
\hline
\textbf{Test Section} & \textbf{Supported Requirement(s)} \\ \hline
Audio Filtering & NFR-3.1, NFR-3.2, NFR-3.3 \\ \hline
Visualization Controller & NFR-6.1, NFR-6.2 \\ \hline
\end{tabular}
\end{table}

\section{Unit Test Description}

\wss{This section should not be filled in until after the MIS (detailed design
  document) has been completed.}

\wss{Reference your MIS (detailed design document) and explain your overall
philosophy for test case selection.}  

\wss{To save space and time, it may be an option to provide less detail in this section.  
For the unit tests you can potentially layout your testing strategy here.  That is, you 
can explain how tests will be selected for each module.  For instance, your test building 
approach could be test cases for each access program, including one test for normal behaviour 
and as many tests as needed for edge cases.  Rather than create the details of the input 
and output here, you could point to the unit testing code.  For this to work, you code 
needs to be well-documented, with meaningful names for all of the tests.}

\subsection{Unit Testing Scope}

\wss{What modules are outside of the scope.  If there are modules that are
  developed by someone else, then you would say here if you aren't planning on
  verifying them.  There may also be modules that are part of your software, but
  have a lower priority for verification than others.  If this is the case,
  explain your rationale for the ranking of module importance.}

\subsection{Tests for Functional Requirements}

\wss{Most of the verification will be through automated unit testing.  If
  appropriate specific modules can be verified by a non-testing based
  technique.  That can also be documented in this section.}

\subsubsection{Module 1}

\wss{Include a blurb here to explain why the subsections below cover the module.
  References to the MIS would be good.  You will want tests from a black box
  perspective and from a white box perspective.  Explain to the reader how the
  tests were selected.}

\begin{enumerate}

\item{test-id1\\}

Type: \wss{Functional, Dynamic, Manual, Automatic, Static etc. Most will
  be automatic}
					
Initial State: 
					
Input: 
					
Output: \wss{The expected result for the given inputs}

Test Case Derivation: \wss{Justify the expected value given in the Output field}

How test will be performed: 
					
\item{test-id2\\}

Type: \wss{Functional, Dynamic, Manual, Automatic, Static etc. Most will
  be automatic}
					
Initial State: 
					
Input: 
					
Output: \wss{The expected result for the given inputs}

Test Case Derivation: \wss{Justify the expected value given in the Output field}

How test will be performed: 

\item{...\\}
    
\end{enumerate}

\subsubsection{Module 2}

...

\subsection{Tests for Nonfunctional Requirements}

\wss{If there is a module that needs to be independently assessed for
  performance, those test cases can go here.  In some projects, planning for
  nonfunctional tests of units will not be that relevant.}

\wss{These tests may involve collecting performance data from previously
  mentioned functional tests.}

\subsubsection{Module ?}
		
\begin{enumerate}

\item{test-id1\\}

Type: \wss{Functional, Dynamic, Manual, Automatic, Static etc. Most will
  be automatic}
					
Initial State: 
					
Input/Condition: 
					
Output/Result: 
					
How test will be performed: 
					
\item{test-id2\\}

Type: Functional, Dynamic, Manual, Static etc.
					
Initial State: 
					
Input: 
					
Output: 
					
How test will be performed: 

\end{enumerate}

\subsubsection{Module ?}

...

\subsection{Traceability Between Test Cases and Modules}

\wss{Provide evidence that all of the modules have been considered.}
				

\newpage

\section{Appendix}

This is where you can place additional information.

\subsection{Symbolic Parameters}

The definition of the test cases will call for SYMBOLIC\_CONSTANTS.
Their values are defined in this section for easy maintenance.

\subsection{Usability Survey Questions?}

\wss{This is a section that would be appropriate for some projects.}

\newpage
\bibliographystyle{IEEEtran}
\bibliography{../../refs/References}

\newpage{}
\section*{Appendix --- Reflection}

\begin{enumerate}
  \item What went well while writing this deliverable?

  \textbf{Sathurshan:} The team had a good understanding of the system as in
  the areas of the system that have the most critical risk to the project and
  functionality to the product. As a result, it helped the team know whats tests
  should be prioritized.

  \textbf{Nirmal:} Having a clear understanding of the project requirements 
  from the SRS made it easier to derive test cases for various components. 
  Furthermore, after working on the POC implementation, I think I had a good 
  understanding of what artifacts can and will be used to test various 
  components. For example, uploading pre-existing test files to automate 
  testing in our pipeline.

  \textbf{Jay:} The team's previous work on the SRS really helped because we already 
  had a clear picture of what each component needed to do. 
  This made it straightforward to figure out what to test and how to test it.

  \textbf{Kalp:} What worked really well for this document was actually our 
  previous well written SRS document. Since we had already gone through the 
  process of writing the SRS document, we were able to hit the ground running 
  with the VnV plan. We were able to use the same structure and format for the 
  VnV plan as we did for the SRS document, which made it easier to write.
  Referencing the SRS document was also really easy to do since it was well 
  written, organized, and discussed. 

  \item What pain points did you experience during this deliverable, and how
    did you resolve them?

  \textbf{Sathurshan:} The team has packed with midterms and assignments from
  other courses which made working on this deliverable and capstone difficult.
  There wasn't a good resolution other than getting the team to work on sections
  of the deliverable when possible.

  \textbf{Nirmal:} Trying to prioritize working on this deliverable with other 
  commitments was very difficult. Especially since this deliverable was smaller 
  compared to the SRS fr example, it made it difficult to push myself to work 
  this in advance, since I thought other things from other courses were more 
  pressing at the time, and that I can probably do this closer to the deadline.

  \textbf{Jay:} Balancing this deliverable with other course work was really challenging. 
  I kept putting it off thinking I could do it later, but then other assignments kept piling up. 
  I resolved this by setting specific time blocks to work on this deliverable.

  \textbf{Kalp:} The only pain point that I experienced wasn't even related to
  the document itself, but rather the fact that we had a lot of content to cover
  in the document, but with a lot of other course work as well at this time of
  year. Having to focus on the upcoming PoC implementation, as well as dealing 
  with the midterm season made it quite difficult to focus on the VnV plan.

  \item What knowledge and skills will the team collectively need to acquire to
  successfully complete the verification and validation of your project?
  Examples of possible knowledge and skills include dynamic testing knowledge,
  static testing knowledge, specific tool usage, Valgrind etc.  You should look
  to identify at least one item for each team member.
  
  \textbf{Team respose:} The following are the knowledge and skills to perform
  verification and validation of the project:

  \begin{enumerate}
    \item gtest: the main testing tool for writing unit test for source code.
    \item Hardware debugging: There aren't many methods to debug on a
    microcontroller. Thus we need someone to investigate on how to debug our
    software on a microcontroller. If not possible, what other ways we can debug
    our software without the hardware.
    \item Integration testing: Testing the integration of the software on the
    hardware to verify it has been done correctly.
    \item Validating the product with the user. It is not intuitive at the
    moment on how we will know that the product addresses the user's problem
    effectively.
    \item Design verification requires an expert to ensure that the team's
    initial design is correct to minimize technical debt since there is not
    a lot of time left in this project.
  \end{enumerate}

  \item For each of the knowledge areas and skills identified in the previous
  question, what are at least two approaches to acquiring the knowledge or
  mastering the skill?  Of the identified approaches, which will each team
  member pursue, and why did they make this choice?

  \begin{enumerate}
    \item gtest:
    \item Hardware debugging: Kalp will be pursuing this experience as his lack
    of experience with hardware debugging was felt as a technical blocker in 
    his sklil set. He will be looking at how to debug hardware 
    on a microcontroller for the project, but also be reading into documentation
    and practicing good hardware debugging practices on his own personal 
    projects. 
    \item Integration testing: Sathurshan will be pursuing this as he has
    experience of integration testing. He has already acquired partial skills
    from industry and will acquire more by looking at how public GitHub projects
    that uses hardware performed integration testing.
    \item Validating the product with the user: Jay will be pursuing this since he has experience 
    with user research from his design background. He will work with the McMaster Sign Language Club to 
    conduct structured interviews and usability testing sessions
    \item Design verification: Nirmal will be pursuing this since he has 
    experience with design verification from previous internships and research 
    background. He will be looking at best practices for design, using design 
    principles and architecture styles as references to verify whether the 
    correct one has been applied. 
  \end{enumerate}
\end{enumerate}

\end{document}