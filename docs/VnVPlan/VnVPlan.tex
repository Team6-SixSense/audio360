\documentclass[12pt, titlepage]{article}

\usepackage{booktabs}
\usepackage{tabularx}
\usepackage{hyperref}
\hypersetup{
    colorlinks,
    citecolor=blue,
    filecolor=black,
    linkcolor=red,
    urlcolor=blue
}
\usepackage[round]{natbib}

\input{../Comments}
%% Common Parts

\newcommand{\progname}{Audio360} % PUT YOUR PROGRAM NAME HERE
\newcommand{\authname}{Team 6, SixSense
\\ Omar Alam
\\ Sathurshan Arulmohan
\\ Nirmal Chaudhari
\\ Kalp Shah
\\ Jay Sharma
} % AUTHOR NAMES                  

\usepackage{hyperref}
    \hypersetup{colorlinks=true, linkcolor=blue, citecolor=blue, filecolor=blue,
                urlcolor=blue, unicode=false}
    \urlstyle{same}
                                


\begin{document}

\title{System Verification and Validation Plan for \progname{}} 
\author{\authname}
\date{\today}
	
\maketitle

\pagenumbering{roman}

\section*{Revision History}

\begin{tabularx}{\textwidth}{p{3cm}p{2cm}X}
\toprule {\bf Date} & {\bf Version} & {\bf Notes}\\
\midrule
Date 1 & 1.0 & Notes\\
Date 2 & 1.1 & Notes\\
\bottomrule
\end{tabularx}

~\\
\wss{The intention of the VnV plan is to increase confidence in the software.
However, this does not mean listing every verification and validation technique
that has ever been devised.  The VnV plan should also be a \textbf{feasible}
plan. Execution of the plan should be possible with the time and team available.
If the full plan cannot be completed during the time available, it can either be
modified to ``fake it'', or a better solution is to add a section describing
what work has been completed and what work is still planned for the future.}

\wss{The VnV plan is typically started after the requirements stage, but before
the design stage.  This means that the sections related to unit testing cannot
initially be completed.  The sections will be filled in after the design stage
is complete.  the final version of the VnV plan should have all sections filled
in.}

\newpage

\tableofcontents

\listoftables
\wss{Remove this section if it isn't needed}

\listoffigures
\wss{Remove this section if it isn't needed}

\newpage

\section{Symbols, Abbreviations, and Acronyms}

\renewcommand{\arraystretch}{1.2}
\begin{tabular}{l l} 
  \toprule		
  \textbf{symbol} & \textbf{description}\\
  \midrule 
  T & Test\\
  \bottomrule
\end{tabular}\\

\wss{symbols, abbreviations, or acronyms --- you can simply reference the SRS
  \citep{SRS} tables, if appropriate}

\wss{Remove this section if it isn't needed}

\newpage

\pagenumbering{arabic}

This document ... \wss{provide an introductory blurb and roadmap of the
  Verification and Validation plan}

\section{General Information}

\subsection{Summary}

\wss{Say what software is being tested.  Give its name and a brief overview of
  its general functions.}

\subsection{Objectives}

\wss{State what is intended to be accomplished.  The objective will be around
  the qualities that are most important for your project.  You might have
  something like: ``build confidence in the software correctness,''
  ``demonstrate adequate usability.'' etc.  You won't list all of the qualities,
  just those that are most important.}

\wss{You should also list the objectives that are out of scope.  You don't have 
the resources to do everything, so what will you be leaving out.  For instance, 
if you are not going to verify the quality of usability, state this.  It is also 
worthwhile to justify why the objectives are left out.}

\wss{The objectives are important because they highlight that you are aware of 
limitations in your resources for verification and validation.  You can't do everything, 
so what are you going to prioritize?  As an example, if your system depends on an 
external library, you can explicitly state that you will assume that external library 
has already been verified by its implementation team.}

\subsection{Challenge Level and Extras}

\wss{State the challenge level (advanced, general, basic) for your project.
Your challenge level should exactly match what is included in your problem
statement.  This should be the challenge level agreed on between you and the
course instructor.  You can use a pull request to update your challenge level
(in TeamComposition.csv or Repos.csv) if your plan changes as a result of the
VnV planning exercise.}

\wss{Summarize the extras (if any) that were tackled by this project.  Extras
can include usability testing, code walkthroughs, user documentation, formal
proof, GenderMag personas, Design Thinking, etc.  Extras should have already
been approved by the course instructor as included in your problem statement.
You can use a pull request to update your extras (in TeamComposition.csv or
Repos.csv) if your plan changes as a result of the VnV planning exercise.}

\subsection{Relevant Documentation}

\wss{Reference relevant documentation.  This will definitely include your SRS
  and your other project documents (design documents, like MG, MIS, etc).  You
  can include these even before they are written, since by the time the project
  is done, they will be written.  You can create BibTeX entries for your
  documents and within those entries include a hyperlink to the documents.}

\citet{SRS}

\wss{Don't just list the other documents.  You should explain why they are relevant and 
how they relate to your VnV efforts.}

\section{Plan}

\wss{Introduce this section.  You can provide a roadmap of the sections to
  come.}

\subsection{Verification and Validation Team}

\wss{Your teammates.  Maybe your supervisor.
  You should do more than list names.  You should say what each person's role is
  for the project's verification.  A table is a good way to summarize this information.}

\subsection{SRS Verification}

\wss{List any approaches you intend to use for SRS verification.  This may
  include ad hoc feedback from reviewers, like your classmates (like your
  primary reviewer), or you may plan for something more rigorous/systematic.}

\wss{If you have a supervisor for the project, you shouldn't just say they will
read over the SRS.  You should explain your structured approach to the review.
Will you have a meeting?  What will you present?  What questions will you ask?
Will you give them instructions for a task-based inspection?  Will you use your
issue tracker?}

\wss{Maybe create an SRS checklist?}

\subsection{Design Verification}

\wss{Plans for design verification}

\wss{The review will include reviews by your classmates}

\wss{Create a checklists?}

\subsection{Verification and Validation Plan Verification}

\wss{The verification and validation plan is an artifact that should also be
verified.  Techniques for this include review and mutation testing.}

\wss{The review will include reviews by your classmates}

\wss{Create a checklists?}

\subsection{Implementation Verification}

\wss{You should at least point to the tests listed in this document and the unit
  testing plan.}

\wss{In this section you would also give any details of any plans for static
  verification of the implementation.  Potential techniques include code
  walkthroughs, code inspection, static analyzers, etc.}

\wss{The final class presentation in CAS 741 could be used as a code
walkthrough.  There is also a possibility of using the final presentation (in
CAS741) for a partial usability survey.}

\subsection{Automated Testing and Verification Tools}

\wss{What tools are you using for automated testing.  Likely a unit testing
  framework and maybe a profiling tool, like ValGrind.  Other possible tools
  include a static analyzer, make, continuous integration tools, test coverage
  tools, etc.  Explain your plans for summarizing code coverage metrics.
  Linters are another important class of tools.  For the programming language
  you select, you should look at the available linters.  There may also be tools
  that verify that coding standards have been respected, like flake9 for
  Python.}

\wss{If you have already done this in the development plan, you can point to
that document.}

\wss{The details of this section will likely evolve as you get closer to the
  implementation.}

\subsection{Software Validation}

\wss{If there is any external data that can be used for validation, you should
  point to it here.  If there are no plans for validation, you should state that
  here.}

\wss{You might want to use review sessions with the stakeholder to check that
the requirements document captures the right requirements.  Maybe task based
inspection?}

\wss{For those capstone teams with an external supervisor, the Rev 0 demo should 
be used as an opportunity to validate the requirements.  You should plan on 
demonstrating your project to your supervisor shortly after the scheduled Rev 0 demo.  
The feedback from your supervisor will be very useful for improving your project.}

\wss{For teams without an external supervisor, user testing can serve the same purpose 
as a Rev 0 demo for the supervisor.}

\wss{This section might reference back to the SRS verification section.}

\section{System Tests}

The following tests are designed to verify the functional and nonfunctional
requirements of the system, software and hardware components included.

\subsection{Tests for Functional Requirements}

The functional tests are organized by system component to ensure comprehensive
coverage of all requirements specified in the SRS. Each test area
corresponds to the component-based structure defined in Section S.1 and S.2 of
the SRS, with tests designed to verify that each component meets its specified
functional requirements under normal operating conditions.

\subsubsection{Audio Capture and Synchronization}

This subsection tests the microphone array's ability to capture and synchronize
audio signals as specified in FR1.2, FR7.1, and FR7.2.
		
\paragraph{Microphone Audio Capture}

\begin{enumerate}

\item{FT-AC-1: Synchronized Multi-Channel Capture\\}

Control: Automatic
					
Initial State: System powered on with all four microphones connected and
calibrated.
					
Input: Single impulse sound source (e.g., hand clap) positioned at a known
location within the detection range.
					
Output: Four synchronized audio waveforms captured from all microphones, each
containing the impulse signal with temporal alignment within margins of 10 microseconds.

Test Case Derivation: Based on FR1.2 (synchronized audio processing) and
verification criterion VC-1.1 from SRS Section S.6.1. Synchronization is
critical for accurate direction of arrival estimation.
					
How the test will be performed: Generate an acoustic impulse at a known position.
Record timestamps from all four microphone channels. Analyze the relative timing
between channels using cross-correlation. Verify that inter-channel timing
differences are within the margins of 10 microsecond tolerance.
					
\item{FT-AC-2: Sampling Rate Verification\\}

Control: Automatic
					
Initial State: System operational with microphone array active.
					
Input: Continuous white noise signal for 10 seconds.
					
Output: Captured audio buffers with consistent 44100 Hz sampling rate across all
channels, verified through timestamp analysis.

Test Case Derivation: Based on NFR7.1 (44100 Hz sampling rate) and FR7.1
(microphone digital conversion). The sampling rate must meet Nyquist criteria
for the human audible range (20 Hz - 20 kHz).

How the test will be performed: Capture 10 seconds of white noise. Analyze buffer
timestamps to calculate effective sampling rate. Verify that the measured
sampling rate is 44100 Hz margins of 100 Hz across all channels.

\item{FT-AC-3: Frequency Range Coverage\\}

Control: Manual
					
Initial State: System powered on with calibrated microphone array.
					
Input: Swept sine wave from 20 Hz to 20 kHz at constant amplitude.
					
Output: Captured audio signal demonstrating frequency response across the entire
human audible range with amplitude variation less than 10 dB.

Test Case Derivation: Based on FR7.2 (human audible range 20 Hz - 20 kHz) and
normal operating condition requirements from the SRS.

How the test will be performed: Play a logarithmic frequency sweep from 20 Hz to 20
kHz using a calibrated speaker. Capture the signal with all microphones. Perform
FFT analysis to verify frequency response coverage across the specified range.

\item{FT-AC-4: Continuous Operation\\}

Control: Automatic
					
Initial State: System initialized and ready for operation.
					
Input: Ambient room audio for 1 hour.
					
Output: Continuous audio capture with no buffer overruns, dropped samples, or
system crashes. Log files show zero errors.

Test Case Derivation: Based on verification criterion VC-1.3 from SRS Section
S.6.1. Long-duration stability is essential for real-world usage.

How the test will be performed: Run the system continuously for 1 hour while
monitoring system logs, memory usage, and buffer states. Verify no errors,
crashes, or data loss occurred during the test period.

\end{enumerate}

\subsubsection{Audio Filtering and Processing}

This subsection tests the audio filtering component's ability to process signals
as specified in FR3.1, FR3.2, FR3.3, and FR3.5.

\paragraph{Signal Processing}

\begin{enumerate}

\item{FT-AF-1: Time to Frequency Domain Conversion\\}

Control: Automatic
					
Initial State: System operational with audio filtering module initialized.
					
Input: Pure sine wave at 1000 Hz with known amplitude.
					
Output: Frequency domain representation showing a peak at 1000 Hz with amplitude
error less than 10\% from the expected value.

Test Case Derivation: Based on FR3.1 (frequency domain conversion) and NFR3.1
(less than 10\% error in frequency domain representation).

How the test will be performed: Generate a 1000 Hz sine wave with known amplitude.
Process through the audio filtering module. Verify that the FFT output shows a
peak at 1000 Hz margins of  10 Hz with amplitude matching the input within 10\% tolerance.

\item{FT-AF-2: Signal Normalization\\}

Control: Automatic
					
Initial State: Audio filtering module active.
					
Input: Audio signals with varying amplitudes (50\%, 100\%, 150\% of nominal
level).
					
Output: Normalized output signals with consistent relative amplitude
relationships.

Test Case Derivation: Based on FR3.2 (amplitude normalization of incoming and
outgoing signals).

How the test will be performed: Input three test signals at different amplitude
levels. Verify that the output signals are normalized while maintaining relative
amplitude ratios. Check that no clipping occurs.

\item{FT-AF-3: Spectral Leakage Filtering\\}

Control: Manual
					
Initial State: System ready with windowing functions configured.
					
Input: Pure tone signal with abrupt start and stop.
					
Output: Frequency domain representation with reduced spectral leakage compared
to unwindowed processing.

Test Case Derivation: Based on FR3.3 (spectral leakage reduction).

How the test will be performed: Process the same signal with and without spectral
leakage filtering. Compare the frequency domain representations. Verify that
filtered output shows narrower peaks with reduced sidelobe energy.

\end{enumerate}

\subsubsection{Direction of Arrival Estimation}

This subsection tests the system's ability to estimate sound source direction as
specified in FR5.2 and FR5.3.

\paragraph{DoA Estimation Accuracy}

\begin{enumerate}

\item{FT-DOA-1: Single Source Direction Estimation\\}

Control: Manual
					
Initial State: Calibrated microphone array in controlled environment.
					
Input: Sound source (1 kHz tone) positioned at known angles: 0 degrees, 45 degrees, 90 degrees, 135 degrees,
180 degrees, 225 degrees, 270 degrees, 315 degrees.
					
Output: Direction estimates within margins of 45 degrees of actual source position for at least
90\% of test cases.

Test Case Derivation: Based on FR5.2 (direction estimation), FR5.3 (angular
representation), NFR5.3 (maximum 45 degrees error), and VC-2.1 from SRS Section S.6.2.

How the test will be performed: Position calibrated speaker at each of the 8 test
angles. For each position, run 10 trials. Record estimated angles. Calculate
angular error for each trial. Verify that 90\% of trials have error less than or
 equal to 45 degrees.

\item{FT-DOA-2: 360 degrees Coverage\\}

Control: Manual
					
Initial State: System operational with DoA algorithm active.
					
Input: Sound sources at 12 equally spaced angles around the user (every 30 degrees).
					
Output: System provides directional estimates for all 12 positions across the
full 360 degrees horizontal plane.

Test Case Derivation: Based on verification criterion VC-2.4 (360 degrees range
coverage) from SRS Section S.6.2.

How the test will be performed: Position speaker at 12 angles (0 degrees, 30 degrees, 60 degrees, ...,
330 degrees). Verify that system produces direction estimates for all positions without
dead zones or failures.

\item{FT-DOA-3: Angular Representation Format\\}

Control: Automatic
					
Initial State: DoA module initialized.
					
Input: Sound source at 45 degrees relative to forward axis.
					
Output: Direction represented as angle in degrees (ex. 30 degrees left) measured
 from the forward axis as defined in SRS Figure showing coordinate system.

Test Case Derivation: Based on FR5.3 (angular representation in radians relative
to forward axis).

How the test will be performed: Position sound source at 45 degrees. Verify that the output
is approximately 45 degrees as well. Check that the reference frame matches the
coordinate system defined in the SRS.

\item{FT-DOA-4: Processing Latency\\}

Control: Automatic
					
Initial State: System running in real-time mode.
					
Input: Abrupt sound onset (hand clap).
					
Output: Direction estimate generated within 500 ms of sound detection.

Test Case Derivation: Based on verification criterion VC-2.2 (processing
latency) from SRS Section S.6.2.

How the test will be performed: Use synchronized high-speed cameras and
microphones. Generate acoustic impulse. Measure time from sound onset to
direction output. Verify latency less than or equal to 500 ms.

\end{enumerate}

\subsubsection{Sound Classification}

This subsection tests the audio classification capabilities as specified in
FR5.1 and FR5.4.

\paragraph{Classification Accuracy and Performance}

\begin{enumerate}

\item{FT-CL-1: Multi-Category Classification\\}

Control: Automatic
					
Initial State: Classification model loaded and operational.
					
Input: Labeled test dataset containing at least 100 samples across minimum 5
categories (speech, vehicle sounds, alarms, household appliances, other).
					
Output: Classification accuracy of at least 90\% on the test dataset.

Test Case Derivation: Based on FR5.1 (sound classification), NFR5.1 (minimum 3
distinct sources), NFR5.2 (90\% accuracy), and VC-3.1 from SRS Section S.6.3.

How the test will be performed: Prepare labeled test dataset with balanced samples
from each category. Run classification on all samples. Calculate overall
accuracy as (correct predictions / total samples). Verify accuracy greater than 
or equal to 90\%.

\item{FT-CL-2: Category Coverage\\}

Control: Manual
					
Initial State: Classification system ready.
					
Input: Representative samples from each of the required sound categories.
					
Output: System successfully identifies at least 5 distinct sound categories
relevant to deaf/hard-of-hearing users.

Test Case Derivation: Based on NFR5.1 (at least 3 distinct sound sources) and
VC-3.3 (minimum 5 categories).

How the test will be performed: Present one sample from each category. Verify that
the system can distinguish between all categories. Document which categories are
supported.

\item{FT-CL-3: Classification Latency\\}

Control: Automatic
					
Initial State: Real-time classification active.
					
Input: Sound onset with clear classification characteristics.
					
Output: Classification label generated within 1 second of sound detection.

Test Case Derivation: Based on VC-3.2 (classification latency) from SRS Section
S.6.3 and overall system latency requirements.

How the test will be performed: Use timestamped audio input. Measure time from sound
onset to classification output. Verify latency less than or equal to 1 second 
across multiple trials.

\item{FT-CL-4: Low Confidence Handling\\}

Control: Automatic
					
Initial State: Classification system operational.
					
Input: Ambiguous or unknown sound samples not in training set.
					
Output: System provides low-confidence indication or "unknown" label rather than
incorrect high-confidence classification.

Test Case Derivation: Based on FR5.4 (notification of low confidence or
unrecognized sounds) and VC-3.4 from SRS Section S.6.3.

How the test will be performed: Present sounds outside the training categories.
Verify that system either assigns low confidence score or returns "unknown"
classification. Confirm it does not confidently misclassify unknown sounds.

\end{enumerate}

\subsubsection{Visualization and Display}

This subsection tests the visual output system as specified in FR6.1, FR6.2,
FR8.1, and FR8.2.

\paragraph{Display Functionality}

\begin{enumerate}

\item{FT-VIZ-1: Direction Notification\\}

Control: Manual
					
Initial State: Smart glasses connected and display active.
					
Input: Sound detected at 90 degrees (left side).
					
Output: Visual indicator on glasses shows direction pointing to the left,
corresponding to 90 degrees.

Test Case Derivation: Based on FR6.1 (direction notification), FR8.2 (direction
display), and goal for visual display (Goal 4).

How the test will be performed: Generate sound at known direction. Observe display
output. Verify that directional indicator matches the actual sound source
location within the system's accuracy limits.

\item{FT-VIZ-2: Classification Display\\}

Control: Manual
					
Initial State: Display system ready.
					
Input: Classified sound (e.g., "vehicle").
					
Output: Display shows classification label "vehicle" alongside directional
indicator.

Test Case Derivation: Based on FR8.1 (classification display) and goal for audio
identification (Goal 3).

How the test will be performed: Play vehicle sound sample. Verify that display shows
both the classification ("vehicle") and direction simultaneously.

\item{FT-VIZ-3: End-to-End Latency\\}

Control: Manual
					
Initial State: Complete system operational.
					
Input: Abrupt sound source activation.
					
Output: Visual indication appears on display within 1 second of sound onset.

Test Case Derivation: Based on VC-4.1 (end-to-end latency less than or equal to 
1 second) from SRS Section S.6.4.

How the test will be performed: Use synchronized video recording. Generate sound
impulse. Measure time from sound onset to visual display appearance. Verify
total latency less than or equal to 1 second.

\item{FT-VIZ-4: Failure Alert Display\\}

Control: Manual
					
Initial State: System monitoring for failures.
					
Input: Simulated failure condition (e.g., microphone disconnection).
					
Output: Display shows alert indicating degraded functionality or feature
failure.

Test Case Derivation: Based on FR6.2 (failure alerts) and safety requirement
SR6.

How the test will be performed: Disconnect one microphone while system is running.
Verify that display shows appropriate warning or error message within 2 seconds.

\end{enumerate}

\subsubsection{System Reliability and Error Handling}

This subsection tests error handling capabilities as specified in FR1.3, FR1.4,
FR2.3, and FR4.4.

\paragraph{Error Detection and Recovery}

\begin{enumerate}

\item{FT-ERR-1: Memory Error Handling\\}

Control: Automatic
					
Initial State: System under normal load.
					
Input: Memory allocation request that would exceed available memory.
					
Output: System detects condition, returns error code, and continues operation
without crashing.

Test Case Derivation: Based on FR1.3 (memory error handling to prevent crashes).

How the test will be performed: Inject memory pressure by allocating buffers until
near limit. Attempt additional allocation. Verify system handles gracefully with
error return, not crash.

\item{FT-ERR-2: Hardware Diagnostics\\}

Control: Automatic
					
Initial State: All hardware components operational.
					
Input: Simulated hardware faults (one at a time): microphone failure, display
disconnection.
					
Output: Diagnostic system detects each fault within 1 second and logs
appropriate error.

Test Case Derivation: Based on FR1.4 (continuous hardware diagnostics) and SR10.

How the test will be performed: Inject each fault type. Monitor diagnostic logs.
Verify detection within 1 second. Check that error types are correctly
identified.

\item{FT-ERR-4: Error Code Propagation\\}

Control: Automatic
					
Initial State: Driver layer active.
					
Input: Hardware access request that fails (e.g., attempt to read from
disconnected peripheral).
					
Output: Driver layer returns error code immediately to calling module.

Test Case Derivation: Based on FR2.3 (error code returns) and NFR2.1 (immediate
error propagation).

How the test will be performed: Trigger hardware access failure. Measure time from
failure to error code return. Verify error code is correct and propagation time
< 10 ms.

\end{enumerate}

\subsubsection{Data Privacy and Security}

This subsection tests privacy and security requirements as specified in FR9.1
and NFR4.3.

\paragraph{Data Handling}

\begin{enumerate}

\item{FT-SEC-1: Closed System Verification\\}

Control: Manual
					
Initial State: Microcontroller powered and running main software.
					
Input: Attempt to connect unauthorized external device.
					
Output: Connection rejected, system remains isolated.

Test Case Derivation: Based on FR9.1 (closed environment with no external
connections except approved peripherals) and SR1.

How the test will be performed: Attempt to connect various unauthorized devices
(USB, network, Bluetooth). Verify all connections are rejected. Confirm only
microphones and display can interface.

\item{FT-SEC-2: Data Disposal\\}

Control: Automatic
					
Initial State: Audio processing pipeline active.
					
Input: Audio sample processed through classification and DoA estimation.
					
Output: Raw audio data is permanently discarded immediately after analysis
completes, with no trace in memory or storage.

Test Case Derivation: Based on NFR4.3 (immediate data disposal after analysis)
and SR2.

How the test will be performed: Process audio sample. Use memory inspection tools to
verify raw audio data is cleared from buffers immediately after processing.
Confirm no audio data persists beyond analysis stage.

\end{enumerate}

\subsection{Tests for Nonfunctional Requirements}

The nonfunctional tests verify performance, reliability, accuracy, and
usability requirements that define system quality attributes beyond basic
functionality. These tests focus on quantitative measurements and quality
metrics specified in the NFRs of the SRS. Where appropriate, tests
reference functional tests from Section 3.1 but with additional focus on
measuring and reporting specific quality metrics.

\subsubsection{Performance and Real-Time Constraints}

This subsection verifies that the system meets timing, throughput, and real-time
processing requirements as specified in NFR1.1, NFR1.2, NFR3.2, and NFR8.1.

\paragraph{Processing Speed and Latency}

\begin{enumerate}

\item{NFT-PERF-1: Real-Time Processing Capability\\}

Type: Dynamic, Automatic

Initial State: System operational under normal load.

Input/Condition: Continuous audio stream at 44100 Hz sampling rate for 5
minutes.

Output/Result: System processes all audio frames in real-time with frame
processing time consistently less than the sampling period (1/44100 seconds =
22.7 microseconds per sample). Timing logs show no frame drops or delayed
processing.

How the test will be performed: Generate continuous test audio at 44100 Hz. Log
processing timestamps for each frame. Calculate processing time distribution.
Verify that 100\% of frames are processed within their deadline. Generate timing
histogram showing margin above real-time requirements. This test validates
NFR1.2 (processing faster than 44100 Hz sample rate).

\item{NFT-PERF-2: Audio Filtering Scalability\\}

Type: Dynamic, Automatic

Initial State: Audio filtering module configured.

Input/Condition: Test signals with varying buffer sizes: 64, 256, 1024, 2048,
4096 frames.

Output/Result: Processing time measurements for each buffer size, with all sizes
up to 4096 frames meeting real-time constraints. Summary table showing
processing time vs. buffer size relationship.

How the test will be performed: Process test signals with each buffer size. Measure
and record processing time for each size. Generate performance curve showing
processing time vs. buffer size. Verify that maximum buffer size (4096 frames)
still meets timing constraints from NFR3.2.

\item{NFT-PERF-3: Display Update Rate\\}

Type: Dynamic, Manual

Initial State: Display system active with visualization controller running.

Input/Condition: Rapidly changing audio scene with multiple sound events per
second.

Output/Result: Display maintains minimum 30 updates per second as measured by
high-speed camera. Frame timing analysis shows consistent refresh rate.

How the test will be performed: Use high-speed camera (120 fps) to record display
output. Analyze video to measure actual display update frequency. Calculate
average and minimum update rates over 30-second test period. Verify compliance
with NFR8.1 (minimum 30 Hz update rate).

\item{NFT-PERF-4: End-to-End Latency Under Load\\}

Type: Dynamic, Manual

Initial State: Full system operational with all features enabled.

Input/Condition: Sound impulses generated at random intervals while system is
under typical processing load.

Output/Result: Average end-to-end latency (sound onset to display update) less 
than or equal to 1 second, with 95th percentile latency less than or equal 
to 1.2 seconds. Latency distribution histogram provided.

How the test will be performed: Generate 50 acoustic impulses at random intervals.
Use synchronized timing system to measure latency for each impulse. Calculate
mean, median, 95th percentile, and maximum latencies. Verify compliance with VC-
4.1 requirement.

\end{enumerate}

\subsubsection{Accuracy and Precision}

This subsection tests measurement accuracy requirements specified in NFR3.1,
NFR5.2, and NFR5.3.

\paragraph{Measurement Quality}

\begin{enumerate}

\item{NFT-ACC-1: Frequency Domain Accuracy\\}

Type: Dynamic, Automatic

Initial State: Audio filtering module operational.

Input/Condition: Set of 20 pure tone test signals at known frequencies (100 Hz,
500 Hz, 1 kHz, 2 kHz, 5 kHz, 10 kHz, 15 kHz) and amplitudes.

Output/Result: Frequency domain representation with < 10\% error for both
frequency and amplitude measurements. Error report showing deviation for each
test signal.

How the test will be performed: Process each test signal through frequency domain
conversion. Compare detected frequency and amplitude to known values. Calculate
relative error for each measurement. Generate error summary table. Verify all
errors < 10\% per NFR3.1.

\item{NFT-ACC-2: Classification Accuracy Validation\\}

Type: Dynamic, Automatic

Initial State: Classification system with trained model loaded.

Input/Condition: Balanced test dataset with 200 samples across 5+ sound
categories, recorded under normal operating conditions.

Output/Result: Overall classification accuracy greater than or equal to 90\%. Confusion matrix showing
per-category performance. Report includes precision and recall for each
category.

How the test will be performed: Run full test dataset through classifier. Generate
confusion matrix. Calculate overall accuracy, per-category precision and recall.
Verify overall accuracy meets NFR5.2 (90\% minimum). Document which categories
meet/exceed target and which may need improvement.

\item{NFT-ACC-3: Direction Estimation Error Distribution\\}

Type: Dynamic, Manual

Initial State: DoA system calibrated and operational.

Input/Condition: Sound sources at 24 known positions (every 15 degrees around user) in
controlled environment. 10 trials per position = 240 total measurements.

Output/Result: 90\% of measurements have angular error less than or equal to 45 degrees. Error distribution
histogram showing error characteristics. Mean absolute error and standard
deviation reported.

How the test will be performed: Position speaker at each test angle. Perform 10
trials per position. Calculate angular error for each trial. Generate error
histogram and statistics. Verify that 90\% of measurements meet NFR5.3 (less than or equal to 45 degrees
error). Identify any systematic biases or dead zones.

\end{enumerate}

\subsubsection{Reliability and Robustness}

This subsection tests system reliability requirements specified in NFR3.3 and
VC-6.1.

\paragraph{Continuous Operation and Error Rates}

\begin{enumerate}

\item{NFT-REL-1: Audio Processing Success Rate\\}

Type: Dynamic, Automatic

Initial State: Complete audio processing pipeline operational.

Input/Condition: Continuous operation for 60 seconds with typical audio input
(mixed speech, environmental sounds, occasional silence).

Output/Result: End-to-end audio processing success rate greater than or equal to 90\%. Failure log
documenting any dropped frames, processing errors, or anomalies.

How the test will be performed: Run system for 60 seconds with continuous audio
input. Monitor all processing stages for errors. Count successful processing
cycles vs. total cycles. Calculate success rate. Verify compliance with NFR3.3.
Document failure modes if success rate < 90\%.

\item{NFT-REL-2: Startup Reliability\\}

Type: Dynamic, Automatic

Initial State: System powered off.

Input/Condition: 20 power-on and initialization cycles.

Output/Result: Successful initialization and operational readiness in greater than or equal to 95\% of
trials (greater than or equal to 19 out of 20 successes).

How the test will be performed: Power cycle system 20 times. For each cycle, verify
successful initialization of all components. Log any initialization failures
with error details. Calculate success rate. Verify compliance with VC-6.1 from
SRS Section S.6.6.

\item{NFT-REL-3: Extended Operation Stability\\}

Type: Dynamic, Automatic

Initial State: System freshly initialized.

Input/Condition: Continuous operation for 8 hours under typical usage patterns
(varying sound environments, periodic silence).

Output/Result: No system crashes, memory leaks, or performance degradation.
Memory usage remains stable. Processing latency remains consistent throughout
test period.

How the test will be performed: Run system continuously for 8 hours. Monitor system
metrics (CPU usage, memory footprint, processing latency) every 5 minutes. Log
any anomalies, crashes, or performance degradation. Generate time-series plots
of key metrics. Verify stable operation throughout test period.

\end{enumerate}

\subsubsection{Usability and User Experience}

This subsection tests user-facing quality requirements specified in NFR6.1,
NFR6.2, and verification criteria from SRS Section S.6.

\paragraph{User Interaction Quality}

\begin{enumerate}

\item{NFT-USE-1: Display Visibility Test\\}

Type: Static, Manual

Initial State: Smart glasses configured with standard display settings.

Input/Condition: Evaluation in 6 lighting conditions: (1) indoor ambient, (2)
bright indoor, (3) dim indoor, (4) outdoor shade, (5) outdoor sun, (6) outdoor
bright sun.

Output/Result: Subjective visibility assessment from at least 3 test
participants. Visual indicators rated as "clearly visible" or better in at least
4 out of 6 lighting conditions.

How the test will be performed: Recruit 3 test participants (ideally including
target users if available). Display standard test patterns in each lighting
condition. Participants rate visibility on 5-point scale (1=invisible to
5=perfectly clear). Average ratings across participants. Verify compliance with
VC-4.2 (visible in typical indoor and outdoor conditions).

\item{NFT-USE-2: Visual Obstruction Measurement\\}

Type: Static, Manual

Initial State: Display showing maximum information load (direction +
classification + status).

Input/Condition: User wearing glasses and looking at target scene.

Output/Result: Visual indicators obstruct less than or equal to 20\% of field of view, measured
through image analysis of user's perspective.

How the test will be performed: Capture images from user's perspective with and
without display active. Use image processing to calculate percentage of view
obstructed by display elements. Verify compliance with VC-4.3 (less than or equal to 20\%
obstruction).

\item{NFT-USE-3: Setup Time Evaluation\\}

Type: Dynamic, Manual

Initial State: System powered off, new test user unfamiliar with device.

Input/Condition: 5 test participants asked to power on and begin using device
without instructions or documentation.

Output/Result: Average setup time less than or equal to 5 minutes across all participants. All
participants successfully begin receiving alerts without assistance.

How the test will be performed: Recruit 5 participants unfamiliar with system. Time
each participant from power-on to first successful alert reception. Record any
difficulties or confusion points. Calculate average time. Verify compliance with
VC-5.1 (5 minute setup time). Document any common usability issues for design
improvement.

\item{NFT-USE-4: Non-Intrusive Display Design\\}

Type: Static, Manual

Initial State: User wearing glasses while performing tasks.

Input/Condition: Participants perform standard activities (reading, walking,
conversation) while wearing active system.

Output/Result: Subjective assessment that display presentation is non-intrusive
and does not significantly interfere with external activities. Minimum 4/5
rating on non-intrusiveness scale from majority of participants.

How the test will be performed: Have 5 participants wear system while performing
typical activities. Collect feedback on display intrusiveness using structured
questionnaire (see Appendix). Calculate average rating. Verify alignment with
NFR6.2 (non-intrusive presentation). This test is particularly important for
safety-critical usage.

\end{enumerate}

\subsubsection{Static Code Quality and Review}

This subsection describes static testing activities including code reviews,
inspections, and walkthroughs.

\paragraph{Code Quality Verification}

\begin{enumerate}

\item{NFT-STATIC-1: Code Review Process\\}

Type: Static

Initial State: Code module submitted for review.

Input/Condition: All safety-critical modules (audio processing, DoA estimation,
classification, error handling) before integration.

Output/Result: Code review checklist completed by at least 2 team members.
All critical issues resolved before approval. Review report documenting findings
and resolutions.

How the test will be performed: Each safety-critical module undergoes peer review by
minimum 2 team members not directly involved in its development. Reviewers use
standardized checklist covering: (1) adherence to coding standards, (2) error
handling completeness, (3) memory safety, (4) timing constraint compliance, (5)
requirement traceability. All "major" issues must be resolved; "minor" issues
documented for future improvement.

\item{NFT-STATIC-2: Code Walkthrough Sessions\\}

Type: Static

Initial State: Major feature or subsystem implementation complete.

Input/Condition: Code walkthrough for each major subsystem: (1) Audio Filtering,
(2) DoA Estimation, (3) Classification, (4) Visualization Controller.

Output/Result: Walkthrough session notes documenting design rationale, potential
issues identified, and action items. Attended by all team members and supervisor.

How the test will be performed: Schedule formal walkthrough session for each
subsystem. Developer presents code design and implementation logic to team and
supervisor (Dr. Mohrenschildt). Team asks questions and identifies potential
issues. Document all findings and assign action items. Sessions provide learning
opportunity and help ensure code is maintainable and well-understood by entire
team.

\item{NFT-STATIC-3: Requirements Traceability Review\\}

Type: Static

Initial State: Implementation phase complete.

Input/Condition: All SRS functional and nonfunctional requirements.

Output/Result: Traceability matrix verified showing all requirements mapped to
implementation modules and test cases. 100\% of critical and important
requirements traced to code and tests.

How the test will be performed: Review implementation against SRS Section S.5
prioritization. Verify all Stage 1 (Critical) and Stage 2 (Important)
requirements have corresponding code modules and test cases. Use traceability
table from Section 3.3 to verify coverage. Document any gaps or missing
implementations with justification or remediation plan.

\end{enumerate}

\subsection{Traceability Between Test Cases and Requirements}

The following tables establish traceability between test cases defined in
Sections 3.1 and 3.2 and the functional and nonfunctional requirements specified
in the SRS. This traceability ensures that all requirements are
adequately verified through testing. The tables are organized by requirement
category to facilitate review and coverage analysis.

\subsubsection{Functional Requirements Traceability}

\noindent
\textbf{Table: Traceability - Functional Requirements to Test Cases}

\noindent
\begin{tabular}{|p{3cm}|p{8cm}|p{3cm}|}
\hline
\textbf{Requirement ID} & \textbf{Requirement Description} & \textbf{Test Cases} \\
\hline
FR1.2 & Firmware shall synchronize audio signals from all microphones & FT-AC-1 \\
\hline
FR1.3 & Firmware shall handle memory errors to prevent crashes & FT-ERR-1 \\
\hline
FR1.4 & System shall perform continuous hardware diagnostics & FT-ERR-2 \\
\hline
FR2.3 & Driver layer shall return error codes & FT-ERR-4 \\
\hline
FR3.1 & Audio filtering shall convert digital waveform to frequency domain & FT-AF-1 \\
\hline
FR3.2 & Audio filtering shall normalize signal amplitudes & FT-AF-2 \\
\hline
FR3.3 & Audio filtering shall reduce spectral leakage & FT-AF-3 \\
\hline
FR5.1 & Frequency analysis shall classify sound sources & FT-CL-1, FT-CL-2, FT-CL-4 \\
\hline
FR5.2 & Frequency analysis shall estimate direction of arrival & FT-DOA-1, FT-DOA-2, FT-DOA-4 \\
\hline
\end{tabular}

\vspace{1em}

\noindent
\begin{tabular}{|p{3cm}|p{8cm}|p{3cm}|}
\hline
\textbf{Requirement ID} & \textbf{Requirement Description} & \textbf{Test Cases} \\
\hline
FR5.3 & Frequency analysis shall represent DoA as angle in radians & FT-DOA-3 \\
\hline
FR5.4 & Frequency analysis shall notify users of low confidence classifications & FT-CL-4 \\
\hline
FR6.1 & Visualization controller shall notify users of sound direction & FT-VIZ-1 \\
\hline
FR6.2 & Visualization controller shall alert users of feature failures & FT-VIZ-4 \\
\hline
FR7.1 & Microphone shall collect soundwaves and translate to digital & FT-AC-2, FT-AC-3 \\
\hline
FR7.2 & Microphone shall collect human audible range (20 Hz - 20 kHz) & FT-AC-3 \\
\hline
FR8.1 & Output display shall notify users of classified sound source & FT-VIZ-2 \\
\hline
FR8.2 & Output display shall notify users of sound direction & FT-VIZ-1 \\
\hline
FR9.1 & Microcontroller shall run software in closed environment & FT-SEC-1 \\
\hline
\end{tabular}

\subsubsection{Nonfunctional Requirements Traceability}

\noindent
\textbf{Table: Traceability - Nonfunctional Requirements to Test Cases}

\noindent
\begin{tabular}{|p{3cm}|p{8cm}|p{3cm}|}
\hline
\textbf{Requirement ID} & \textbf{Requirement Description} & \textbf{Test Cases} \\
\hline
NFR1.2 & Firmware shall process audio faster than 44100 Hz sample rate & NFT-PERF-1 \\
\hline
NFR2.1 & Driver shall immediately propagate errors & FT-ERR-4 \\
\hline
NFR3.1 & Audio filtering shall represent frequency domain with < 10\% error & FT-AF-1, NFT-ACC-1 \\
\hline
NFR3.2 & Audio filtering shall scale to 4096 frames without missing timing constraints & NFT-PERF-2 \\
\hline
NFR3.3 & Audio filtering shall have 90\% success rate over 60 seconds & NFT-REL-1 \\
\hline
NFR4.3 & Audio360 Engine shall permanently discard audio data after analysis & FT-SEC-2 \\
\hline
NFR5.1 & Frequency analysis shall classify at least 3 distinct sound sources & FT-CL-2 \\
\hline
NFR5.2 & Frequency analysis shall achieve 90\% classification accuracy & FT-CL-1, NFT-ACC-2 \\
\hline
NFR5.3 & Frequency analysis shall estimate DoA with maximum 45 degrees error & FT-DOA-1, NFT-ACC-3 \\
\hline
NFR6.1 & Visualization controller shall prioritize safety-critical information & NFT-USE-4 \\
\hline
NFR6.2 & Visualization controller shall present information non-intrusively & NFT-USE-2, NFT-USE-4 \\
\hline
NFR7.1 & Microphone shall sample at 44100 Hz & FT-AC-2 \\
\hline
NFR8.1 & Output display shall operate at minimum 30 updates per second & NFT-PERF-3 \\
\hline
\end{tabular}

\subsubsection{Verification Criteria Traceability}

The following table maps verification criteria from SRS Section S.6 to their
corresponding test cases.

\noindent
\textbf{Table: Traceability - Verification Criteria to Test Cases}

\noindent
\begin{tabular}{|p{3cm}|p{8cm}|p{3cm}|}
\hline
\textbf{VC ID} & \textbf{Verification Criterion} & \textbf{Test Cases} \\
\hline
VC-1.1 & Synchronized sampling within margins of 10 microseconds & FT-AC-1 \\
\hline
VC-1.2 & Minimum 44.1 kHz sampling rate & FT-AC-2 \\
\hline
VC-1.3 & Continuous operation for 1 hour without errors & FT-AC-4 \\
\hline
VC-2.1 & DoA estimation within margins of 45 degrees for 90\% of test cases & FT-DOA-1, NFT-ACC-3 \\
\hline
VC-2.2 & DoA processing latency less than or equal to 500 ms & FT-DOA-4 \\
\hline
VC-2.4 & 360 degrees horizontal plane coverage & FT-DOA-2 \\
\hline
VC-3.1 & Classification accuracy greater than or equal to 90\% on test dataset & FT-CL-1, NFT-ACC-2 \\
\hline
VC-3.2 & Classification latency less than or equal to 1 second & FT-CL-3 \\
\hline
VC-3.3 & Support for greater than or equal to 5 distinct sound categories & FT-CL-2 \\
\hline
VC-3.4 & Graceful handling of unknown sounds & FT-CL-4 \\
\hline
VC-4.1 & End-to-end latency less than or equal to 1 second & FT-VIZ-3, NFT-PERF-4 \\
\hline
VC-4.2 & Display visible in typical lighting conditions & NFT-USE-1 \\
\hline
VC-4.3 & Visual indicators obstruct less than or equal to 20\% of field of view & NFT-USE-2 \\
\hline
VC-5.1 & Setup time less than or equal to 5 minutes for new users & NFT-USE-3 \\
\hline
VC-6.1 & System initialization success rate greater than or equal to 95\% & NFT-REL-2 \\
\hline
\end{tabular}

\subsubsection{Safety Requirements Traceability}

The following table maps safety and security requirements from the Hazard
Analysis to their corresponding test cases.

\noindent
\textbf{Table: Traceability - Safety Requirements to Test Cases}

\noindent
\begin{tabular}{|p{3cm}|p{8cm}|p{3cm}|}
\hline
\textbf{SR ID} & \textbf{Safety Requirement} & \textbf{Test Cases} \\
\hline
SR1 & Microcontroller shall run software in closed environment with only approved peripherals & FT-SEC-1 \\
\hline
SR2 & System shall permanently discard audio data after analysis & FT-SEC-2 \\
\hline
SR5 & Visualization shall be non-intrusive to allow safe external activities & NFT-USE-2, NFT-USE-4 \\
\hline
SR6 & Visualization shall alert users of feature failures & FT-VIZ-4 \\
\hline
SR7 & System shall achieve 90\% audio processing success rate over 60 seconds & NFT-REL-1 \\
\hline
SR9 & System shall estimate DoA with maximum 45 degrees error & FT-DOA-1, NFT-ACC-3 \\
\hline
SR10 & System shall perform continuous hardware diagnostics & FT-ERR-2 \\
\hline
\end{tabular}

\subsubsection{Goals Traceability}

The following table maps high-level goals from the SRS to test cases that
validate achievement of those goals.

\noindent
\textbf{Table: Traceability - Goals to Test Cases}

\noindent
\begin{tabular}{|p{3cm}|p{8cm}|p{4cm}|}
\hline
\textbf{Goal ID} & \textbf{Goal Description} & \textbf{Test Cases} \\
\hline
Goal 1 & Capture real-time audio from microphone array with synchronized sampling & FT-AC-1, FT-AC-2, FT-AC-3, FT-AC-4 \\
\hline
Goal 2 & Analyze audio to determine direction of arrival with minimal error and latency & FT-DOA-1, FT-DOA-2, FT-DOA-3, FT-DOA-4, NFT-ACC-3 \\
\hline
Goal 3 & Analyze audio to classify sound sources & FT-CL-1, FT-CL-2, FT-CL-3, FT-CL-4, NFT-ACC-2 \\
\hline
Goal 4 & Display audio classification and direction in real-time without obstructing view & FT-VIZ-1, FT-VIZ-2, FT-VIZ-3, NFT-USE-2, NFT-PERF-4 \\
\hline
Goal 5 & Provide user-friendly interaction with easy setup and understandable indicators & NFT-USE-3, NFT-USE-4 \\
\hline
\end{tabular}

\subsubsection{Coverage Analysis}

\textbf{Functional Requirements Coverage:} All 19 functional requirements
identified in the SRS are traced to at least one test case, providing 100\%
coverage of functional requirements.

\textbf{Nonfunctional Requirements Coverage:} All 13 key nonfunctional
requirements related to performance, accuracy, reliability, and usability are
traced to test cases. Coverage: 100\%.

\textbf{Verification Criteria Coverage:} All 15 verification criteria from SRS
Section S.6 are mapped to specific test cases, ensuring systematic validation
of acceptance criteria.

\textbf{Safety Requirements Coverage:} 9 out of 10 safety requirements from the
Hazard Analysis are covered by test cases. SR8 (low confidence notification) is
tested through FT-CL-4.

\textbf{Goals Coverage:} All 5 high-level project goals are validated through
multiple test cases, ensuring comprehensive coverage of stakeholder objectives.

\textbf{Priority Requirements Coverage:} All Stage 1 (Critical) and Stage 2
(Important) requirements from SRS Section S.5 are covered by test cases. Stage
3 (Desirable) requirements will be addressed based on implementation progress
and available resources.

\section{Unit Test Description}

\wss{This section should not be filled in until after the MIS (detailed design
  document) has been completed.}

\wss{Reference your MIS (detailed design document) and explain your overall
philosophy for test case selection.}  

\wss{To save space and time, it may be an option to provide less detail in this section.  
For the unit tests you can potentially layout your testing strategy here.  That is, you 
can explain how tests will be selected for each module.  For instance, your test building 
approach could be test cases for each access program, including one test for normal behaviour 
and as many tests as needed for edge cases.  Rather than create the details of the input 
and output here, you could point to the unit testing code.  For this to work, you code 
needs to be well-documented, with meaningful names for all of the tests.}

\subsection{Unit Testing Scope}

\wss{What modules are outside of the scope.  If there are modules that are
  developed by someone else, then you would say here if you aren't planning on
  verifying them.  There may also be modules that are part of your software, but
  have a lower priority for verification than others.  If this is the case,
  explain your rationale for the ranking of module importance.}

\subsection{Tests for Functional Requirements}

\wss{Most of the verification will be through automated unit testing.  If
  appropriate specific modules can be verified by a non-testing based
  technique.  That can also be documented in this section.}

\subsubsection{Module 1}

\wss{Include a blurb here to explain why the subsections below cover the module.
  References to the MIS would be good.  You will want tests from a black box
  perspective and from a white box perspective.  Explain to the reader how the
  tests were selected.}

\begin{enumerate}

\item{test-id1\\}

Type: \wss{Functional, Dynamic, Manual, Automatic, Static etc. Most will
  be automatic}
					
Initial State: 
					
Input: 
					
Output: \wss{The expected result for the given inputs}

Test Case Derivation: \wss{Justify the expected value given in the Output field}

How the test will be performed: 
					
\item{test-id2\\}

Type: \wss{Functional, Dynamic, Manual, Automatic, Static etc. Most will
  be automatic}
					
Initial State: 
					
Input: 
					
Output: \wss{The expected result for the given inputs}

Test Case Derivation: \wss{Justify the expected value given in the Output field}

How the test will be performed: 

\item{...\\}
    
\end{enumerate}

\subsubsection{Module 2}

...

\subsection{Tests for Nonfunctional Requirements}

\wss{If there is a module that needs to be independently assessed for
  performance, those test cases can go here.  In some projects, planning for
  nonfunctional tests of units will not be that relevant.}

\wss{These tests may involve collecting performance data from previously
  mentioned functional tests.}

\subsubsection{Module ?}
		
\begin{enumerate}

\item{test-id1\\}

Type: \wss{Functional, Dynamic, Manual, Automatic, Static etc. Most will
  be automatic}
					
Initial State: 
					
Input/Condition: 
					
Output/Result: 
					
How the test will be performed: 
					
\item{test-id2\\}

Type: Functional, Dynamic, Manual, Static etc.
					
Initial State: 
					
Input: 
					
Output: 
					
How the test will be performed: 

\end{enumerate}

\subsubsection{Module ?}

...

\subsection{Traceability Between Test Cases and Modules}

\wss{Provide evidence that all of the modules have been considered.}
				
\bibliographystyle{plainnat}

\bibliography{../../refs/References}

\newpage

\section{Appendix}

This is where you can place additional information.

\subsection{Symbolic Parameters}

The definition of the test cases will call for SYMBOLIC\_CONSTANTS.
Their values are defined in this section for easy maintenance.

\subsection{Usability Survey Questions?}

\wss{This is a section that would be appropriate for some projects.}

\newpage{}
\section*{Appendix --- Reflection}

\wss{This section is not required for CAS 741}

The information in this section will be used to evaluate the team members on the
graduate attribute of Lifelong Learning.

\input{../Reflection.tex}

\begin{enumerate}
  \item What went well while writing this deliverable? 
  \item What pain points did you experience during this deliverable, and how
    did you resolve them?
  \item What knowledge and skills will the team collectively need to acquire to
  successfully complete the verification and validation of your project?
  Examples of possible knowledge and skills include dynamic testing knowledge,
  static testing knowledge, specific tool usage, Valgrind etc.  You should look to
  identify at least one item for each team member.
  \item For each of the knowledge areas and skills identified in the previous
  question, what are at least two approaches to acquiring the knowledge or
  mastering the skill?  Of the identified approaches, which will each team
  member pursue, and why did they make this choice?
\end{enumerate}

\end{document}