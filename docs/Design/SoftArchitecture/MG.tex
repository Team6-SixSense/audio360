\documentclass[12pt, titlepage]{article}

\usepackage{fullpage}
\usepackage{multirow}
\usepackage{booktabs}
\usepackage{tabularx}
\usepackage{graphicx}
\usepackage{float}
\usepackage{hyperref}
\usepackage{longtable}
\usepackage{xr}
\usepackage{cite}
\hypersetup{ colorlinks, citecolor=blue, filecolor=black, linkcolor=red,
    urlcolor=blue }

\input{../../Comments}
%% Common Parts

\newcommand{\progname}{Audio360} % PUT YOUR PROGRAM NAME HERE
\newcommand{\authname}{Team 6, SixSense
\\ Omar Alam
\\ Sathurshan Arulmohan
\\ Nirmal Chaudhari
\\ Kalp Shah
\\ Jay Sharma
} % AUTHOR NAMES                  

\usepackage{hyperref}
    \hypersetup{colorlinks=true, linkcolor=blue, citecolor=blue, filecolor=blue,
                urlcolor=blue, unicode=false}
    \urlstyle{same}
                                


\externaldocument[SRS-]{../../SRS/SRS}

\newcounter{acnum}
\newcommand{\actheacnum}{AC\theacnum}
\newcommand{\acref}[1]{AC\ref{#1}}

\newcounter{ucnum}
\newcommand{\uctheucnum}{UC\theucnum}
\newcommand{\uref}[1]{UC\ref{#1}}

\newcounter{mnum}
\newcommand{\mthemnum}{M\themnum}
\newcommand{\mref}[1]{M\ref{#1}}

\begin{document}

\title{Module Guide for \progname{}} 
\author{\authname}
\date{\today}

\maketitle

\pagenumbering{roman}

\section{Revision History}

\begin{tabularx}{\textwidth}{p{3cm}p{2cm}X} \toprule {\bf Date} & {\bf Version}
& {\bf Notes}\\
\midrule
2025-11-13 & 1.0 & Initial write-up\\
\bottomrule
\end{tabularx}

\newpage

\section{Reference Material}

This section records information for easy reference.

\subsection{Symbols, Abbreviations and Acronyms}

\renewcommand{\arraystretch}{1.2}

\begin{table}[H]
  \centering
  \begin{tabular}{l l} 
    \toprule		
    \textbf{symbol} & \textbf{description}\\
    \midrule 
    AC & Anticipated Change\\
    \progname & 360 Audio analysis system on smart glasses\\
    DAG & Directed Acyclic Graph \\
    DOA & Direction of Arrival \\
    FR & Functional Requirement\\
    HUD & Heads Up Display \\
    ICA & Independent Component Analysis \\
    M & Module \\
    MG & Module Guide \\
    NFR & Non-functional Requirement\\
    OS & Operating System \\
    R & Requirement\\
    SC & Scientific Computing \\
    SPI & Serial Peripheral Interface \\
    SRS & Software Requirements Specification\\
    UC & Unlikely Change \\
    USART & Universal Synchronous/Asynchronous Receiver/Transmitter \\
    \bottomrule
  \end{tabular}\\
  \caption{Symbols, abbreviations and acronyms used in the MG document.}
\end{table}

See SRS Documentation at
\hyperref[SRS-sec:symbols]{Symbols, Abbreviations, and Acronyms}
for a complete table used in \progname.

\newpage

\tableofcontents

\listoftables

\listoffigures

\newpage

\pagenumbering{arabic}

\section{Introduction}

Decomposing a system into modules is a commonly accepted approach to developing
software.  A module is a work assignment for a programmer or programming
team~\cite{ParnasEtAl1984}.  We advocate a decomposition based on the principle
of information hiding~\cite{Parnas1972a}.  This principle supports design for
change, because the ``secrets'' that each module hides represent likely future
changes.  Design for change is valuable in SC, where modifications are frequent,
especially during initial development as the solution space is explored.  

Our design follows the rules laid out by \cite{ParnasEtAl1984}, as follows:
\begin{itemize}
\item System details that are likely to change independently should be the
  secrets of separate modules.
\item Each data structure is implemented in only one module.
\item Any other program that requires information stored in a module's data
  structures must obtain it by calling access programs belonging to that module.
\end{itemize}

After completing the first stage of the design, the Software Requirements
Specification (SRS), the Module Guide (MG) is developed~\cite{ParnasEtAl1984}.
The MG specifies the modular structure of the system and is intended to allow
both designers and maintainers to easily identify the parts of the software.
The potential readers of this document are as follows:

\begin{itemize}
\item New project members: This document can be a guide for a new project member
  to easily understand the overall structure and quickly find the relevant
  modules they are searching for.
\item Maintainers: The hierarchical structure of the module guide improves the
  maintainers' understanding when they need to make changes to the system. It is
  important for a maintainer to update the relevant sections of the document
  after changes have been made.
\item Designers: Once the module guide has been written, it can be used to check
  for consistency, feasibility, and flexibility. Designers can verify the system
  in various ways, such as consistency among modules, feasibility of the
  decomposition, and flexibility of the design.
\end{itemize}

The rest of the document is organized as follows. Section \ref{SecChange} lists
the anticipated and unlikely changes of the software requirements. Section
\ref{SecMH} summarizes the module decomposition that was constructed according
to the likely changes. Section \ref{SecConnection} specifies the connections
between the software requirements and the modules. Section \ref{SecMD} gives a
detailed description of the modules. Section \ref{SecTM} includes two
traceability matrices. One checks the completeness of the design against the
requirements provided in the SRS. The other shows the relation between
anticipated changes and the modules. Section \ref{SecUse} describes the use
relation between modules.

\section{Anticipated and Unlikely Changes} \label{SecChange}

This section lists possible changes to the system. According to the likeliness
of the change, the possible changes are classified into two categories.
Anticipated changes are listed in Section \ref{SecAchange}, and unlikely changes
are listed in Section \ref{SecUchange}.

\subsection{Anticipated Changes} \label{SecAchange}

Anticipated changes are the source of the information that is to be hidden
inside the modules. Ideally, changing one of the anticipated changes will only
require changing the one module that hides the associated decision. The approach
adapted here is called design for change.

\begin{description}
\item[\refstepcounter{acnum} \actheacnum \label{acHardware}:]
  The specific hardware on which the software is running. \\
  \textbf{Rationale:} There will be a benefit to using a hardware that is
  smaller (better integration with glasses), cheaper (reduce costs for user),
  and stronger compute (allows for further advanced audio processing), if one
  exists in the market.

\item[\refstepcounter{acnum} \actheacnum \label{acNumMic}:] 
  The number of microphones in microphone array. \\
  \textbf{Rationale:} Depending on the performance on audio classification and
  directional analysis features, more or less microphones may be required than
  what is specified \hyperref[SRS-FR9_2]{FR9.2}.

\item[\refstepcounter{acnum} \actheacnum \label{acMicArrangement}:] 
  The arrangement of the microphones relative to each other. \\
  \textbf{Rationale:} The system may be deployed on devices other than glasses,
  such as a hat. As a result, the microphone arrangement can change.

\item[\refstepcounter{acnum} \actheacnum \label{acOutput}:]
  The output display hardware. At the moment, it is Rokid Glasses
  \cite{RokidGlasses}, which projects visuals onto glasses lens. Display may be
  changed to a simpler screen. \\
  \textbf{Rationale:} Due to Rokid Glasses availability and cost, it may be
  swapped with a cheaper and easily available display that can be integrated
  easily onto a glasses frame.

\item[\refstepcounter{acnum} \actheacnum \label{NumClassification}:]
  The number of classification of audio sources. \\
  \textbf{Rationale:} \hyperref[SRS-NFR5_1]{NFR5.1} specifies at least 3
  distinct audio classification of audio sources required. This can be changed
  to detect a single audio classification if detecting one source is more
  critical.

\item[\refstepcounter{acnum} \actheacnum \label{SampleRate}:]
  The sample rate that audio is sampled from the environment from the
  environment. \\
  \textbf{Rationale:} \hyperref[SRS-NFR7_1]{NFR7.1} states to sample at 16 kHz
  as that is the minimum to analyze audible sounds to human. However, the scope
  can expand to detect sounds outside the human audible range to notify users of
  other critical sounds that human would have never been able to process on
  their own.

\item[\refstepcounter{acnum} \actheacnum \label{SampleSize}:]
  The sample size of audio data to analyze each time step. Currently specified
  to be up to 4096 samples from \hyperref[SRS-NFR3_3]{NFR3.2}. \\
  \textbf{Rationale:} For higher quality of audio processing, an higher sample
  rate is required.

\item[\refstepcounter{acnum} \actheacnum \label{acVizContent}:] 
  The visualization content on the output display. \\
  \textbf{Rationale:} Stakeholders may require alternative information or a
  different presentation format that is more effective and appropriate for the
  display. This is related to \hyperref[SRS-NFR6_2]{NFR6.2}

\item[\refstepcounter{acnum} \actheacnum \label{acHardwareAccel}:]
  Traditional algorithms may be used over
  \hyperref[SRS-def:hardware_acceleration]{hardware acceleration}. \\
  \textbf{Rationale:} Hardware acceleration can consume more power than
  traditional software algorithms. The trade-off of slower processing over power
  consumption may be desirable.

\item[\refstepcounter{acnum} \actheacnum \label{acConfidenceAction}:]
  Action when low confidence occurs during audio source classification. \\
  \textbf{Rationale:} According to \hyperref[SRS-FR5_4]{FR5.4}, the system shall
  notify user when results from classification is low confidence. However, this
  may be disregarded if the accuracy of the classification module is near
  perfect. 

\item[\refstepcounter{acnum} \actheacnum \label{acAudioGen}:] The method of
  generating simulated microphone array input for testing and proof of concept.
  \\
  \textbf{Rationale:} Since the use case is primarily for testing and proof of
  concept, the team will be prioritize methods that are easily integrable and
  provide sufficient functionality of testing. This can change as the team
  experiments with different methods at earlier stages. The final product will
  replace module linked to this anticipated change entirely with hardware
  microphone drivers. See table \ref{TblACT} for linked modules.

\item[\refstepcounter{acnum} \actheacnum \label{acSDCardLogging}:]
  The SD card logging method for diagnostic audio data.\\
  \textbf{Rationale:} The SD card diagnostic method is useful for bulk data
  logging during testing. In a production system it is going to be removed since
  no debug logs will be captured.

\item[\refstepcounter{acnum} \actheacnum \label{acICA}:]
  The implementation of Independent Component Analysis (ICA) for separating
  multiple simultaneous audio sources from mixed signals. \\
  \textbf{Rationale:} The SRS mentions multi-source handling as a functionality
  goal (Section G.4), where the system should detect and track multiple
  simultaneous sound sources. ICA would enable the extraction of individual
  audio sources from mixed microphone signals, improving classification and
  direction estimation accuracy in complex acoustic environments. However it is
  complicated to implement and is defined as a stretch goal by the team.
\end{description}

\subsection{Unlikely Changes} \label{SecUchange}

The module design should be as general as possible. However, a general system is
more complex. Sometimes this complexity is not necessary. Fixing some design
decisions at the system architecture stage can simplify the software design. If
these decision should later need to be changed, then many parts of the design
will potentially need to be modified. Hence, it is not intended that these
decisions will be changed.

\begin{description}
\item[\refstepcounter{ucnum} \uctheucnum \label{ucInput}:]
  The input devices for providing audio data from the environment are
  microphones. \\
 \textbf{Rationale:} Cheapest and easiest method of getting real-time audio from
 the environment of the user. Although,~\acref{acAudioGen} mentions the
 potential of changing inputs for testing and proof of concept, \uref{ucInput}
 is focused on the final product.

\item[\refstepcounter{ucnum} \uctheucnum \label{ucClosed}:]
  The processor works as a closed system. That is no external devices other than
  the microphones and display can be connected, and all processing occurs on the
  processor (processing is not offloaded to another device). \\
 \textbf{Rationale:} Adheres to user \hyperref[SRS-goal:user_safety]{Goal 7},
 user safety and privacy, as well as \hyperref[SRS-FR9_1]{FR9.1}.

\item[\refstepcounter{ucnum} \uctheucnum \label{ucMicSych}:]
  Microphones must always be synchronized at each sample.\\
 \textbf{Rationale:} Microphone synchronization is a requirement for directional
 analysis, which a core feature. Relates to \hyperref[SRS-FR1_2]{FR1.2:}

\item[\refstepcounter{ucnum} \uctheucnum \label{ucHardwareAbstraction}:]
  The core features interact with hardware through an hardware abstraction
  layer.\\
 \textbf{Rationale:} This allows the achieve \hyperref[acHardware]{hardware
 antricipated change} without system redesign.

\end{description}

\section{Module Hierarchy} \label{SecMH}

This section provides an overview of the module design. Modules are summarized
in a hierarchy decomposed by secrets in Table \ref{TblMH}. The modules listed
below, which are leaves in the hierarchy tree, are the modules that will
actually be implemented.

\begin{description}
\item [\refstepcounter{mnum} \mthemnum \label{mMicInput}:] Microphone Input
  Module
\item [\refstepcounter{mnum} \mthemnum \label{mUSART}:] USART Communication
  Module
\item [\refstepcounter{mnum} \mthemnum \label{mSDInterface}:] SD Card Interface
  Module
\item [\refstepcounter{mnum} \mthemnum \label{mMicDiag}:] Microphone Diagnostic
  Module 
\item [\refstepcounter{mnum} \mthemnum \label{mAudioGen}:] Audio Generation
  Module
\item [\refstepcounter{mnum} \mthemnum \label{mDOA}:] DOA (Direction of Arrival)
  Processor Module
\item [\refstepcounter{mnum} \mthemnum \label{mAudio360Engine}:] Audio360 Engine
  Module
\item [\refstepcounter{mnum} \mthemnum \label{mAudioSampling}:] Audio Sampling
  Module
\item [\refstepcounter{mnum} \mthemnum \label{mAudioSpectralLeakage}:] Audio
  Spectral Leakage Prevention Module
\item [\refstepcounter{mnum} \mthemnum \label{mAudioNormalizer}:] Audio
  Normalizer Module
\item [\refstepcounter{mnum} \mthemnum \label{mAudioAnomaly}:] Audio Anomaly
  Detection Module
\item [\refstepcounter{mnum} \mthemnum \label{mFFT}:] Fast Fourier Transform
  Module 
\item [\refstepcounter{mnum} \mthemnum \label{mLogging}:] Logging Module
\item [\refstepcounter{mnum} \mthemnum \label{mFaultManager}:] Fault Manager
  Module 
\item [\refstepcounter{mnum} \mthemnum \label{mMelFilter}:] Mel Filter Module
\item [\refstepcounter{mnum} \mthemnum \label{mDCT}:] Discrete Cosine Transform
Module
\item [\refstepcounter{mnum} \mthemnum \label{mPCA}:] Principle Component
Analysis Module
\item [\refstepcounter{mnum} \mthemnum \label{mLDA}:] Linear Discriminant
Analysis Module
\item [\refstepcounter{mnum} \mthemnum \label{mClassification}:] Classification
  Module
\item [\refstepcounter{mnum} \mthemnum \label{mVisualization}:] Visualization
  Module
\item [\refstepcounter{mnum} \mthemnum \label{mICA}:] Independent Component
  Analysis (ICA) Module
\end{description}


\begin{table}[H]
\centering
\begin{tabular}{p{0.3\textwidth} p{0.6\textwidth}}
\toprule
\textbf{Level 1} & \textbf{Level 2}\\
\midrule

\multirow{4}{0.3\textwidth}{Hardware-Hiding Module} & \mref{mMicInput}\\ 
& \mref{mUSART} \\
& \mref{mSDInterface} \\
& \mref{mMicDiag}\\
\midrule

\multirow{1}{0.3\textwidth}{Behaviour-Hiding Module} & \mref{mAudio360Engine}\\
& \mref{mVisualization}\\
\midrule

\multirow{16}{0.3\textwidth}{Software Decision Module} & \mref{mAudioGen}\\
& \mref{mDOA}\\
& \mref{mAudioSampling}\\
& \mref{mAudioSpectralLeakage}\\
& \mref{mAudioNormalizer}\\
& \mref{mAudioAnomaly}\\
& \mref{mFFT}\\
& \mref{mLogging}\\
& \mref{mFaultManager}\\
& \mref{mMelFilter}\\
& \mref{mDCT}\\
& \mref{mPCA}\\
& \mref{mLDA}\\
& \mref{mClassification}\\
& \mref{mICA}\\
\bottomrule

\end{tabular}
\caption{Module Hierarchy}
\label{TblMH}
\end{table}

\section{Connection Between Requirements and Design} \label{SecConnection}

The design of the system is intended to satisfy the requirements developed in
the SRS. In this stage, the system is decomposed into modules. The connection
between requirements and modules is listed in Table~\ref{TblRT}.

\begin{enumerate}

\item \hyperref[SRS-NFR2_1]{NFR2.1} and \hyperref[SRS-FR1_4]{FR1.4} specify that
the system shall detect and propagate faults to the embedded firmware layer for
appropriate handling. A software fault manager module, \mref{mFaultManager}, is
responsible for monitoring and reacting to software error states. In addition,
the \mref{mMicDiag} module monitors the microphone array for hardware faults.

\item \hyperref[SRS-NFR3_3]{NFR3.2} specifies that the system shall support
  different audio sample sizes. Sample size directly affects both the speed and
  quality of audio analysis. Consequently, a dedicated module,
  \mref{mAudioSampling}, is responsible for managing this.

\item \hyperref[SRS-FR3_4]{FR3.4} recommends the use of
  \hyperref[SRS-def:hardware_acceleration]{hardware acceleration} to leverage
  hardware optimizations. However, there may be cases where traditional software
  algorithms perform more efficiently, such as when processing smaller datasets.
  Therefore, specific computational tasks are encapsulated within dedicated
  modules that determine whether to use hardware acceleration based on the input
  characteristics. This design consideration affects
  \mref{mAudioSpectralLeakage}, \mref{mAudioNormalizer}, and \mref{mFFT}.

\item \hyperref[SRS-FR3_5]{FR3.5} requires the system to detect anomalies in the
audio stream. To address this, a dedicated module, \mref{mAudioAnomaly}, is
responsible for monitoring the audio data and identifying anomalies such as
clipping or dropouts.

\item \hyperref[SRS-FR4_5]{FR4.5} requires that the software be portable across
  different hardware platforms. This directly influenced the design decision to
  abstract the Audio360 Engine from the hardware layer. Inputs to the Audio360
  Engine are maintained in a generic format, enabling deployment on various
  hardware targets, including simulation environments.

\item \hyperref[SRS-FR5_1]{FR5.1}, \hyperref[SRS-FR5_4]{FR5.4} requires the
  frequency analysis component to classify audio sources. The classification
  will be done by performing Principle Component Analysis (PCA) on extracted
  features from the Mel-Frequency domain and performing Linear Discriminant
  Analysis (LDA) on components to predict the classification. Since this is a
  multi-step process, specific computational tasks are encapsulated within
  de-coupled modules that will be called individually. This design affects
  \mref{mMelFilter}, \mref{mDCT}, \mref{mPCA}, \mref{mLDA} and
  \mref{mClassification}. 

\item \hyperref[SRS-FR5_2]{FR5.2}, \hyperref[SRS-NFR5_3]{NFR5.3} requires
  determining direction of sound source in radians (\hyperref[SRS-FR5_3]{FR5.3})
  within a specific error threshold. As a result, DOA processor module
  \mref{mDOA} is responsible for implementing these requirements. This modular
  design facilitates testing to ensure that the specified error threshold is
  satisfied.

\item \hyperref[SRS-FR6_1]{FR6.1}, \hyperref[SRS-FR6_2]{FR6.2},
  \hyperref[SRS-FR8_1]{FR8.1}, and \hyperref[SRS-FR8_2]{FR8.2} specify that the
  system shall display sound classification and direction information to users.
  \hyperref[SRS-NFR6_1]{NFR6.1} requires prioritization of safety-critical
  information, while \hyperref[SRS-NFR6_2]{NFR6.2} mandates non-intrusive
  presentation. \hyperref[SRS-NFR8_1]{NFR8.1} requires a minimum of 30 display
  updates per second for low latency. To satisfy these requirements, a dedicated
  visualization module is responsible for receiving classification and direction
  data from the Audio360 Engine, formatting the information according to display
  constraints, and managing the visual presentation on the smart glasses
  display. The module implements prioritization logic to ensure safety-critical
  alerts are displayed prominently while maintaining a non-obstructive HUD-style
  interface. This design decision separates visualization concerns from audio
  processing, allowing the display format and content to be modified
  independently (supporting \hyperref[acVizContent]{AC: Visualization Content})
  without affecting core audio analysis functionality.

\item The SRS mentions multi-source handling as a functionality goal
  (\hyperref[SRS-sec:g4]{Section G.4}), where the system should detect and track
  multiple simultaneous sound sources when feasible. To support this capability,
  an Independent Component Analysis (ICA) module, \mref{mICA}, is designed to
  separate mixed audio signals into individual source components. This module
  would enable improved classification and direction estimation accuracy when
  multiple sound sources are present simultaneously by separating them before
  processing. The module is designed to be conditionally invoked by the Audio360
  Engine when multiple sources are detected, allowing the system to operate with
  or without this capability.
\end{enumerate}


\section{Module Decomposition} \label{SecMD}

Modules are decomposed according to the principle of ``information hiding''
proposed by \cite{ParnasEtAl1984}. The \emph{Secrets} field in a module
decomposition is a brief statement of the design decision hidden by the module.
The \emph{Services} field specifies \emph{what} the module will do without
documenting \emph{how} to do it. For each module, a suggestion for the
implementing software is given under the \emph{Implemented By} title. If the
entry is \emph{Driver Layer}, this means that the module is provided by the
\hyperref[SRS-def:stm32]{STM32} or by standard programming language libraries.
\emph{\progname{}} means the module will be implemented by the \progname{}
software.

Only the leaf modules in the hierarchy have to be implemented. If a dash
(\emph{--}) is shown, this means that the module is not a leaf and will not have
to be implemented.

\subsection{Hardware Hiding Modules}

%% MICROPHONE INPUT MODULE.
\subsubsection{Microphone Input Module \mref{mMicInput}}
\begin{description}
\item [Secrets: ] The configuration and initialization of the hardware
peripherals required to read data from the microphone array. 
\item [Services: ] Provides a driver interface to read audio sample data from
the microphone array as a buffered stream.
\item [Implemented By: ] Driver Layer
\item[Type of Module:] Library 
\end{description}

%% USART COMMUNICATION MODULE.
\subsubsection{USART Communication Module \mref{mUSART}}
\begin{description}
\item [Secrets: ] The configuration and initialization of the hardware
peripherals required for USART serial communication with external devices.
\item [Services: ] Provides an interface for sending and receiving serial data
using the USART protocol on the serial pins of the microcontroller.
\item [Implemented By: ] Driver Layer
\item[Type of Module:] Library 
\end{description}

%% SD CARD INTERFACE MODULE.
\subsubsection{SD Card Interface Module \mref{mSDInterface}}
\begin{description}
\item [Secrets:] The configuration and initialization of the SPI hardware
peripherals and clocks required for SD card communication. This module also
hides the SD card file system management and driver used to send commands to the
SD card.
\item [Services:] Provides an interface for reading and writing files to an SD
card module connected to the microcontroller via SPI.
\item [Implemented By:] Driver Layer
\item[Type of Module:] Library 
\end{description}

%% MICROPHONE DIAGNOSTIC MODULE.
\subsubsection{Microphone Diagnostic Module \mref{mMicDiag}}
\begin{description}
\item [Secrets: ] The internal logic used to perform diagnostic tests on the
microphone array hardware using hardware ADC converters. 
\item [Services: ] Provides diagnostic state information about the microphone
array hardware and its connections to the microcontroller. 
\item [Implemented By: ] Driver Layer
\item [Type of Module:] Abstract Data Type
\end{description}


\subsection{Behaviour-Hiding Module}

% AUDIO360 ENGINE MODULE.
\subsubsection{Audio360 Engine Module (\mref{mAudio360Engine})}

\begin{description}
\item[Secrets:] The internal coordination logic that determines the execution
  order and timing of core audio processing features. Also, the mechanisms that
  manage data flow between hardware interfaces and the software modules.
\item[Services:] Orchestrates the overall audio processing by receiving raw
  input data and managing module communication.
\item[Implemented By:] \progname
\item[Type of Module:] Abstract Data Type 
\end{description}

% VISUALIZATION MODULE.
\subsubsection{Visualization Module (\mref{mVisualization})}
\begin{description}
  \item[Secrets:] The algorithms and data structures used to format and render
  visual information on the smart glasses display. This includes the layout
  strategy for HUD elements, prioritization logic for safety-critical
  information, and the rendering pipeline for directional indicators and
  classification labels. The module also hides the specific display hardware
  interface details, allowing the visualization format to be modified
  independently of the underlying display technology.
  \item[Services:] Accepts classification labels and direction angles (in
  radians) from the Audio360 Engine, formats this information according to
  display constraints and prioritization rules, and renders visual indicators on
  the smart glasses display. The module ensures non-intrusive presentation while
  maintaining a minimum update rate of 30 frames per second. It also handles
  display of error states and safety feature failure alerts.
  \item[Implemented By:] \progname
  \item[Type of Module:] Abstract Data Type 
\end{description}

\subsection{Software Decision Module}

% AUDIO GENERATION MODULE.
\subsubsection{Audio Generation Module (\mref{mAudioGen})}

\begin{description}
\item[Secrets:] The algorithm for simulating room acoustics and generating
  synthetic microphone array data from a given audio source and position. This
  includes room response calculations and spatial audio propagation models. 
\item[Services:] Provides simulated four-channel microphone array output for
  testing and proof-of-concept development. Takes as input an audio source
  signal and 3D position coordinates, and outputs four synchronized audio
  streams representing what each microphone in the array would capture given the
  room acoustics and source location.
\item[Implemented By:] Python using pyroomacoustics library
\item[Type of Module:] Library
\item[Notes:] This module is temporary and intended only for proof-of-concept
  development. It will be removed in the final product once physical hardware
  microphones are integrated. The module enables algorithm development and
  testing without requiring the complete hardware setup.
\end{description}

% DOA PROCESSOR MODULE
\subsubsection{DOA (Directional of Arrival) Processor Module (\mref{mDOA})}

\begin{description}
\item[Secrets:] The specific direction of arrival estimation algorithm and its
  implementation details, including signal processing techniques (MUSIC,
  SRP-PHAT, or FRIDA), frequency analysis methods, and coordinate system
  transformations.
\item[Services:] Analyzes four synchronized microphone audio streams to estimate
  the direction of arrival of a sound source. Takes as input four time-domain or
  frequency-domain audio signals and outputs an angle (in degrees) representing
  the estimated direction of the sound source relative to the microphone array's
  reference axis. Provides real-time directional audio analysis with target
  accuracy of 45 degrees as specified in the SRS
  (\hyperref[SRS-NFR5_3]{NFR5.3}).
\item[Implemented By:] \progname{} (Python for prototyping, C/C++ for embedded
  implementation)
\item[Type of Module:] Abstract Data Type
\item[Notes:] This module is core to the system's primary functionality of
  providing directional awareness. It implements one or more DOA algorithms and
  may include algorithm selection logic based on environmental conditions or
  computational constraints. Ideally, there should be no changes to this module
  for the final product aside from the internal implementation details
  (programming language for efficiency and combatilibility).
\end{description}


% AUDIO SAMPLING MODULE.
\subsubsection{Audio Sampling Module (\mref{mAudioSampling})}
\begin{description}
\item[Secrets:] The data structure used to store sampled audio data, the
  scheduling of sampling times, and the memory optimization strategy for
  acquiring and storing samples.
\item[Services:] Provides sampling of audio data from a given source while
  preserving the order of individual samples.
\item[Implemented By:] \progname
\item[Type of Module:] Abstract Data Type
\end{description}

% AUDIO SPECTRAL LEAKAGE PREVENTION MODULE.
\subsubsection{Audio Spectral Leakage Prevention Module
  (\mref{mAudioSpectralLeakage})}
\begin{description}
\item[Secrets:] Windowing function strategy (Windowing is the method to minimize
  spectral leakage).
\item[Services:] Applying windowing on audio signal to reduce
  \hyperref[SRS-def:spectral_leakage]{spectral leakage} before
  \hyperref[SRS-def:frequency_domain]{frequency domain} processing.
\item[Implemented By:] \progname
\item[Type of Module:] Abstract Object
\end{description}

% AUDIO NORMALIZER MODULE.
\subsubsection{Audio Normalizer Module (\mref{mAudioNormalizer})}
\begin{description}
\item[Secrets:] The normalization algorithm.
\item[Services:] Processes audio signals to maintain consistent amplitude across
  different sources.
\item[Implemented By:] \progname
\item[Type of Module:] Abstract Object
\end{description}

% AUDIO ANOMALY DETECTION MODULE.
\subsubsection{Audio Anomaly Detection Module (\mref{mAudioAnomaly})}
\begin{description}
\item[Secrets:] The detection algorithms used to identify anomalies in the audio
  stream. Also includes the list of dependent modules that must be notified upon
  specific anomaly occurrences.
\item[Services:]  Detects anomalies in the audio signal, such as clipping,
  dropouts, or empty buffer conditions, and issues appropriate notifications.
\item[Implemented By:] \progname
\item[Type of Module:] Abstract Object
\end{description}

% FAST FOURIER TRANSFORM MODULE.
\subsubsection{Fast Fourier Transform Module (\mref{mFFT})}
\begin{description}
\item[Secrets:] The specific algorithm to compute the fourier transform,
  potentially including
  \hyperref[SRS-def:hardware_acceleration]{hardware acceleraion}
  provided by the STM32 platform.
\item[Services:] Computes the discrete Fourier transform of the input signal to
  obtain its \hyperref[SRS-def:frequency_domain]{frequency domain}
  representation.
\item[Implemented By:] \progname
\item[Type of Module:] Abstract Object
\end{description}

% LOGGING MODULE.
\subsubsection{Logging Module (\mref{mLogging})}
\begin{description}
\item[Secrets:] The method used to route character streams to output sources,
  and the internal log formatting strategy, including time-stamping and file
  source information.
\item[Services:] Provides centralized logging of data streams to designated
  output destinations for debugging and monitoring. 
\item[Implemented By:] \progname
\item[Type of Module:] Abstract Object
\end{description}

% SOFTWARE FAULT MANAGER MODULE.
\subsubsection{Software Fault Manager Module (\mref{mFaultManager})}
\begin{description}
\item[Secrets:] The internal mechanisms and decision logic used to detect,
  manage, and recover from software system faults.
\item[Services:] Monitors system error states and manages critical faults to
maintain core functionality when possible.
\item[Implemented By:] \progname
\item[Type of Module:] Abstract Data Type
\end{description}

% MEL FILTER MODULE.
\subsubsection{Mel Filter Module (\mref{mMelFilter})}
\begin{description}
  \item[Secrets:] The specific algorithm for converting a signal from the
  spectral domain to the mel spectral domain. 
  \item[Services:] Accepts a time-frequency domain signal and outputs the
  corresponding time-mel-frequency signal based on the desired number of
  mel-bands. 
  \item[Implemented By:] \progname
  \item[Type of Module:] Abstract Object
\end{description}

% DISCRETE COSINE TRANSFORM MODULE.
\subsubsection{Discrete Cosine Transform Module (\mref{mDCT})}
\begin{description}
  \item[Secrets:] The specific algorithm for computing the Discrete Cosine
  Transform (DCT) from a signal in the mel spectral domain.
  \item[Services:] Accepts a time-mel-frequency signal, and runs the algorithm
  that computes the DCT of the signal. This will output the Mel Frequency
  Ceptral Coefficients (MFCC) that will describe the main features of the input
  signal.  
  \item[Implemented By:] \progname
  \item[Type of Module:] Abstract Object
\end{description}

% PRINCIPLE COMPONENT ANALYSIS MODULE.
\subsubsection{Principle Component Analysis Module (\mref{mPCA})}
\begin{description}
  \item[Secrets:] The specific algorithm for transforming the MFCC feature
  vector into a lower-dimensional space by extracting only the most significant
  variance. Also includes the projection matrix and mean feature matrix learned
  during offline training.
  \item[Services:] Accepts MFCC feature vector and produces new representation
  that preserves only the most significant variance from the original MFCC
  vector. This reduces the computational load for the LDA (\mref{mLDA}) module. 
  \item[Implemented By:] \progname
  \item[Type of Module:] Abstract Data Type
\end{description}

% LINEAR DISCRIMINANT ANALYSIS MODULE.
\subsubsection{Linear Discriminant Analysis Module (\mref{mLDA})}
\begin{description}
  \item[Secrets:] The specific algorithm for applying Linear Discriminant
  Analysis (LDA) to determine which class the sound most likely belongs to and
  the probability. Also includes the project matrix and class-specific weight
  and bias values learned during offline training.
  \item[Services:] Accepts feature vector extracted from the PCA (\mref{mPCA})
  module and performs linear discriminant classification to output the
  classification label corresponding to the detected sound category. 
  \item[Implemented By:] \progname
  \item[Type of Module:] Abstract Data Type
\end{description}

% CLASSIFICATION MODULE.
\subsubsection{Classification Module (\mref{mClassification})}
\begin{description}
  \item[Secrets:] The modules and the order to call them in to get the most
  probable classification using raw input audio. It also contains heuristics for
  determining when a classification is considered low-confidence. 
  \item[Services:] Accepts raw input audio, passes the audio through various
  modules to determine the most probable classification and it's confidence.
  Internally determines whether the classification is considered low confidence
  and outputs the classification with a flag if the confidence is low. 
  \item[Implemented By:] \progname
  \item[Type of Module:] Abstract Data Type
\end{description}

% ICA MODULE.
\subsubsection{Independent Component Analysis (ICA) Module (\mref{mICA})}
\begin{description}
  \item[Secrets:] The specific ICA algorithm implementation (e.g., FastICA) and
  its parameters for blind source separation. This includes the statistical
  independence criteria, optimization methods, and convergence thresholds used
  to separate mixed audio signals into independent source components.
  \item[Services:] Accepts multi-channel audio signals from the microphone array
  and separates them into individual independent audio source components. Takes
  as input mixed audio signals from multiple microphones and outputs separated
  audio streams, each representing an individual sound source. This enables
  improved classification and direction estimation when multiple sound sources
  are present simultaneously. The module may be conditionally invoked by the
  Audio360 Engine when multiple sources are detected.
  \item[Implemented By:] \progname
  \item[Type of Module] Abstract Object
  \item[Notes:] This module is a stretch goal and may or may not be implemented
  depending on computational constraints and development timeline. It supports
  the multi-source handling functionality mentioned in the SRS
  (\hyperref[SRS-sec:g4]{Section G.4}). If implemented, it would enhance the
  system's ability to handle complex acoustic environments with overlapping
  sound sources by separating them before classification and direction
  estimation.
\end{description}

\section{Traceability Matrix} \label{SecTM}

This section shows two traceability matrices: between the modules and the
requirements and between the modules and the anticipated changes.

\subsection{Modules to Requirements Traceability}

The following table maps each module to the functional and non-functional
requirements from the SRS that it helps satisfy. Requirements are referenced
using the labels defined in the SRS document.

\begin{longtable}{p{0.2\textwidth} p{0.6\textwidth}}
\toprule
\textbf{Req.} & \textbf{Modules}\\
\midrule
\hyperref[SRS-FR1_2]{FR1.2} & \mref{mMicInput}\\
\hyperref[SRS-FR1_4]{FR1.4} & \mref{mFaultManager}, \mref{mMicDiag}\\
\hyperref[SRS-NFR1_1]{NFR1.1} & \mref{mAudioSampling} \\
\hyperref[SRS-NFR1_2]{NFR1.2} & \mref{mAudioSampling}\\
\hyperref[SRS-NFR1_3]{NFR1.3} & \mref{mMicInput}\\
\hline
\hyperref[SRS-FR2_1]{FR2.1} & \mref{mMicInput}, \mref{mUSART},
  \mref{mSDInterface} \\
\hyperref[SRS-FR2_3]{FR2.3} & \mref{mFaultManager} \\
\hyperref[SRS-NFR2_1]{NFR2.1} & \mref{mFaultManager} \\
\hline
\hyperref[SRS-FR3_1]{FR3.1} & \mref{mFFT} \\
\hyperref[SRS-FR3_2]{FR3.2} & \mref{mAudioNormalizer} \\
\hyperref[SRS-FR3_3]{FR3.3} & \mref{mAudioSpectralLeakage} \\
\hyperref[SRS-FR3_4]{FR3.4} & \mref{mAudioSpectralLeakage},
  \mref{mAudioNormalizer}, \mref{mFFT} \\
\hyperref[SRS-FR3_5]{FR3.5} & \mref{mAudioAnomaly} \\
\hyperref[SRS-NFR3_1]{NFR3.1} & \mref{mFFT} \\
\hyperref[SRS-NFR3_2]{NFR3.2} & \mref{mAudioSampling} \\
\hline
\hyperref[SRS-FR4_1]{FR4.1} & \mref{mAudio360Engine} \\
\hyperref[SRS-FR4_2]{FR4.2} & \mref{mAudio360Engine} \\
\hyperref[SRS-FR4_3]{FR4.3} & \mref{mAudio360Engine} \\
\hyperref[SRS-FR4_4]{FR4.4} & \mref{mAudio360Engine}, \mref{mFaultManager} \\
\hyperref[SRS-NFR4_1]{NFR4.1} & \mref{mAudioSampling} \\
\hyperref[SRS-NFR4_2]{NFR4.2} & \mref{mAudioSampling} \\
\hyperref[SRS-NFR4_3]{NFR4.3} & \mref{mAudioSampling} \\
\hline
\hyperref[SRS-FR5_1]{FR5.1} & \mref{mMelFilter}, \mref{mDCT}, \mref{mPCA},
\mref{mLDA}, \mref{mClassification}\\
\hyperref[SRS-FR5_2]{FR5.2} & \mref{mDOA} \\
\hyperref[SRS-FR5_3]{FR5.3} & \mref{mDOA} \\
\hyperref[SRS-FR5_4]{FR5.4} & \mref{mClassification}\\
\hyperref[SRS-NFR5_3]{NFR5.3} & \mref{mDOA} \\
\hline
\hyperref[SRS-FR6_1]{FR6.1} & \mref{mAudio360Engine}, \mref{mVisualization} \\
\hyperref[SRS-FR6_2]{FR6.2} & \mref{mVisualization}, \mref{mFaultManager} \\
\hyperref[SRS-NFR6_1]{NFR6.1} & \mref{mVisualization} \\
\hyperref[SRS-NFR6_2]{NFR6.2} & \mref{mVisualization} \\
\hline
\hyperref[SRS-FR7_1]{FR7.1} & \mref{mMicInput}, \mref{mAudioSampling} \\
\hyperref[SRS-FR7_2]{FR7.2} & \mref{mMicInput} \\
\hyperref[SRS-NFR7_1]{NFR7.1} & \mref{mMicInput} \\
\hline
\hyperref[SRS-FR8_1]{FR8.1} & \mref{mVisualization} \\
\hyperref[SRS-FR8_2]{FR8.2} & \mref{mVisualization} \\
\hyperref[SRS-NFR8_1]{NFR8.1} & \mref{mVisualization} \\
\bottomrule
\caption{Trace Between Requirements and Modules}
\label{TblRT}
\end{longtable}


\subsection{Modules to Anticipated Changes Traceability}

The following table maps anticipated changes to the modules that would need to
be modified if those changes occur. This supports the principle of information
hiding by showing which design decisions are localized to specific modules.

\begin{table}[H]
\centering
\begin{tabular}{p{0.25\textwidth} p{0.65\textwidth}}
\toprule
\textbf{AC} & \textbf{Modules}\\
\midrule
\acref{acHardware} & -\\
\acref{acNumMic} & \mref{mMicInput}\\
\acref{acMicArrangement} & -\\
\acref{acOutput} & \mref{mVisualization}\\
\acref{NumClassification} & \mref{mClassification}\\
\acref{SampleRate} & \mref{mAudioSampling}\\
\acref{SampleSize} & \mref{mAudioSampling}\\
\acref{acVizContent} & \mref{mVisualization}\\
\acref{acHardwareAccel} & \mref{mFFT}\\
\acref{acConfidenceAction} & \mref{mAudio360Engine}\\
\acref{acAudioGen} & \mref{mAudioGen}, \mref{mAudioGen} \\
\acref{acSDCardLogging} & \mref{mSDInterface}, \mref{mLogging} \\
\acref{acICA} & \mref{mICA}\\
\bottomrule
\end{tabular}
\caption{Trace Between Anticipated Changes and Modules}
\label{TblACT}
\end{table}

\acref{acHardware}, \acref{acMicArrangement}, are tied to hardware changes, thus
there is no corresponding software module that traces to them.

\section{Use Hierarchy Between Modules} \label{SecUse}

In this section, the uses hierarchy between modules is provided.
\cite{Parnas1978} said of two programs A and B that A {\em uses} B if correct
execution of B may be necessary for A to complete the task described in its
specification. That is, A {\em uses} B if there exist situations in which the
correct functioning of A depends upon the availability of a correct
implementation of B.  Figure \ref{FigUH} illustrates the use relation between
the modules. It can be seen that the graph is a directed acyclic graph (DAG).
Each level of the hierarchy offers a testable and usable subset of the system,
and modules in the higher level of the hierarchy are essentially simpler because
they use modules from the lower levels.

\begin{figure}[H]
\centering
\includegraphics[width=\textwidth]{diagrams/uses_hierarchy.png}
\caption{Use hierarchy among modules}
\label{FigUH}
\end{figure}


\section{User Interfaces}

The user interface for \progname is primarily visual, displayed through smart
glasses in a heads-up display (HUD) format. The interface provides real-time
visual indicators showing sound source classification and direction of arrival
information.

\begin{figure}[H]
    \centering
    \includegraphics[width=0.9\textwidth]{../../Images/Smart glasses HUD.png}
    \caption{Smart glasses HUD visualization showing sound detection interface.
    The display shows a detected sound source (Vehicle) with directional
    information overlaid on the user's field of view.}
    \label{fig:smart_glasses_hud}
\end{figure}

\begin{figure}[H]
    \centering
    \includegraphics[width=0.9\textwidth]{../../Images/Smart Glasses HUD Close Up.png}
    \caption{Close-up view of the smart glasses HUD interface. This view
    provides a detailed look at the AR overlay elements including the sound
    detection indicator, classification label, and directional arrow displayed
    on the left lens.}
    \label{fig:smart_glasses_hud_closeup}
\end{figure}


\section{Design of Communication Protocols}

Not applicable at this stage of the project since there is no SDK or
connectivity information available for the Rokid smart glasses. Communication
protocols will be designed once the hardware integration phase begins and more
details    
about the smart glasses platform are known.

\section{Timeline}

A roadmap for completing the module implementations is provided in the
\href{https://github.com/orgs/Team6-SixSense/projects/7}{Github project
roadmap}. It is organized by milestones corresponding to the major phases of the
project timeline (Rev 0, Rev 1).

Major implementation for the following modules have been completed as of the
writing of this document:
\begin{itemize}
\item Audio Generation Module (\mref{mAudioGen})
\item Logging Module (\mref{mLogging})
\item Fast Fourier Transform Module (\mref{mFFT})
\item SD Card Interface Module (\mref{mSDInterface})
\item USART Communication Module (\mref{mUSART})
\end{itemize} 
No tasks are thus generated for these modules in the project roadmap.

\newpage
\bibliographystyle{IEEEtran}
\bibliography{../../../refs/References}

\newpage{}

\end{document}