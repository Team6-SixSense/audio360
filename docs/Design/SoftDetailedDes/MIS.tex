\documentclass[12pt, titlepage]{article}

\usepackage{amsmath, mathtools}

\usepackage{amsfonts}
\usepackage{amssymb}
\usepackage{graphicx}
\usepackage{colortbl}
\usepackage{xr}
\usepackage{hyperref}
\usepackage{longtable}
\usepackage{xfrac}
\usepackage{tabularx}
\usepackage{float}
\usepackage{siunitx}
\usepackage{booktabs}
\usepackage{multirow}
\usepackage[section]{placeins}
\usepackage{caption}
\usepackage{fullpage}
\usepackage{cite}

\hypersetup{
bookmarks=true,     % show bookmarks bar?
colorlinks=true,       % false: boxed links; true: colored links
linkcolor=red,          % color of internal links (change box color with linkbordercolor)
citecolor=blue,      % color of links to bibliography
filecolor=magenta,  % color of file links
urlcolor=cyan          % color of external links
}

\usepackage{array}

\externaldocument[SRS-]{../../SRS/SRS}
\externaldocument[MG-]{../SoftArchitecture/MG}
\newcommand{\mref}[1]{M\ref{#1}}

\input{../../Comments}
%% Common Parts

\newcommand{\progname}{Audio360} % PUT YOUR PROGRAM NAME HERE
\newcommand{\authname}{Team 6, SixSense
\\ Omar Alam
\\ Sathurshan Arulmohan
\\ Nirmal Chaudhari
\\ Kalp Shah
\\ Jay Sharma
} % AUTHOR NAMES                  

\usepackage{hyperref}
    \hypersetup{colorlinks=true, linkcolor=blue, citecolor=blue, filecolor=blue,
                urlcolor=blue, unicode=false}
    \urlstyle{same}
                                


\begin{document}

\title{Module Interface Specification for \progname{}}

\author{\authname}

\date{\today}

\maketitle

\pagenumbering{roman}

\section{Revision History}

\begin{tabularx}{\textwidth}{p{3cm}p{2cm}X}
\toprule {\bf Date} & {\bf Version} & {\bf Notes}\\
\midrule
2025-11-13 & 1.0 & Initial write-up\\
\bottomrule
\end{tabularx}

~\newpage

\section{Symbols, Abbreviations and Acronyms}

\begin{table}[H]
  \centering
  \begin{tabular}{l l} 
    \toprule		
    \textbf{symbol} & \textbf{description}\\
    \midrule 
    \progname & 360 Audio analysis system on smart glasses\\
    FR & Functional Requirement\\
    M & Module \\
    MG & Module Guide \\
    MIS & Module Interface Specification \\
    NFR & Non-functional Requirement\\
    R & Requirement\\
    SPI & Serial Peripheral Interface \\
    SRS & Software Requirements Specification\\
    USART & Universal Synchronous/Asynchronous Receiver/Transmitter \\
    \bottomrule
  \end{tabular}\\
  \caption{Symbols, abbreviations and acronyms used in the MIS document.}
\end{table}

See SRS Documentation at
\hyperref[SRS-sec:symbols]{Symbols, Abbreviations, and Acronyms}
for a complete table used in \progname.

\newpage

\tableofcontents

\newpage

\pagenumbering{arabic}

\section{Introduction}

The following document details the Module Interface Specifications for
\wss{Fill in your project name and description}

Complementary documents include the System Requirement Specifications
and Module Guide.  The full documentation and implementation can be
found at \url{...}.  \wss{provide the url for your repo}

\section{Notation}

\wss{You should describe your notation.  You can use what is below as
  a starting point.}

The structure of the MIS for modules comes from \cite{HoffmanAndStrooper1995},
with the addition that template modules have been adapted from
\cite{GhezziEtAl2003}.  The mathematical notation comes from Chapter 3 of
\cite{HoffmanAndStrooper1995}.  For instance, the symbol := is used for a
multiple assignment statement and conditional rules follow the form $(c_1
\Rightarrow r_1 | c_2 \Rightarrow r_2 | ... | c_n \Rightarrow r_n )$.

The following table summarizes the primitive data types used by \progname. 

\begin{center}
\renewcommand{\arraystretch}{1.2}
\noindent 
\begin{tabular}{l l p{7.5cm}} 
\toprule 
\textbf{Data Type} & \textbf{Notation} & \textbf{Description}\\ 
\midrule
character & char & a single symbol or digit\\
integer & $\mathbb{Z}$ & a number without a fractional component in (-$\infty$, $\infty$) \\
natural number & $\mathbb{N}$ & a number without a fractional component in [1, $\infty$) \\
real & $\mathbb{R}$ & any number in (-$\infty$, $\infty$)\\
\bottomrule
\end{tabular} 
\end{center}

\noindent
The specification of \progname \ uses some derived data types: sequences, strings, and
tuples. Sequences are lists filled with elements of the same data type. Strings
are sequences of characters. Tuples contain a list of values, potentially of
different types. In addition, \progname \ uses functions, which
are defined by the data types of their inputs and outputs. Local functions are
described by giving their type signature followed by their specification.

\section{Module Decomposition}

The following table is taken directly from the Module Guide document for this project.

\begin{table}[h!]
\centering
\begin{tabular}{p{0.3\textwidth} p{0.6\textwidth}}
\toprule
\textbf{Level 1} & \textbf{Level 2}\\
\midrule

{Hardware-Hiding} & ~ \\
\midrule

\multirow{7}{0.3\textwidth}{Behaviour-Hiding} & Input Parameters\\
& Output Format\\
& Output Verification\\
& Temperature ODEs\\
& Energy Equations\\ 
& Control Module\\
& Specification Parameters Module\\
\midrule

\multirow{3}{0.3\textwidth}{Software Decision} & {Sequence Data Structure}\\
& ODE Solver\\
& Plotting\\
\bottomrule

\end{tabular}
\caption{Module Hierarchy}
\label{TblMH}
\end{table}

\newpage
\section{Data Types}

\newpage
\subsection{Generics}

\subsubsection{int32}\label{generic:int32}
32 bits signed integer ($\mathbb{Z}$).

\subsubsection{int64}\label{generic:int64}
64 bits signed integer ($\mathbb{Z}$).

\subsubsection{uint32}\label{generic:uint32}
32 bits unsigned integer ($\mathbb{N}$).

\subsubsection{uint64}\label{generic:uint64}
64 bits unsigned integer ($\mathbb{N}$).

\subsubsection{float32}\label{generic:float32}
32 bits floating point ($\mathbb{R}$).

\subsubsection{float64}\label{generic:float64}
64 bits floating point ($\mathbb{R}$).

\newpage
\subsection{Enums}
\subsubsection{Audio360State}\label{state:Audio360State}
\begin{enumerate}
  \item AudioClassificationProcess:
    \label{state:Audio360State:AudioClassificationState}
    State when audio classification is running.
  \item DirectionalAnalysisProcess:
    \label{state:Audio360State:DirectionalAnalysisState}
    State when directional analysis is running.
  \item OutputProcess:
    \label{state:Audio360State:OutputProcessState}
    State when output processing is running.
\end{enumerate}

\subsubsection{Audio360Status}\label{state:Audio360Status}
\begin{enumerate}
  \item Uninitialized: \label{state:Audio360Status:Uninitialized}
    Audio360 Engine is not initialized.
  \item Initialized: \label{state:Audio360Status:Initialized}
    Audio360 Engine is initialized, but not ready.
  \item Ready: \label{state:Audio360Status:Ready}
    Audio360 Engine ready for requests.
  \item Running: \label{state:Audio360Status:Running}
    Audio360 Engine is running. Can not accept new requests.
  \item Error: \label{state:Audio360Status:Error}
    Audio360 Engine is stuck at an unhandled error.
\end{enumerate}

\newpage
\subsection{Data Structures}

~\newpage

\section{MIS of \wss{Module Name}} \label{Module} \wss{Use labels for
  cross-referencing}

\wss{You can reference SRS labels, such as R.}

\wss{It is also possible to use \LaTeX for hypperlinks to external documents.}

\subsection{Module}

\wss{Short name for the module}

\subsection{Uses}


\subsection{Syntax}

\subsubsection{Exported Constants}

\subsubsection{Exported Access Programs}

\begin{center}
\begin{tabularx}{\linewidth}{l X X X}
\hline
\textbf{Name} & \textbf{In} & \textbf{Out} & \textbf{Exceptions} \\
\hline
\wss{accessProg} & - & - & - \\
\hline
\end{tabularx}
\end{center}

\subsection{Semantics}

\subsubsection{State Variables}

\wss{Not all modules will have state variables.  State variables give the module
  a memory.}

\subsubsection{Environment Variables}

\wss{This section is not necessary for all modules.  Its purpose is to capture
  when the module has external interaction with the environment, such as for a
  device driver, screen interface, keyboard, file, etc.}

\subsubsection{Assumptions}

\wss{Try to minimize assumptions and anticipate programmer errors via
  exceptions, but for practical purposes assumptions are sometimes appropriate.}

\subsubsection{Access Routine Semantics}

\noindent \wss{accessProg}():
\begin{itemize}
\item transition: \wss{if appropriate} 
\item output: \wss{if appropriate} 
\item exception: \wss{if appropriate} 
\end{itemize}

\wss{A module without environment variables or state variables is unlikely to
  have a state transition.  In this case a state transition can only occur if
  the module is changing the state of another module.}

\wss{Modules rarely have both a transition and an output.  In most cases you
  will have one or the other.}

\subsubsection{Local Functions}

\wss{As appropriate} \wss{These functions are for the purpose of specification.
  They are not necessarily something that is going to be implemented
  explicitly.  Even if they are implemented, they are not exported; they only
  have local scope.}

\newpage
\section{MIS of DOA Processor Module}

\subsection{Module}

This module processes audio data to estimate the direction of arrival of a sound 
source. This includes frequency domain analysis, signal processing, and 
direction estimation algorithms.

\subsection{Uses}

\begin{enumerate}
  \item Audio Generation Module (\mref{mAudioGen})
  \item Audio Filtering Module (\mref{mFFT})
  \item Audio Normalizer Module (\mref{mAudioNormalizer})
  \item Audio Spectral Leakage Module (\mref{mAudioSpectralLeakage})
  \item Audio Anomaly Module (\mref{mAudioAnomaly})
\end{enumerate}

\subsection{Syntax}

\subsubsection{Exported Constants}

\subsubsection{Exported Access Programs}

\begin{center}
\begin{tabularx}{\linewidth}{l X X X}
\hline
\textbf{Name} & \textbf{In} & \textbf{Out} & \textbf{Exceptions} \\
\hline
calculateDirection & audioData[][]: [][\hyperref[generic:float32]{float32}] 
& direction: \hyperref[generic:float32]{float32} 
& AudioProcessingFailure \\
\hline
\end{tabularx}
\end{center}

\subsection{Semantics}

\subsubsection{State Variables}

None.

\subsubsection{Environment Variables}

None.

\subsubsection{Assumptions}

None.

\subsubsection{Access Routine Semantics}

\noindent calculateDirection():
\begin{itemize}
\item transition: None
\item output: direction
\item exception: AudioProcessingFailure
\end{itemize}

\subsubsection{Local Functions}

None.

\newpage
\section{MIS of Audio Generation Module} \label{MAudioGeneration}

\subsection{Module}

This module generates audio data by simulating room acoustics and generating 
synthetic microphone array data from a given audio source and position. This 
includes room response calculations and spatial audio propagation 
models. 

\subsection{Uses}

\begin{enumerate}
  \item Hardware-Hiding Module (\mref{mHH})
  \item Python Standard Libraries
\end{enumerate}

\subsection{Syntax}

\subsubsection{Exported Constants}

\subsubsection{Exported Access Programs}

\begin{center}
\begin{tabularx}{\linewidth}{l X X X}
\hline
\textbf{Name} & \textbf{In} & \textbf{Out} & \textbf{Exceptions} \\
\hline
generateAudio & audioSource: \hyperref[generic:float32]{float32}[], 
position: \hyperref[generic:float32]{float32}[], 
outputFile: str & micrphoneData[][]: [][\hyperref[generic:float32]{float32}] 
& AudioGenerationFailure \\
\hline
\end{tabularx}
\end{center}

\subsection{Semantics}

\subsubsection{State Variables}

None.

\subsubsection{Environment Variables}

None.

\subsubsection{Assumptions}

The pyroomacoustics module is assumed to be working correctly (generating 
realistic and reliable audio data based on configuration parameters).

\subsubsection{Access Routine Semantics}

\noindent generateAudio():
\begin{itemize}
\item transition: None
\item output: micrphoneData
\item exception: AudioGenerationFailure
\end{itemize}

\subsubsection{Local Functions}

\begin{itemize}
\item simulateRoom(room: pra.ShoeBox) -> bool
\end{itemize}
  
\newpage
\section{MIS of Audio360 Engine \label{MAudio360Engine} - 
  \mref{MG-mAudio360Engine}}

\subsection{Module}

Orchestrates the overall audio processing by receiving raw input data and
managing \hyperref[MAudio360Engine_Uses]{module} communication.

\subsection{Uses}\label{MAudio360Engine_Uses}

\begin{enumerate}
  \item Audio classification
  \item Directional Analysis
  \item Output
  \item Audio Sampling Module
\end{enumerate}

\subsection{Syntax}

\subsubsection{Exported Constants}

None

\subsubsection{Exported Access Programs}

\begin{center}
\begin{tabularx}{\linewidth}{l X X X}
\hline
\textbf{Name} & \textbf{In} & \textbf{Out} & \textbf{Exceptions} \\
\hline
runProgram & - & - & ProgramStartFailure, ProgramRunTimeFailure \\
\hline
\end{tabularx}
\end{center}

\subsection{Semantics}

\subsubsection{State Variables}

\begin{itemize}
  \item state [\hyperref[state:Audio360State]{Audio360State}]:
    Determines the current state of the the Audio360 engine. Each state will run
    a specific module in \hyperref[MAudio360Engine_Uses]{used modules}.
\end{itemize}

\subsubsection{Environment Variables}

\begin{itemize}
  \item status [\hyperref[state:Audio360State]{Audio360Status}]:
    Status of the module. This encapsulates initialization, running, ready,
    or errored.
\end{itemize}

\subsubsection{Assumptions}

None

\subsubsection{Access Routine Semantics}

\noindent run():
\begin{itemize}
\item transition: The state machine in figure
  \ref{fig:audio360_engine_state_machine} outlines the transition of this
  module.

\begin{figure}[h!]
    \centering 
    \includegraphics[width=\textwidth]{diagrams/Audio360EngineStateMachine.png}
    \caption{Internal state machine of Audio360 Engine module.}
    \label{fig:audio360_engine_state_machine}
\end{figure}

\item output: None
\item exception: ProgramStartFailure, ProgramRunTimeFailure
\end{itemize}

\subsubsection{Local Functions}

None

\newpage

\bibliographystyle{IEEEtran}
\bibliography{../../../refs/References}

\newpage

\section{Appendix} \label{Appendix}

\wss{Extra information if required}

\newpage{}

\section*{Appendix --- Reflection}

\begin{enumerate}
  \item What went well while writing this deliverable? 
  
  \textbf{Sathurshan:} The system was decomposed into modules that was small
  enough. This allowed the team to easily split up the work without having
  much dependencies on each other. Furthermore, writing the MIS has helped the
  team further align with design decisions and formally document.

  \item What pain points did you experience during this deliverable, and how
    did you resolve them?

  \textbf{Sathurshan:} The main pain point was the deliverable deadline as the
  team was asked to finish the first revision in a week while preparing for the
  proof of concept of demo. The team expressed the concerns with the professor,
  and fornately received an extension allowing us to put more effort into this
  document.

  \item Which of your design decisions stemmed from speaking to your client(s)
  or a proxy (e.g. your peers, stakeholders, potential users)? For those that
  were not, why, and where did they come from?

  \textbf{Sathurshan:} The supervisor mentionned the difficulty of getting
  access to hardware that will meet our software specification. As a result, we
  designed the system such that the software is portable and can take input
  from different sources. This ensures that the capstone project can still
  be successful in the case we are not able to find the right hardware within
  our budget.

  \item While creating the design doc, what parts of your other documents (e.g.
  requirements, hazard analysis, etc), it any, needed to be changed, and why?

  \textbf{Sathurshan:} SRS needed to be updated. It was updated because part
  of writing this document required reviewing the SRS. From this, I have found
  parts of the requirements that can be improved based on recent better
  understanding of the system.

  \item What are the limitations of your solution?  Put another way, given
  unlimited resources, what could you do to make the project better?
  (LO\_ProbSolutions)
  
  \textbf{Sathurshan:} The limitation of the solution is compute and memory
  power. Given further time, the team would be able to design a system that
  is more low level and is optimized in accomplishing the tasks that are
  required.
  
  \item Give a brief overview of other design solutions you considered.  What
  are the benefits and tradeoffs of those other designs compared with the chosen
  design?  From all the potential options, why did you select the documented
  design? (LO\_Explores)

  \textbf{Sathurshan:} Personally, there was not a lot of time to think about
  other designs. The team has been implementing the software before this
  document was created since the proof of concept is one week after this
  document is due. Thus, the team already considered and analyzed high level
  designs months ago and is too far back to document them.

\end{enumerate}


\end{document}