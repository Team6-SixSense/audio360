\documentclass[12pt, titlepage]{article}

\usepackage{amsmath, mathtools}

\usepackage{amsfonts}
\usepackage{amssymb}
\usepackage{graphicx}
\usepackage{colortbl}
\usepackage{xr}
\usepackage{hyperref}
\usepackage{longtable}
\usepackage{xfrac}
\usepackage{tabularx}
\usepackage{float}
\usepackage{siunitx}
\usepackage{booktabs}
\usepackage{multirow}
\usepackage[section]{placeins}
\usepackage{caption}
\usepackage{fullpage}
\usepackage{cite}

\hypersetup{
bookmarks=true,     % show bookmarks bar?
colorlinks=true,       % false: boxed links; true: colored links
linkcolor=red,          % color of internal links (change box color with linkbordercolor)
citecolor=blue,      % color of links to bibliography
filecolor=magenta,  % color of file links
urlcolor=cyan          % color of external links
}

\usepackage{array}

\externaldocument[SRS-]{../../SRS/SRS}
\externaldocument[MG-]{../SoftArchitecture/MG}
\newcommand{\mref}[1]{M\ref{#1}}

\input{../../Comments}
%% Common Parts

\newcommand{\progname}{Audio360} % PUT YOUR PROGRAM NAME HERE
\newcommand{\authname}{Team 6, SixSense
\\ Omar Alam
\\ Sathurshan Arulmohan
\\ Nirmal Chaudhari
\\ Kalp Shah
\\ Jay Sharma
} % AUTHOR NAMES                  

\usepackage{hyperref}
    \hypersetup{colorlinks=true, linkcolor=blue, citecolor=blue, filecolor=blue,
                urlcolor=blue, unicode=false}
    \urlstyle{same}
                                


\begin{document}

\title{Module Interface Specification for \progname{}}

\author{\authname}

\date{\today}

\maketitle

\pagenumbering{roman}

\section{Revision History}

\begin{tabularx}{\textwidth}{p{3cm}p{2cm}X}
\toprule {\bf Date} & {\bf Version} & {\bf Notes}\\
\midrule
2025-11-13 & 1.0 & Initial write-up\\
\bottomrule
\end{tabularx}

~\newpage

\section{Symbols, Abbreviations and Acronyms}

\begin{table}[H]
  \centering
  \begin{tabular}{l l} 
    \toprule		
    \textbf{symbol} & \textbf{description}\\
    \midrule 
    \progname & 360 Audio analysis system on smart glasses\\
    FR & Functional Requirement\\
    M & Module \\
    MG & Module Guide \\
    MIS & Module Interface Specification \\
    NFR & Non-functional Requirement\\
    R & Requirement\\
    SPI & Serial Peripheral Interface \\
    SRS & Software Requirements Specification\\
    USART & Universal Synchronous/Asynchronous Receiver/Transmitter \\
    \bottomrule
  \end{tabular}\\
  \caption{Symbols, abbreviations and acronyms used in the MIS document.}
\end{table}

See SRS Documentation at
\hyperref[SRS-sec:symbols]{Symbols, Abbreviations, and Acronyms}
for a complete table used in \progname.

\newpage

\tableofcontents

\newpage

\pagenumbering{arabic}

\section{Introduction}

The following document details the Module Interface Specifications for
\wss{Fill in your project name and description}

Complementary documents include the System Requirement Specifications
and Module Guide.  The full documentation and implementation can be
found at \url{...}.  \wss{provide the url for your repo}

\section{Notation}

\wss{You should describe your notation.  You can use what is below as
  a starting point.}

The structure of the MIS for modules comes from \cite{HoffmanAndStrooper1995},
with the addition that template modules have been adapted from
\cite{GhezziEtAl2003}.  The mathematical notation comes from Chapter 3 of
\cite{HoffmanAndStrooper1995}.  For instance, the symbol := is used for a
multiple assignment statement and conditional rules follow the form $(c_1
\Rightarrow r_1 | c_2 \Rightarrow r_2 | ... | c_n \Rightarrow r_n )$.

The following table summarizes the primitive data types used by \progname. 

\begin{center}
\renewcommand{\arraystretch}{1.2}
\noindent 
\begin{tabular}{l l p{7.5cm}} 
\toprule 
\textbf{Data Type} & \textbf{Notation} & \textbf{Description}\\ 
\midrule
character & char & a single symbol or digit\\
integer & $\mathbb{Z}$ & a number without a fractional component in
  (-$\infty$, $\infty$) \\
natural number & $\mathbb{N}$ & a number without a fractional component in
  [1, $\infty$) \\
real & $\mathbb{R}$ & any number in (-$\infty$, $\infty$)\\
array/list & x[] & Ordered list of type x. \\
negate & $\neg$ & Logical math NOT.\\
and & $\wedge$ & Logical math AND \\
or & $\vee$ & Logical math OR \\
implies & $\implies$ & Logical math implies \\
assignment & $:=$ & For A := B. B is assigned to A. \\
for all & $\forall$ & Referencing all items. For example:
  $(\forall~variable~|~condition:~statement)$ \\
\bottomrule
\end{tabular} 
\end{center}

\noindent
The specification of \progname \ uses some derived data types: sequences, strings, and
tuples. Sequences are lists filled with elements of the same data type. Strings
are sequences of characters. Tuples contain a list of values, potentially of
different types. In addition, \progname \ uses functions, which
are defined by the data types of their inputs and outputs. Local functions are
described by giving their type signature followed by their specification.

\section{Module Decomposition}

The following table is taken directly from the Module Guide document for this project.

\begin{table}[h!]
\centering
\begin{tabular}{p{0.3\textwidth} p{0.6\textwidth}}
\toprule
\textbf{Level 1} & \textbf{Level 2}\\
\midrule

{Hardware-Hiding} & ~ \\
\midrule

\multirow{7}{0.3\textwidth}{Behaviour-Hiding} & Input Parameters\\
& Output Format\\
& Output Verification\\
& Temperature ODEs\\
& Energy Equations\\ 
& Control Module\\
& Specification Parameters Module\\
\midrule

\multirow{3}{0.3\textwidth}{Software Decision} & {Sequence Data Structure}\\
& ODE Solver\\
& Plotting\\
\bottomrule

\end{tabular}
\caption{Module Hierarchy}
\label{TblMH}
\end{table}

\newpage
\section{Data Types}

\subsection{Generics}


\subsubsection{int16}\label{generic:int16}
16 bits signed integer ($\mathbb{Z}$).

\subsubsection{int32}\label{generic:int32}
32 bits signed integer ($\mathbb{Z}$).

\subsubsection{int64}\label{generic:int64}
64 bits signed integer ($\mathbb{Z}$).

\subsubsection{uint16}\label{generic:uint16}
16 bits unsigned integer ($\mathbb{N}$).

\subsubsection{uint32}\label{generic:uint32}
32 bits unsigned integer ($\mathbb{N}$).

\subsubsection{uint64}\label{generic:uint64}
64 bits unsigned integer ($\mathbb{N}$).

\subsubsection{float32}\label{generic:float32}
32 bits floating point ($\mathbb{R}$).

\subsubsection{float64}\label{generic:float64}
64 bits floating point ($\mathbb{R}$).

\newpage
\subsection{Enums}
\subsubsection{Audio360State}\label{enum:Audio360State}
\begin{enumerate}
  \item AudioClassificationProcess:
    \label{enum:Audio360State:AudioClassificationState}
    State when audio classification is running.
  \item DirectionalAnalysisProcess:
    \label{enum:Audio360State:DirectionalAnalysisState}
    State when directional analysis is running.
  \item OutputProcess:
    \label{enum:Audio360State:OutputProcessState}
    State when output processing is running.
\end{enumerate}

\subsubsection{Audio360Status}\label{enum:Audio360Status}
\begin{enumerate}
  \item Uninitialized: \label{enum:Audio360Status:Uninitialized}
    Audio360 Engine is not initialized.
  \item Initialized: \label{enum:Audio360Status:Initialized}
    Audio360 Engine is initialized, but not ready.
  \item Ready: \label{enum:Audio360Status:Ready}
    Audio360 Engine ready for requests.
  \item Running: \label{enum:Audio360Status:Running}
    Audio360 Engine is running. Can not accept new requests.
  \item Error: \label{enum:Audio360Status:Error}
    Audio360 Engine is stuck at an unhandled error.
\end{enumerate}

subsubsection{Audio360Status}\label{enum:FaultState}
\begin{enumerate}
  \item NoFault: \label{enum:FaultState:NoFault}
    No fault state.
  \item MicrophoneXFault: \label{enum:FaultState:MicFault}
    Fault with microphone X.
  \item AudioClassificationFault: \label{enum:FaultState:ClassificationFault}
    Fault with audio classification.
  \item directionalAnalsysiFault: \label{enum:FaultState:DOAFault}
    Fault with the directional analysis.
\end{enumerate}


\newpage
\subsection{Data Structures}
\subsubsection{FrequencyDomain}\label{struct:FrequencyDomain}
\begin{enumerate}
  \item N [\hyperref[generic:uint16]{uint16}]: \label{state:FrequencyDomain:N}
    The number of data points.
  \item frequency [\hyperref[generic:float32]{float32}[]]:
    \label{state:FrequencyDomain:frequency}
    The frequency represented in Hz.
  \item real [\hyperref[generic:float32]{float32}[]]:
    \label{state:FrequencyDomain:real}
    The real component of thre frequency contribution.
  \item img [\hyperref[generic:float32]{float32}[]]:
    \label{state:FrequencyDomain:img}
    The imaginary component of thre frequency contribution.
  \item magnitude [\hyperref[generic:float32]{float32}[]]:
    \label{state:FrequencyDomain:magnitude}
    The magnitude of the frequency component.
\end{enumerate}


\newpage

\section{MIS of \wss{Module Name}} \label{mod:Module} \wss{Use labels for
  cross-referencing}

\wss{You can reference SRS labels, such as R.}

\wss{It is also possible to use \LaTeX for hypperlinks to external documents.}

\subsection{Module}

\wss{Short name for the module}

\subsection{Uses}


\subsection{Syntax}

\subsubsection{Exported Constants}

\subsubsection{Exported Access Programs}

\begin{center}
\begin{tabularx}{\linewidth}{l X X X}
\hline
\textbf{Name} & \textbf{In} & \textbf{Out} & \textbf{Exceptions} \\
\hline
\wss{accessProg} & - & - & - \\
\hline
\end{tabularx}
\end{center}

\subsection{Semantics}

\subsubsection{State Variables}

\wss{Not all modules will have state variables.  State variables give the module
  a memory.}

\subsubsection{Environment Variables}

\wss{This section is not necessary for all modules.  Its purpose is to capture
  when the module has external interaction with the environment, such as for a
  device driver, screen interface, keyboard, file, etc.}

\subsubsection{Assumptions}

\wss{Try to minimize assumptions and anticipate programmer errors via
  exceptions, but for practical purposes assumptions are sometimes appropriate.}

\subsubsection{Access Routine Semantics}

\noindent \wss{accessProg}():
\begin{itemize}
\item transition: \wss{if appropriate} 
\item output: \wss{if appropriate} 
\item exception: \wss{if appropriate} 
\end{itemize}

\wss{A module without environment variables or state variables is unlikely to
  have a state transition.  In this case a state transition can only occur if
  the module is changing the state of another module.}

\wss{Modules rarely have both a transition and an output.  In most cases you
  will have one or the other.}

\subsubsection{Local Functions}

\wss{As appropriate} \wss{These functions are for the purpose of specification.
  They are not necessarily something that is going to be implemented
  explicitly.  Even if they are implemented, they are not exported; they only
  have local scope.}
  
\newpage
\section{MIS of Audio360 Engine \label{mod:Audio360Engine} - 
  \mref{MG-mAudio360Engine}}

\subsection{Module}

Orchestrates the overall audio processing by receiving raw input data and
managing \hyperref[mod:Audio360Engine_Uses]{module} communication.

\subsection{Uses}\label{mod:Audio360Engine_Uses}

\begin{enumerate}
  \item \hyperref[mod:Classification]{Classification Module}
  \item \hyperref[mod:DOA]{DOA Processor Module}
  \item \hyperref[mod:FaultManager]{Fault Manager Module} 
  \item \item \hyperref[mod:Logging]{Logging Module} 
\end{enumerate}

\subsection{Syntax}

\subsubsection{Exported Constants}

None

\subsubsection{Exported Access Programs}

\begin{center}
\begin{tabularx}{\linewidth}{l X X X}
\hline
\textbf{Name} & \textbf{In} & \textbf{Out} & \textbf{Exceptions} \\
\hline
runProgram & - & - & ProgramStartFailure, ProgramRunTimeFailure \\
\hline
\end{tabularx}
\end{center}

\subsection{Semantics}

\subsubsection{State Variables}

\begin{itemize}
  \item state [\hyperref[enum:Audio360State]{Audio360State}]:
    Determines the current state of the the Audio360 engine. Each state will run
    a specific module in \hyperref[mod:Audio360Engine_Uses]{used modules}.
\end{itemize}

\subsubsection{Environment Variables}

\begin{itemize}
  \item status [\hyperref[enum:Audio360Status]{Audio360Status}]:
    Status of the module. This encapsulates initialization, running, ready,
    or errored.
\end{itemize}

\subsubsection{Assumptions}

None

\subsubsection{Access Routine Semantics}

\noindent run():
\begin{itemize}
\item transition: The state machine in figure
  \ref{fig:audio360_engine_state_machine} outlines the transition of this
  module.

\begin{figure}[h!]
    \centering 
    \includegraphics[width=\textwidth]{diagrams/Audio360EngineStateMachine.png}
    \caption{Internal state machine of Audio360 Engine module.}
    \label{fig:audio360_engine_state_machine}
\end{figure}

\item output: None
\item exception: ProgramStartFailure, ProgramRunTimeFailure
\end{itemize}

\subsubsection{Local Functions}

None

\newpage

\section{MIS of Audio Sampling Module \label{mod:AudioSampling} -
  \mref{MG-mAudioSampling}}

\subsection{Module}
Provides sampling of audio data from a given source while preserving the order
of individual samples.


\subsection{Uses}

\begin{itemize}
  \item \hyperref[mod:MicInput]{Microphone Input Module}
  \item \hyperref[mod:MicDiag]{Microphone Diagnostic Module}
\end{itemize}

\subsection{Syntax}

\subsubsection{Exported Constants}

\begin{itemize}
  \item sampleWindowSize [\hyperref[generic:uint16]{uint16}]:
    \label{var:sampleWindowSize}
    The number of samples in a window. The value is a power of 2.
\end{itemize}

\subsubsection{Exported Access Programs}

\begin{center}
\begin{tabularx}{\linewidth}{l X X X}
\hline
\textbf{Name} & \textbf{In} & \textbf{Out} & \textbf{Exceptions} \\
\hline
sample & inputSource & - & InputSouceError \\
getSampledData & - & sampledData & - \\
\hline
\end{tabularx}
\end{center}

\subsection{Semantics}

\subsubsection{State Variables}

\begin{itemize}
  \item sampledData [\hyperref[generic:uint32]{uint32}[]]:
    \label{var:sampleData}  
    A cyclic array of size \hyperref[var:sampleWindowSize]{sampleWindowSize}
    that stored the sampled data.
  \item windowPosition [\hyperref[generic:uint32]{uint32}]:
    \label{var:windowPosition}  
    The current reference index position of the array storing sampled data.
    This denotes the starting point of the cyclic array. 
\end{itemize}

\subsubsection{Environment Variables}

None

\subsubsection{Assumptions}

None

\subsubsection{Access Routine Semantics}

\noindent sample():
\begin{itemize}
\item transition: 
  \begin{enumerate}
    \item $sampleData[windowPosition] := inputSource.newData$
    \item $((windowPosition + 1) \geq sampleWindowSize) \implies $
      $windowPosition:= 0~\vee \neg ((windowPosition + 1) \geq $
      $sampleWindowSize) \implies$ \\ $ windowPosition:= windowPosition + 1$
  \end{enumerate}

\item output: None
\item exception: InputSouceError
\end{itemize}

\noindent getSampledData():
\begin{itemize}
\item transition: None 
\item output: \hyperref[var:sampleData]{sampleData}
\item exception: None
\end{itemize}

\subsubsection{Local Functions}

None

\newpage
\section{MIS of Audio Spectral Leakage Prevention Module
  \label{mod:AudioSpectralLeakage} - \mref{MG-mAudioSpectralLeakage}}

\subsection{Module}

Applying windowing on audio signal to reduce
\hyperref[SRS-def:spectral_leakage]{spectral leakage} before
\hyperref[SRS-def:frequency_domain]{frequency domain} processing.

\subsection{Uses}

None

\subsection{Syntax}

\subsubsection{Exported Constants}

None

\subsubsection{Exported Access Programs}

\begin{center}
\begin{tabularx}{\linewidth}{l X X X}
\hline
\textbf{Name} & \textbf{In} & \textbf{Out} & \textbf{Exceptions} \\
\hline
applyFilter & sampleWindow & filteredWindow & - \\
\hline
\end{tabularx}
\end{center}

\subsection{Semantics}
\subsubsection{State Variables}

None

\subsubsection{Environment Variables}

None

\subsubsection{Assumptions}

None

\subsubsection{Access Routine Semantics}

\noindent applyFilter():
\begin{itemize}
\item transition: None
\item output:
  \begin{enumerate}
    \item $(\forall~i~|~0 \leq i \leq N - 1:~filteredWindow[i] := $
    $sampleWindow[i] * \sin(\dfrac{\pi * i}{N - 1})^2)$ \\
    such that $N = $ size of sampleWindow.
  \end{enumerate}
\item exception: None
\end{itemize}


\subsubsection{Local Functions}

None


\newpage

\section{MIS of Audio Normalizer Module \label{mod:AudioNormalizer} -
  \mref{MG-mAudioNormalizer}}

\subsection{Module}

Processes audio signals to maintain consistent amplitude across different
sources.

\subsection{Uses}

None

\subsection{Syntax}
\subsubsection{Exported Constants}

None

\subsubsection{Exported Access Programs}

\begin{center}
\begin{tabularx}{\linewidth}{l X X X}
\hline
\textbf{Name} & \textbf{In} & \textbf{Out} & \textbf{Exceptions} \\
\hline
normalize & sampleWindow, source & filteredWindow & - \\
\hline
\end{tabularx}
\end{center}

\subsection{Semantics}

\subsubsection{State Variables}

None

\subsubsection{Environment Variables}

None

\subsubsection{Assumptions}

Normalization factor is pre-deteremined and known for all given sources.

\subsubsection{Access Routine Semantics}

\noindent normalize():
\begin{itemize}
\item transition: None
\item output:
  \begin{enumerate}
    \item $(\forall~i~|~0 \leq i \leq N - 1:~filteredWindow[i] := $
    $\dfrac{sampleWindow[i]}{F_{source}})$ \\
    such that $N = $ size of sampleWindow and $F_{source}$ is normalization
    factor of a the sampleWindow source.
  \end{enumerate}
\item exception: None
\end{itemize}

\subsubsection{Local Functions}

None

\newpage
\section{MIS of Fast Fourier Transform - \mref{MG-mFFT}} \label{mod:FFT}

\subsection{Module}

Computes the discrete Fourier transform of the input signal to 
obtain its \hyperref[SRS-def:frequency_domain]{frequency domain}

\subsection{Uses}

\begin{itemize}
  \item \hyperref[mod:AudioSpectralLeakage]{Audio Spectral Leakage Prevention
    Module}
\end{itemize}

\subsection{Syntax}

\subsubsection{Exported Constants}

None

\subsubsection{Exported Access Programs}

\begin{center}
\begin{tabularx}{\linewidth}{l X X X}
\hline
\textbf{Name} & \textbf{In} & \textbf{Out} & \textbf{Exceptions} \\
\hline
signalToFrequency & signal & frequency & - \\
\hline
\end{tabularx}
\end{center}

\subsection{Semantics}

\subsubsection{State Variables}

None

\subsubsection{Environment Variables}

\begin{itemize}
  \item inputsize [uint32]: The size of the signal input, as in number of
     samples. This is required to interface with hardware acceleration methods
     on the microcontroller.
\end{itemize}

\subsubsection{Assumptions}

None

\subsubsection{Access Routine Semantics}

\noindent signalToFrequency():
\begin{itemize}
\item transition: None
\item output:
  \begin{enumerate}
    \item $(\forall~k~|~0 \leq k \leq N - 1:~frequency[k] := $
    $\sum_{n=0}^{N-1} x_n * e^{-2i \pi n \frac{k}{N}})$ \\
    such that $N = $ size of input signal.
  \end{enumerate}
\item exception: None
\end{itemize}


\subsubsection{Local Functions}

\begin{itemize}
  \item createOutput(): returns a
    \hyperref[struct:FrequencyDomain]{FrequencyDomain}. This function extract
    features from the FFT and stores it in the data structure.
\end{itemize}

\newpage

\section{MIS of Logging Module \label{mod:Logging} -
  \mref{MG-mLogging}}

\subsection{Module}

Provides centralized logging of data streams to designated output destinations
for debugging and monitoring.

\subsection{Uses}

\begin{itemize}
  \item \hyperref[mod:USART]{USART Communication Module}
  \item \hyperref[mod:SDInterface]{SD Card Interface Module}
\end{itemize}

\subsection{Syntax}
\subsubsection{Exported Constants}

None

\subsubsection{Exported Access Programs}

\begin{center}
\begin{tabularx}{\linewidth}{l X X X}
\hline
\textbf{Name} & \textbf{In} & \textbf{Out} & \textbf{Exceptions} \\
\hline
log & text & - & - \\
\hline
\end{tabularx}
\end{center}

\subsection{Semantics}

\subsubsection{State Variables}

None

\subsubsection{Environment Variables}

None

\subsubsection{Assumptions}

None

\subsubsection{Access Routine Semantics}

\noindent log():
\begin{itemize}
\item transition: None
\item output: input text is logged to output source (SD card or USART port)
\item exception: None
\end{itemize}

\subsubsection{Local Functions}

None

\newpage 




\newpage 

\section{MIS of Mel Filter Module \label{mod:MelFilter} - 
\mref{MG-mMelFilter}}

\subsection{Module}

Converts a Spectrogram into the equivalent Mel-Spectrogram by applying a bank 
of $n$ mel-scaled triangular filters to each frequency frame.

\subsection{Uses}

None

\subsection{Syntax}

\subsubsection{Exported Constants}

None

\subsubsection{Exported Access Programs}

\begin{center}
\begin{tabularx}{\linewidth}{l X X X}
\hline
\textbf{Name} & \textbf{In} & \textbf{Out} & \textbf{Exceptions} \\
\hline
applyMelFilterBank & spectrogram: 2D real array & mel-spectrogram: 2D 
real array & invalidSpectrogramDimension \\
\hline
\end{tabularx}
\end{center}

\subsection{Semantics}
\subsubsection{State Variables}

None

\subsubsection{Environment Variables}

None

\subsubsection{Assumptions}

\begin{itemize}
  \item The input spectrogram is the output of the 
  \hyperref[MG-mFFT]{FFT Module} across a constant buffer time. This is 
  arranged as a 2D matrix where the rows represent time frames and columns 
  represent frequency bins. This matrix can be represented as $S(t, k)$ where 
  $k$ is the number of frequency bins, and $t$ is the number of time frames. 
  \item The mel filterbank matrix is precomputed offline and stored as a read 
  only constant within the system. It is computed offline using pre-determined 
  information like Sampling Rate, FFT Size, maximum and minimum frequencies and 
  number of mel filters. This matrix can be represented as $H(m,k)$, where $m$ 
  is the number of mel-filters, $k$ is the number of frequency bins and $H(m,k)$ 
  is a weight that tells you how much frequency bin $k$ contributes to mel band 
  $m$.
\end{itemize}

\subsubsection{Access Routine Semantics}

\noindent applyMelFilterBank():
\begin{itemize}
\item \textbf{transition:}
  None (function doesn't have states)
\item \textbf{output:}
  Returns the equivalent mel-spectrogram, where each time frame is converted 
  from linear-frequency bin representation to mel-scaled frequency bins. This
  is done by:
  \begin{enumerate}
    \item Multiplying each spectrogram frame by the mel filterbank matrix. 
    \item Summing weighted energy contributions per mel filter. 
  \end{enumerate}
    \[
    E(t, m) = \sum_{k} S(t, k) \cdot H(m, k)
    \]
\item \textbf{exception:} 
  Invalid spectrogram dimension. If the dimension of the input spectrogram is 
  not $S(t,k)$ and perhaps something like $S(t,p)$ where $p \neq k$, then the 
  matrix multiplication would throw an invalid dimension error, which must be 
  exposed from the module as invalid spectrogram dimension as well. This would 
  only happens if the number of frequency bins in the computted FFT is not 
  equal to the number of frequency bins in the pre-computed mel filterbank 
  matrix. 
\end{itemize}

\subsubsection{Local Functions}

None

\newpage 

\section{MIS of Discrete Cosine Transform Module \label{mod:DCT} - 
\mref{MG-mDCT}}

\subsection{Module}

Converts a mel-spectrogram into a 2D array containing the sequence of Mel 
Frequency Cepstral Coefficients (MFCC) vectors by applying the Discrete Cosine Transform (DCT) to each time frame. This reduces overlap between 
frequency bands that carry similar information and ensures each coefficient is 
uncorrelated to the other one. Pre-processing step that makes it easier to 
perform Principle Component Analysis later on. 

\subsection{Uses}

None

\subsection{Syntax}

\subsubsection{Exported Constants}

None

\subsubsection{Exported Access Programs}

\begin{center}
\begin{tabularx}{\linewidth}{l X X X}
\hline
\textbf{Name} & \textbf{In} & \textbf{Out} & \textbf{Exceptions} \\
\hline
applyDCT & mel-spectrogram: 2D real array & mfcc: 2D 
real array & invalidDimensions \\
\hline
\end{tabularx}
\end{center}

\subsection{Semantics}

\subsubsection{State Variables}

None

\subsubsection{Environment Variables}

None

\subsubsection{Assumptions}

\begin{itemize}
  \item The input mel-spectrogram is the output of the 
  \hyperref[MG-mMelFilter]{Mel Filter Module}. This input will be of size 
  $E(t, m)$, where $t$ is the number of time frames, and $m$ is the number of 
  mel-frequency bins. 
  \item The DCT transform matrix $D(c,m)$ will be computed offline, where 
  $c$ is the desired number of MFCC coefficients for each time frame, and $m$ is 
  the number of mel-frequency bands. $D(c,m)$ represents the unique cosine wave 
  that is associated with the $c^{\text{th}}$ coefficient and $m^{\text{th}}$
  band. A combination of these waves will be used to uniquely represent the 
  spectrogram at time frame $t$. 

    \[
    D(c,m) = \cos\!\left(\frac{\pi c}{M}(m - 0.5)\right)
    \]
  
  \item Only 13 MFCC coefficients will be calculated for each time frame $t$. 
  
\end{itemize}

\subsubsection{Access Routine Semantics}

\noindent applyDCT():
\begin{itemize}
\item \textbf{transition:}
  None (function doesn't have states)
\item \textbf{output:}
  Returns a matrix \textit{mfcc(t,c)}, such that for each time frame $t$ and each 
  MFCC coefficient index $c$, the value of the matrix at $mfcc(t,c)$ can be
  computed as
      \[
      mfcc(t, c) = \sum_{m=1}^{M} E(t, m) \cdot DCT(c, m)
      \]

  Where $E(t, m)$ represents the mel-spectrogram at time frame $t$ and 
  mel-frequency $m$. And $DCT(c,m)$ represents the DCT transform matrix that 
  was computed offline. 
\item \textbf{exception:} 
  Invalid spectrogram dimension. If the dimension of the input spectrogram is 
  not $E(c,m)$ and perhaps something like $E(c,p)$ where $m \neq p$ then the 
  matrix multiplication would throw an invalid dimension error, which must be 
  exposed from the module as invalid spectrogram dimension as well. This would 
  only happens if for some reason the number of mel-frequency bands returned 
  by the \hyperref[MG-mMelFilter]{Mel Filter Module} is not the same as the 
  number of mel-frequency bands in the DCT matrix computation.
\end{itemize}

\subsubsection{Local Functions}

None

\newpage 

\section{MIS of Principle Component Analysis Module \label{mod:PCA} - 
\mref{MG-mPCA}}

\subsection{Module}

Principle Component Analysis (PCA) highlights the features with the greatest 
variance from an input feature matrix. In this case, uses the MFCC feature 
vector for each time frame to find the direction of greatest variance. This 
helps with eliminating noise and unimportant information.

\subsection{Uses}

Non

\subsection{Syntax}

\subsubsection{Exported Constants}

None

\subsubsection{Exported Access Programs}

\begin{center}
\begin{tabularx}{\linewidth}{l X X X}
\hline
\textbf{Name} & \textbf{In} & \textbf{Out} & \textbf{Exceptions} \\
\hline
applyPCA & mfcc: 2D real array & pcaFeatures: 2D 
real array & invalidDimensions \\
\hline
\end{tabularx}
\end{center}

\subsection{Semantics}

\subsubsection{State Variables}

None

\subsubsection{Environment Variables}

None

\subsubsection{Assumptions}

\begin{itemize}
  \item The input MFCC matrix is the output of the
  \hyperref[MG-mDCT]{Discrete Cosine Transform Module}. This input will be of 
  size $mfcc(t,c)$, where $t$ is the number of time frames, and $c$ is the 
  number of coefficients. 
  \item The PCA mean vector and PCA projection matrix will be derived offline 
  during the system training and stored as read-only constants. The PCA mean 
  vector will be of size $(1,c)$, and will be computed as the average MFCC 
  vectors across all training samples. 
    \[
    \bar{x} = \frac{1}{T} \sum_{t=1}^{T} x(t)
    \]

  The PCA projection matrix will be of size $(K, c)$, where $K$ is the number 
  of eigenvectors with the maximum variance and $c$ is the number of MFCC 
  coefficients. These eigenvectors are extracted from the following covariance 
  matrix. 

    \[
    C = \frac{1}{T} \sum_{t=1}^{T} 
    \left(x(t) - \bar{x}\right)
    \left(x(t) - \bar{x}\right)^{T}
    \]
   
  
\end{itemize}

\subsubsection{Access Routine Semantics}

\noindent applyPCA():
\begin{itemize}
\item \textbf{transition:}
  None (function doesn't have states)
\item \textbf{output:}
  Returns a matrix \textit{pca(t, K)}, where $t$ represents the number of time 
  frames and $K$ represents the number of PCA components retained (proportional 
  to the number of eignevectors). Essentially, each input mfcc matrix is 
  projected onto the vectors of maximum variance, therefore extracting only the 
  most important features of the MFCC based on training. This projection can be 
  computed using the following matrix multiplication. 

  \[
  \text{pca} = \left( mfcc - \mathbf{1}\bar{x} \right) P^{T}
  \]

  Where $P$ represents the projection matrix, $\bar{x}$ represents the PCA 
  mean vector, and $mfcc$ represents the full mfcc matrix. 
\item \textbf{exception:} 
  Invalid mfcc dimension. If the dimension of the input mfcc matrix is 
  not $(t,c)$ and perhaps something like $(t,p)$ where $c \neq p$ then the 
  matrix multiplication would throw an invalid dimension error, which must be 
  exposed from the module as invalid mfcc dimension as well. This would 
  only happens if for some reason the number of coefficients returned 
  by the \hyperref[MG-mDCT]{Discrete Cosine Transform} is not the same as the 
  number of coefficients in the projection matrix.
\end{itemize}

\subsubsection{Local Functions}

None

\newpage 

\section{MIS of Linear Discriminant Analysis Module \label{mod:LDA} - 
\mref{MG-mLDA}}

\subsection{Module}

Classifies given feature vector into one of the predefined set of
sound classes. This is module is optimized for class seperation. 
The feature vector in this case is already narrowed down to the 
features with the most variance using the \hyperref[MG-mPCA]{PCA} 
module. 

\subsection{Uses}

None

\subsection{Syntax}

\subsubsection{Exported Constants}

None

\subsubsection{Exported Access Programs}

\begin{center}
\begin{tabularx}{\linewidth}{l X X X}
\hline
\textbf{Name} & \textbf{In} & \textbf{Out} & \textbf{Exceptions} \\
\hline
applyLDA & pcaFeatures: 2D real array & classLabels: 1D 
real array & invalidDimensions \\
\hline
\end{tabularx}
\end{center}

\subsection{Semantics}

\subsubsection{State Variables}

None

\subsubsection{Environment Variables}

None

\subsubsection{Assumptions}

\begin{itemize}
  \item The input pcaFeatures is the output of the
  \hyperref[MG-mPCA]{Principle Component Analysis Module}. This input will be 
  of size $(t,K)$, where $t$ is the number of time frames, and $K$ is 
  number of features selected with the highest variances from PCA.
  \item The LDA projection matrix, and class weight parameters are 
  computed offline using labelled training data. They are stored 
  as read-only constants in the system. The LDA projection matrix is of size 
  $(K,C-1)$, where $C$ is the number of classes, and $K$ is the number of
  variance eigenvectors. In essence, this matrix maps the input feature space 
  into the discriminant subspace where class seperation is maximized.
  
\end{itemize}

\subsubsection{Access Routine Semantics}

\noindent applyLDA():
\begin{itemize}
\item \textbf{transition:}
  None (function doesn't have states)
\item \textbf{output:}

  Returns a vector of class labels, one per time frame. 

  Each feature vector from PCA is projected onto the discriminant space using 
  the LDA projection matrix using the following multiplication. 

  \[
  z(t,:) = pca(t,:) \cdot W_{\text{LDA}}
  \]

  Where $pca(t,:)$ represents the PCA eigenvectors at time $t$, and 
  $W_{\text{LDA}}$ represents the LDA projection matrix. 

  Classification is then performed using linear discriminant functions for 
  each class $j$, where $w_j$ and $b_j$ is the weight and bias matricies 
  that were pre-computed offline during training for class $j$. 

  \[
  g_j(t) = w_j^{T} z(t,:) + b_j
  \]

  The predicted class for that time frame is then assigned as the class with 
  the maximum value. 

  \[
  \text{classLabel}(t) = \arg\max_{j} \, g_j(t)
  \]

  The output of the module is then a vector of length t, representing the best 
  classification for all time frames. 


\item \textbf{exception:} 
  Invalid feature space dimension. If the dimension of the input feature matrix 
  is not $(t,K)$ and perhaps something like $(t,p)$ where $K \neq p$ then the 
  matrix multiplication would throw an invalid dimension error, which must be 
  exposed from the module as invalid feature dimension as well. This would 
  only happens if for some reason the number of eigenvectors returned 
  by the \hyperref[MG-mPCA]{PCA} module is not the same as the 
  number of eignvectors in the precomputed LDA matrix.
\end{itemize}

\subsubsection{Local Functions}

None

\newpage 

\section{MIS of Classification Module \label{mod:Classification} - 
\mref{MG-mClassification}}

\subsection{Module}

Orchestrates the overall audio classification flow by receiving signals in the 
frequency domain and encapsulating the complexity of the various modules 
involved. 

\subsection{Uses}

\begin{enumerate}
  \item \hyperref[mod:AudioSampling]{Audio Sampling Module}
  \item \hyperref[mod:AudioNormalizer]{Audio Normalizer Module}
  \item \hyperref[mod:FFT]{Fast Fourier Transform Module}
  \item \hyperref[mod:MelFilter]{Mel Filter Module}
  \item \hyperref[mod:DCT]{Discrete Cosine Transform Module}
  \item \hyperref[mod:PCA]{Principle Component Analysis Module}
  \item \hyperref[mod:LDA]{Linear Discrete Analysis Module}
  \item 
\end{enumerate}

\subsection{Syntax}

\subsubsection{Exported Constants}

None

\subsubsection{Exported Access Programs}

\begin{center}
\begin{tabularx}{\linewidth}{l X X X}
\hline
\textbf{Name} & \textbf{In} & \textbf{Out} & \textbf{Exceptions} \\
\hline
classify & spectrogram: 2D real array & - & ClassificationStartFailure, 
ClassificationRunTimeFailure \\
\hline
\end{tabularx}
\end{center}

\subsection{Semantics}

\subsubsection{State Variables}

\begin{itemize}
  \item state[classificationState]: Stores the current classification. 
  Will continuously be updated at runtime after recieving continuous audio 
  signals. 
  \item state[classificationStatus]: Stores the confidence of the 
  classification that was done. Values for status include high, medium or low 
  confidence. 
\end{itemize}

\subsubsection{Environment Variables}

\begin{itemize}
  \item state[status]: Encapsulates initialization, running, 
  ready or errored status of the module.  
\end{itemize}

\subsubsection{Assumptions}
None.

\subsubsection{Access Routine Semantics}

\noindent applyLDA():
\begin{itemize}
\item \textbf{transition:}
  \hyperref[fig:classification_state_machine]{State machine} below outlines the 
  transition of this module.

\begin{figure}[h!]
    \centering 
    \includegraphics[width=\textwidth]{diagrams/ClassificationStateMachine.png}
    \caption{Internal state machine of Classification module.}
    \label{fig:classification_state_machine}
\end{figure}

\item \textbf{output:} None
\item \textbf{exception:} ClassificationStartFailure, 
ClassificationRunTimeFailure

\end{itemize}

\subsubsection{Local Functions}

None

\newpage 

\section{MIS of Fault Manager Module - \mref{MG-mFaultManager}
  \label{mod:FaultManager}} 

\subsection{Module}

Monitors system health and manages critical faults to maintain core
functionality.

\subsection{Uses}

\begin{enumerate}
  \item \hyperref[mod:MicDiag]{Microphone Diagnostic Module}
  \item \hyperref[mod:Classification]{Classification Module}
  \item \hyperref[mod:DOA]{DOA Processor Module}
  \item \hyperref[mod:AudioAnomaly]{Audio Anomaly Detection Module}
\end{enumerate}

\subsection{Syntax}

\subsubsection{Exported Constants}

None

\subsubsection{Exported Access Programs}

\begin{center}
\begin{tabularx}{\linewidth}{l X X X}
\hline
\textbf{Name} & \textbf{In} & \textbf{Out} & \textbf{Exceptions} \\
\hline
runFaultAnalysis & - & - & - \\
\hline
\end{tabularx}
\end{center}

\subsection{Semantics}

\subsubsection{State Variables}

\begin{itemize}
  \item faultState [\hyperref[enum:FaultState]{FaultState}]:
    \label{faultState}
    The current fault state of the overall software system.
\end{itemize}

\subsubsection{Environment Variables}

\begin{itemize}
  \item \hyperref[faultState]{faultState}: This variable is shared with the
    display, to notifiy user of the internal status of the software system.
\end{itemize}

\subsubsection{Assumptions}

Fault Manager module itself does not run into any critical faults.

\subsubsection{Access Routine Semantics}

\noindent faultState():
\begin{itemize}
\item transition: update the value of faultState depending on the overal health
  of the software system.
\item output: None
\item exception: None
\end{itemize}

\subsubsection{Local Functions}

None
\newpage

\bibliographystyle{IEEEtran}
\bibliography{../../../refs/References}

\newpage

\section{Appendix} \label{Appendix}

\wss{Extra information if required}

\newpage{}

\section*{Appendix --- Reflection}

\begin{enumerate}
  \item What went well while writing this deliverable? 
  
  \textbf{Sathurshan:} The system was decomposed into modules that was small
  enough. This allowed the team to easily split up the work without having
  much dependencies on each other. Furthermore, writing the MIS has helped the
  team further align with design decisions and formally document.

  \item What pain points did you experience during this deliverable, and how
    did you resolve them?

  \textbf{Sathurshan:} The main pain point was the deliverable deadline as the
  team was asked to finish the first revision in a week while preparing for the
  proof of concept of demo. The team expressed the concerns with the professor,
  and fornately received an extension allowing us to put more effort into this
  document.

  \item Which of your design decisions stemmed from speaking to your client(s)
  or a proxy (e.g. your peers, stakeholders, potential users)? For those that
  were not, why, and where did they come from?

  \textbf{Sathurshan:} The supervisor mentionned the difficulty of getting
  access to hardware that will meet our software specification. As a result, we
  designed the system such that the software is portable and can take input
  from different sources. This ensures that the capstone project can still
  be successful in the case we are not able to find the right hardware within
  our budget.

  \item While creating the design doc, what parts of your other documents (e.g.
  requirements, hazard analysis, etc), it any, needed to be changed, and why?

  \textbf{Sathurshan:} SRS needed to be updated. It was updated because part
  of writing this document required reviewing the SRS. From this, I have found
  parts of the requirements that can be improved based on recent better
  understanding of the system.

  \item What are the limitations of your solution?  Put another way, given
  unlimited resources, what could you do to make the project better?
  (LO\_ProbSolutions)
  
  \textbf{Sathurshan:} The limitation of the solution is compute and memory
  power. Given further time, the team would be able to design a system that
  is more low level and is optimized in accomplishing the tasks that are
  required.
  
  \item Give a brief overview of other design solutions you considered.  What
  are the benefits and tradeoffs of those other designs compared with the chosen
  design?  From all the potential options, why did you select the documented
  design? (LO\_Explores)

  \textbf{Sathurshan:} Personally, there was not a lot of time to think about
  other designs. The team has been implementing the software before this
  document was created since the proof of concept is one week after this
  document is due. Thus, the team already considered and analyzed high level
  designs months ago and is too far back to document them.

\end{enumerate}


\end{document}