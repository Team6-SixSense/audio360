\documentclass[12pt, titlepage]{article}

\usepackage{amsmath, mathtools}

\usepackage{amsfonts}
\usepackage{amssymb}
\usepackage{graphicx}
\usepackage{colortbl}
\usepackage{xr}
\usepackage{hyperref}
\usepackage{longtable}
\usepackage{xfrac}
\usepackage{tabularx}
\usepackage{float}
\usepackage{siunitx}
\usepackage{booktabs}
\usepackage{multirow}
\usepackage[section]{placeins}
\usepackage{caption}
\usepackage{fullpage}
\usepackage{cite}
\usepackage{breqn}

\hypersetup{
bookmarks=true,     % show bookmarks bar?
colorlinks=true,       % false: boxed links; true: colored links
linkcolor=red,          % color of internal links (change box color with linkbordercolor)
citecolor=blue,      % color of links to bibliography
filecolor=magenta,  % color of file links
urlcolor=cyan          % color of external links
}

\usepackage{array}

\externaldocument[SRS-]{../../SRS/SRS}
\externaldocument[MG-]{../SoftArchitecture/MG}
\newcommand{\mref}[1]{M\ref{#1}}

\input{../../Comments}
%% Common Parts

\newcommand{\progname}{Audio360} % PUT YOUR PROGRAM NAME HERE
\newcommand{\authname}{Team 6, SixSense
\\ Omar Alam
\\ Sathurshan Arulmohan
\\ Nirmal Chaudhari
\\ Kalp Shah
\\ Jay Sharma
} % AUTHOR NAMES                  

\usepackage{hyperref}
    \hypersetup{colorlinks=true, linkcolor=blue, citecolor=blue, filecolor=blue,
                urlcolor=blue, unicode=false}
    \urlstyle{same}
                                


\begin{document}

\title{Module Interface Specification for \progname{}}

\author{\authname}

\date{\today}

\maketitle

\pagenumbering{roman}

\section{Revision History}

\begin{tabularx}{\textwidth}{p{3cm}p{2cm}X}
\toprule {\bf Date} & {\bf Version} & {\bf Notes}\\
\midrule
2025-11-13 & 1.0 & Initial write-up\\
\bottomrule
\end{tabularx}

~\newpage

\section{Symbols, Abbreviations and Acronyms}

\begin{table}[H]
  \centering
  \begin{tabular}{l l} 
    \toprule		
    \textbf{symbol} & \textbf{description}\\
    \midrule 
    \progname & 360 Audio analysis system on smart glasses\\
    DOA & Direction of Arrival \\
    FR & Functional Requirement\\
    FFT & Fast Fourier Transform \\
    ICA & Independent Component Analysis \\
    M & Module \\
    MG & Module Guide \\
    MIS & Module Interface Specification \\
    NFR & Non-functional Requirement\\
    R & Requirement\\
    SPI & Serial Peripheral Interface \\
    SRS & Software Requirements Specification\\
    USART & Universal Synchronous/Asynchronous Receiver/Transmitter \\
    \bottomrule
  \end{tabular}\\
  \caption{Symbols, abbreviations and acronyms used in the MIS document.}
\end{table}

See SRS Documentation at
\hyperref[SRS-sec:symbols]{Symbols, Abbreviations, and Acronyms}
for a complete table used in \progname.

\newpage

\tableofcontents

\newpage

\pagenumbering{arabic}

\section{Introduction}

The following document details the Module Interface Specifications for
\progname{}, an assistive device system for smart glasses that provides
real-time visual indications of sound source locations and classifications
to aid individuals who are deaf or hard of hearing.

Complementary documents include the System Requirement Specifications
and Module Guide.  The full documentation and implementation can be
found at \url{https://github.com/Team6-SixSense/audio360}.

\section{Notation}

The structure of the MIS for modules comes from \cite{HoffmanAndStrooper1995},
with the addition that template modules have been adapted from
\cite{GhezziEtAl2003}.  The mathematical notation comes from Chapter 3 of
\cite{HoffmanAndStrooper1995}.  For instance, the symbol := is used for a
multiple assignment statement and conditional rules follow the form $(c_1
\Rightarrow r_1 | c_2 \Rightarrow r_2 | ... | c_n \Rightarrow r_n )$.

The following table summarizes the primitive data types used by \progname. 

\begin{center}
\renewcommand{\arraystretch}{1.2}
\noindent 
\begin{tabular}{l l p{7.5cm}} 
\toprule 
\textbf{Data Type} & \textbf{Notation} & \textbf{Description}\\ 
\midrule
character & char & a single symbol or digit\\
integer & $\mathbb{Z}$ & a number without a fractional component in
  (-$\infty$, $\infty$) \\
natural number & $\mathbb{N}$ & a number without a fractional component in
  [1, $\infty$) \\
real & $\mathbb{R}$ & any number in (-$\infty$, $\infty$)\\
array/list & x[] & Ordered list of type x. \\
negate & $\neg$ & Logical math NOT.\\
and & $\wedge$ & Logical math AND \\
or & $\vee$ & Logical math OR \\
implies & $\implies$ & Logical math implies \\
assignment & $:=$ & For A := B. B is assigned to A. \\
for all & $\forall$ & Referencing all items. For example:
  $(\forall~variable~|~condition:~statement)$ \\
\bottomrule
\end{tabular} 
\end{center}

\noindent
The specification of \progname \ uses some derived data types: sequences, strings, and
tuples. Sequences are lists filled with elements of the same data type. Strings
are sequences of characters. Tuples contain a list of values, potentially of
different types. In addition, \progname \ uses functions, which
are defined by the data types of their inputs and outputs. Local functions are
described by giving their type signature followed by their specification.

\section{Module Decomposition}

The following table is taken directly from the Module Guide document for this project.

\begin{table}[H]
\centering
\begin{tabular}{p{0.3\textwidth} p{0.6\textwidth}}
\toprule
\textbf{Level 1} & \textbf{Level 2}\\
\midrule

\multirow{4}{0.3\textwidth}{Hardware-Hiding Module}
& Microphone Input Module\\ 
& USART Communication Module \\
& SD Card Module \\
& SD Card Interface Module \\
\midrule

\multirow{2}{0.3\textwidth}{Behaviour-Hiding Module}
& Audio360 Enginer Module\\
& Visualization Module\\
\midrule

\multirow{15}{0.3\textwidth}{Software Decision Module}
& Audio Generation Module\\
& DOA Processor Module\\
& Audio Sampling Module\\
& Audio Spectral Leakage Prevention Module\\
& Audio Normalizer Module\\
& Audio Anomaly Detection Module\\
& Fast Fourier Transform Module\\
& Logging Module\\
& Fault Manager Module\\
& Mel Filter Module\\
& Discret Cosine Transform Module \\
& Principle Component Analysis Module \\
& Linear Discriminant Analysis Module \\
& Classification Module \\
& Independent Component Analysis\\
\bottomrule
\end{tabular}
\caption{Module Hierarchy}
\label{TblMH}
\end{table}

\newpage
\section{Data Types}

\subsection{Generics} \label{sec:generics}

\subsubsection{bool}\label{generic:bool}
Logical 0 or 1.

\subsubsection{int8}\label{generic:int8}
8 bits signed integer ($\mathbb{Z}$).

\subsubsection{int16}\label{generic:int16}
16 bits signed integer ($\mathbb{Z}$).

\subsubsection{int32}\label{generic:int32}
32 bits signed integer ($\mathbb{Z}$).

\subsubsection{int64}\label{generic:int64}
64 bits signed integer ($\mathbb{Z}$).

\subsubsection{uint8}\label{generic:uint8}
8 bits unsigned integer ($\mathbb{N}$).

\subsubsection{uint16}\label{generic:uint16}
16 bits unsigned integer ($\mathbb{N}$).

\subsubsection{uint32}\label{generic:uint32}
32 bits unsigned integer ($\mathbb{N}$).

\subsubsection{uint64}\label{generic:uint64}
64 bits unsigned integer ($\mathbb{N}$).

\subsubsection{float32}\label{generic:float32}
32 bits floating point ($\mathbb{R}$).

\subsubsection{float64}\label{generic:float64}
64 bits floating point ($\mathbb{R}$).

\subsubsection{string}\label{generic:string}
Character stream.

\newpage
\subsection{Enums}
\subsubsection{Audio360State}\label{enum:Audio360State}
\begin{enumerate}
  \item AudioClassificationProcess:
    \label{enum:Audio360State:AudioClassificationState}
    State when audio classification is running.
  \item DirectionalAnalysisProcess:
    \label{enum:Audio360State:DirectionalAnalysisState}
    State when directional analysis is running.
  \item OutputProcess:
    \label{enum:Audio360State:OutputProcessState}
    State when output processing is running.
\end{enumerate}

\subsubsection{Audio360Status}\label{enum:Audio360Status}
\begin{enumerate}
  \item Uninitialized: \label{enum:Audio360Status:Uninitialized}
    Audio360 Engine is not initialized.
  \item Initialized: \label{enum:Audio360Status:Initialized}
    Audio360 Engine is initialized, but not ready.
  \item Ready: \label{enum:Audio360Status:Ready}
    Audio360 Engine ready for requests.
  \item Running: \label{enum:Audio360Status:Running}
    Audio360 Engine is running. Can not accept new requests.
  \item Error: \label{enum:Audio360Status:Error}
    Audio360 Engine is stuck at an unhandled error.
\end{enumerate}

\subsubsection{Audio360Status}\label{enum:FaultState}
\begin{enumerate}
  \item NoFault: \label{enum:FaultState:NoFault}
    No fault state.
  \item MicrophoneXFault: \label{enum:FaultState:MicFault}
    Fault with microphone X.
  \item AudioClassificationFault: \label{enum:FaultState:ClassificationFault}
    Fault with audio classification.
  \item directionalAnalsysiFault: \label{enum:FaultState:DOAFault}
    Fault with the directional analysis.
\end{enumerate}


\newpage
\subsection{Data Structures}
\subsubsection{FrequencyDomain}\label{struct:FrequencyDomain}
\begin{enumerate}
  \item N [\hyperref[generic:uint16]{uint16}]: \label{state:FrequencyDomain:N}
    The number of data points.
  \item frequency [\hyperref[generic:float32]{float32}[]]:
    \label{state:FrequencyDomain:frequency}
    The frequency represented in Hz.
  \item real [\hyperref[generic:float32]{float32}[]]:
    \label{state:FrequencyDomain:real}
    The real component of thre frequency contribution.
  \item img [\hyperref[generic:float32]{float32}[]]:
    \label{state:FrequencyDomain:img}
    The imaginary component of thre frequency contribution.
  \item magnitude [\hyperref[generic:float32]{float32}[]]:
    \label{state:FrequencyDomain:magnitude}
    The magnitude of the frequency component.
\end{enumerate}

\subsubsection{CALLBACK\_MIC\_DATA} \label{ds:cb-mic-data}
\begin{enumerate}
  \item func\_callback : ([MIC\_INPUT\_BUFFER\_SIZE] of int16) \(\rightarrow\ \{0, 1\}\)
  Defines the callback function type for microphone data subscription.
 
\end{enumerate}

\subsubsection{MIC\_INPUT\_CALLBACK} \label{ds:mic-input-callback}
\begin{enumerate}
  \item pair of (id: \(\mathbb{N}\), callback:
    \hyperref[ds:cb-mic-data]{CALLBACK\_MIC\_DATA})
\end{enumerate}

\subsubsection{Dictionary} \label{ds:dict}
\begin{enumerate}
  \item For any type $T_1$, $T_2$ from
  \hyperref[sec:generics]{generic data types}, $T_1$ (key) maps to $T_2$
  (value). Key $:= set~of~T_1$.
\end{enumerate}

\subsubsection{List} \label{ds:list}
\begin{enumerate}
  \item For any type $T$ from \hyperref[sec:generics]{generic data types}, a 
    list is a collection of ordered objects of type $T$.
\end{enumerate}


\newpage

\section{MIS of Microphone Input Module \label{mod:MicInput} - \mref{MG-mMicInput}}

\subsection{Module}
This module handles the task of interfacing and receiving audio sample
data from the microphone array connected to the microcontroller and providing
a buffered stream to modules fetching the stream.

\subsection{Uses}
\begin{enumerate}
  \item Serial Audio Interface (SAI) Driver - STM32 HAL Library \cite{STM_USER_MANUAL}
  \item Direct Memory Access (DMA) Driver - STM32 HAL Library \cite{STM_USER_MANUAL}
  \item \hyperref[mod:MicDiag]{Microphone Diagnostic Module}
  \item \hyperref[ds:cb-mic-data]{CALLBACK\_MIC\_DATA} 
  \item SAI\_HandleTypeDef - STM32 HAL Library \cite{STM_USER_MANUAL}
\end{enumerate}

\subsection{Syntax}

\subsubsection{Exported Constants}

\begin{itemize}
  \item MIC\_INPUT\_IDLE = 0
  \item MIC\_INPUT\_ERROR = 1
  \item MIC\_INPUT\_RECEIVED = 2
  \item MIC\_INPUT\_BUFFER\_SIZE = 4096
  \item MIC\_INPUT\_SAMPLE\_RATE = 16000
  \item NUM\_MICROPHONES = 4
\end{itemize}

\subsubsection{Exported Access Programs}


\begin{center}
\begin{tabularx}{\linewidth}{l X X X}
\hline
\textbf{Name} & \textbf{In} & \textbf{Out} & \textbf{Exceptions} \\
\hline
initMicInput & - & \hyperref[generic:uint8]{uint8} & MicInputInitFailure \\
\hline
subscribeMicInput & callback: \hyperref[ds:cb-mic-data]{CALLBACK\_MIC\_DATA}
& subscriptionCB: \hyperref[ds:mic-input-callback]{MIC\_INPUT\_CALLBACK} & MicInputSubscriptionFailure \\
\hline
unsubscribeMicInput & subscriptionID: \(\mathbb{N}\) & - & MicInputUnsubscriptionFailure \\
\hline
micInput\_DMA\_Rx & handle: SAI\_HandleTypeDef* & - & - \\
\hline
micInput\_diagnostic\_error & errorCode: \(\mathbb{N}\) & - & - \\
\end{tabularx}
\end{center}

\subsection{Semantics}

\subsubsection{State Variables}
\begin{itemize}
  \item micInputStatus : \(\mathbb{N}\)
  \item subscriptions: sequence of \hyperref[ds:mic-input-callback]{MIC\_INPUT\_CALLBACK}
  \item nextSubscriptionID: \(\mathbb{N}\)
\end{itemize}

\subsubsection{Environment Variables}
\begin{itemize}
  \item audioBuffer: sequence [MIC\_INPUT\_BUFFER\_SIZE \(\times\) 
  (MIC\_INPUT\_SAMPLE\_RATE \(\times\) 2) \(\times\) 
  NUM\_MICROPHONES] of \(\mathbb{N}\) 
\end{itemize}

\subsubsection{Assumptions}
\begin{itemize}
  \item Callback functions provided in subscribeMicInput are valid function pointers.
\end{itemize}

\subsubsection{Access Routine Semantics}
\noindent initMicInput():
\begin{itemize}
\item transition: micInputStatus := MIC\_INPUT\_IDLE \(\wedge\) nextSubscriptionID := 0 \(\wedge\)
subscriptions := \(\langle\ \rangle\) \(\wedge\ (\forall i \in \mathbb{N} | 0 \le i < |\text{audioBuffer}|: \text{audioBuffer}[i] := 0)\) 
\(\wedge\) Initialize SAI and DMA drivers \(\wedge\) \hyperref[mdiag:subscribe]{subscribeDiagnostics(micInput\_diagnostic\_error)}
\item output: micInputStatus
\item exception: exc := MicInputInitFailure if SAI or DMA initialization fails.
\end{itemize}
\noindent subscribeMicInput(callback):
\begin{itemize}
\item transition: nextSubscriptionID := nextSubscriptionID + 1 \(\wedge\) 
subscriptions := subscriptions \(\parallel\) \(\langle\) pair of (id: 
nextSubscriptionID, callback: callback) \(\rangle\)

\item output: subscriptionCB := pair of (id: nextSubscriptionID, callback: callback) 
\item exception: exc := MicInputSubscriptionFailure if subscription fails for 
any reason.
\end{itemize}

\noindent unsubscribeMicInput(subscriptionID):
\begin{itemize}
\item transition: isValidCallback(subscriptionID) \(\Rightarrow\) \\ 
subscriptions := (subscriptions
  - \{cb: MIC\_INPUT\_CALLBACK \(\vert\) cb \(\in\) subscriptions: 
  cb.id \(\equiv\) subscriptionID\})
\item output: None
\item exception: exc := \(\neg\) isValidCallback(subscriptionID) \(\Rightarrow\) MicInputUnsubscriptionFailure
\end{itemize}

\noindent micInput\_DMA\_Rx(handle):
\begin{itemize}
  \item transition: (handle.ErrorCode \(\equiv\) HAL\_SAI\_ERROR\_NONE) \(\Rightarrow\) \\
  micInputStatus := MIC\_INPUT\_RECEIVED \(\wedge\)
  (\(\forall\) cb: MIC\_INPUT\_CALLBACK \(\vert\) cb \(\in\) subscriptions : 
  cb.callback(audioBuffer))
  \item output: None
  \item exception: None
\end{itemize}

\noindent micInput\_diagnostic\_error(errorCode):
\begin{itemize}
  \item transition: (errorCode \(\not\equiv\) 0) \(\Rightarrow\) micInputStatus := MIC\_INPUT\_ERROR
  \item output: None
  \item exception: None
\end{itemize} 


\subsubsection{Local Functions}
\begin{itemize}
  \item isValidCallback: \(\mathbb{N} \rightarrow\) \hyperref[generic:bool]{boolean}\\
  isValidCallback(id) \(\equiv\ (\exists \text{ cb}: \text{MIC\_INPUT\_CALLBACK } \vert  
  \text{ cb} \in \text{subscriptions} : cb.id \equiv id)\) 
\end{itemize}

\newpage

\section{MIS of USART Communication Module \label{mod:UsartComm} - \mref{MG-mUSART}}
\subsection{Module}
This module handles USART communication between the microcontroller and an 
external device.

\subsection{Uses}
\begin{enumerate}
  \item USART Driver - STM32 HAL Library \cite{STM_USER_MANUAL}
  \item USART\_Init() - STM32 HAL Library \cite{STM_USER_MANUAL}
  \item UART\_Transmit() - STM32 HAL Library \cite{STM_USER_MANUAL}
\end{enumerate}

\subsection{Syntax}
\subsubsection{Exported Constants}
\begin{itemize}
  \item USART\_BAUDRATE = 115200
  \item USART\_TIMEOUT = 1000
  \item USART\_BUFFER\_SIZE = 256
\end{itemize}

\subsubsection{Exported Access Programs}
\begin{center}
\begin{tabularx}{\linewidth}{l X X X}
\hline
\textbf{Name} & \textbf{In} & \textbf{Out} & \textbf{Exceptions} \\
initUSART & - & \hyperref[generic:uint8]{uint8} & USARTInitFailure \\
\hline
transmitData & data: sequence of \hyperref[generic:uint8]{uint8} & - & USARTTxFailure \\
\hline
receiveData & length: \(\mathbb{N}\) & data: sequence of \hyperref[generic:uint8]{uint8} & USARTRxFailure \\

\end{tabularx}
\end{center}

\subsection{Semantics}

\subsubsection{State Variables}
None.

\subsubsection{Environment Variables}
\begin{itemize}
  \item usartBuffer: sequence [USART\_BUFFER\_SIZE] of uint8
\end{itemize}

\subsubsection{Assumptions}
None.

\subsubsection{Access Routine Semantics}
\noindent initUSART():
\begin{itemize}
\item transition: \((\forall i \in \mathbb{N} | 0 \le i < |\text{usartBuffer}|: \text{usartBuffer}[i] := 0) \wedge\)\\
  USART\_Init(USART\_BAUD\_RATE, ...).
\item output: (Successful Initialization) \(\Rightarrow\) 0 \(\wedge\ \neg\) 
  (Successful Initialization) \(\Rightarrow\) 1
\item exception: exc := USARTInitFailure if USART initialization fails.
\end{itemize}

\noindent transmitData(data):
\begin{itemize}
  \item transition: (\(\vert data \vert\ \leq \) USART\_BUFFER\_SIZE) \(\Rightarrow\) 
   copy\_array\_data(data,usartBuffer, \(\vert data \vert\ \leq \)) \(\wedge\) 
   (\(\vert data \vert\ > \) USART\_BUFFER\_SIZE) \(\Rightarrow\) 
    copy\_array\_data(data,usartBuffer, USART\_BUFFER\_SIZE) \(\wedge\) 
    UART\_Transmit(usartBuffer, \(\min(\vert data \vert\ ,\) USART\_BUFFER\_SIZE), USART\_TIMEOUT)
  \item output: None
  \item exception: exc := (UART\_Transmit(...) \(\equiv\) HAL\_ERROR) \(\Rightarrow\) USARTTxFailure 
\end{itemize}

\noindent receiveData(length):
\begin{itemize}
  \item transition: (\(length \leq \) USART\_BUFFER\_SIZE) \(\Rightarrow\) 
    UART\_Receive(usartBuffer, length, USART\_TIMEOUT) \(\wedge\)
    (\(length > \) USART\_BUFFER\_SIZE) \(\Rightarrow\) UART\_Receive(usartBuffer, 
    USART\_BUFFER\_SIZE, USART\_TIMEOUT)
  \item output: data
  \item exception: exc := (UART\_Receive(...) \(\equiv\) HAL\_ERROR) \(\Rightarrow\) USARTRxFailure
\end{itemize}

\subsection{Local Functions}
\begin{itemize}
  \item copy\_array\_data: sequence of \(\mathbb{N} \times\) sequence of \(\mathbb{N} \times \mathbb{N}\ \rightarrow \text{void}\)  \\
  copy\_array\_data(src, dest, length) \(\equiv\) 
  (\(\forall i \in \mathbb{N} | 0 \le i \le \text{length} : \text{dest}[i] := \text{src}[i]\))
\end{itemize}

\newpage
\section{MIS of SD Card Interface Module \label{mod:sdcard} - \mref{MG-mSDInterface}}

\subsection{Module}
This module manages reading from and writing to an SD card connected to the
microcontroller via the SPI interface.

\subsection{Uses}
\begin{enumerate}
  \item SPI Driver - STM32 HAL Library \cite{STM_USER_MANUAL}
  \item FATFS Library - Chan's FatFs \cite{CHAN_FATFS}
\end{enumerate}

\subsection{Syntax}
\subsubsection{Exported Constants}
None.

\subsubsection{Exported Access Programs}
\begin{center}
\begin{tabularx}{\linewidth}{l X X X}
\hline
\textbf{Name} & \textbf{In} & \textbf{Out} & \textbf{Exceptions} \\
\hline
initSDCard & - & \hyperref[generic:uint8]{uint8} & SDCardInitFailure \\
\hline
writeLog & data: \hyperref[generic:string]{string} & - & SDCardWriteFailure \\
\hline
\end{tabularx}
\end{center}

\subsection{Semantics}
\subsubsection{State Variables}
None.
\subsubsection{Environment Variables}
None.
\subsubsection{Assumptions}
None.

\subsubsection{Access Routine Semantics}
\noindent initSDCard():
\begin{itemize}
\item transition: HAL\_SPI\_Init(...) \(\wedge\) f\_mount(...)
\item output: (Successful Initialization) \(\Rightarrow\) 0 \(\wedge\ \neg\) 
  (Successful Initialization) \(\Rightarrow\) 1
\item exception: exc := (SPI \(\vee\) FATFS initialization failure) \(\Rightarrow\) SDCardInitFailure.
\end{itemize}

\noindent writeLog(data):
\begin{itemize}
  \item transition: f\_write(data) \(\wedge\) f\_sync()
  \item output: None
  \item exception: exc := ((f\_write() \(\not\equiv\) FR\_OK) \(\vee\) (f\_sync() failure \(\not\equiv\) FR\_OK))
  \(\Rightarrow\) SDCardWriteFailure 
\end{itemize}

\subsection{Local Functions}
None.

\newpage

\section{MIS of Microphone Diagnostic Module \label{mod:MicDiag}- \mref{MG-mMicDiag}}
\subsection{Module}
This module performs diagnostics on the microphone array to detect hardware
errors.

\subsection{Uses}
\begin{itemize}
  \item ADC Driver - STM32 HAL Library \cite{STM_USER_MANUAL}
\end{itemize}

\subsection{Syntax}
\subsubsection{Exported Constants}
\begin{itemize}
  \item MIC\_DIAG\_ERROR = 1
\end{itemize}

\subsubsection{Exported Access Programs}
\begin{center}
\begin{tabularx}{\linewidth}{l X X X}
\hline
\textbf{Name} & \textbf{In} & \textbf{Out} & \textbf{Exceptions} \\
\hline
initMicDiagnostics & - & \hyperref[generic:uint8]{uint8} & MicDiagInitFailure \\
\hline
runMicDiagnostics & - & \hyperref[generic:uint8]{uint8} & MicDiagFailure \\
\hline
subscribeDiagnostics & callback: \(\mathbb{N} \rightarrow \text{void}\) & - & - \\
\end{tabularx}
\end{center}

\subsection{Semantics}
\subsubsection{State Variables}
\begin{itemize}
  \item diagStatus : \(\mathbb{N}\)
  \item callbacks: sequence of (\(\mathbb{N} \rightarrow \text{void}\))
\end{itemize}

\subsubsection{Environment Variables}
\begin{itemize}
  \item ADC\_Handle: ADC\_HandleTypeDef
  \item adcValue: \(\mathbb{R}\)
\end{itemize}

\subsubsection{Assumptions}
None.

\subsubsection{Access Routine Semantics}
\noindent initMicDiagnostics():
\begin{itemize}
  \item transition: diagStatus := 0 \(\wedge\) ADC\_Init(ADC\_Handle).
  \item output: (Successful Initialization) \(\Rightarrow\) 0 \(\wedge\ \neg\) 
    (Successful Initialization) \(\Rightarrow\) 1
  \item exception: exc := (ADC\_Init(...) \(\equiv\) HAL\_ERROR) \(\Rightarrow\) MicDiagInitFailure.
\end{itemize}

\noindent runMicDiagnostics():
\begin{itemize}
  \item transition: adcValue = HAL\_ADC\_GetValue(ADC\_Handle) \(\wedge\) 
    (map(adcValue, 0, 65535, 0, 3.3) \(<\) 3.1) \(\Rightarrow\) diagStatus := MIC\_DIAG\_ERROR \(\wedge\) 
    (\(\forall\) cb: (\(\mathbb{N} \rightarrow \text{void}\)) \(\vert\) cb \(\in\) callbacks : cb(diagStatus))
    \item output: diagStatus
    \item exception: exc := (ADC read failure) \(\Rightarrow\) MicDiagFailure
\end{itemize}

\noindent subscribeDiagnostics(callback)\label{mdiag:subscribe}:
\begin{itemize}
  \item transition: callbacks := callbacks \(\parallel\) \(\langle\) callback \(\rangle\)
  \item output: None
  \item exception: None
\end{itemize}


\subsection{Local Functions}
\begin{itemize}
  \item map: \(\mathbb{N} \times \mathbb{N} \times \mathbb{N} 
  \times \mathbb{R} \times \mathbb{R} \rightarrow \mathbb{R}\)\\
  map(x, in\_min, in\_max, out\_min, out\_max) \(\equiv\) 
  (x - in\_min) \(\times\) (out\_max - out\_min + 1) / (in\_max - in\_min + 1) + out\_min
\end{itemize}

\newpage

\section{MIS of Audio Generation Module - \mref{MG-mAudioGen}}
\label{mod:AudioGen}

\subsection{Module}

This module generates audio data by simulating room acoustics and generating 
synthetic microphone array data from a given audio source and position. This 
includes room response calculations and spatial audio propagation 
models. 

\subsection{Uses}

\begin{enumerate}
  \item Python Standard Libraries
\end{enumerate}

\subsection{Syntax}

\subsubsection{Exported Constants}

\subsubsection{Exported Access Programs}

\begin{center}
\begin{tabularx}{\linewidth}{l X X X}
\hline
\textbf{Name} & \textbf{In} & \textbf{Out} & \textbf{Exceptions} \\
\hline
generateAudio & audioSource: \hyperref[generic:float32]{float32}[], 
position: \hyperref[generic:float32]{float32}[], 
outputFile: \hyperref[generic:string]{string} & 
micrphoneData[][]: [][\hyperref[generic:float32]{float32}] 
& AudioGenerationFailure \\
\hline
\end{tabularx}
\end{center}

\subsection{Semantics}

\subsubsection{State Variables}

None.

\subsubsection{Environment Variables}

None.

\subsubsection{Assumptions}

The pyroomacoustics module is assumed to be working correctly (generating 
realistic and reliable audio data based on configuration parameters).

\subsubsection{Access Routine Semantics}

\noindent generateAudio():
\begin{itemize}
\item transition: None
\item output: microphoneData
\item exception: AudioGenerationFailure
\end{itemize}

\subsubsection{Local Functions}

\begin{itemize}
\item simulateRoom(room: pra.ShoeBox) $\rightarrow$
  \hyperref[generic:bool]{bool}
\end{itemize}

\newpage
\section{MIS of Direction of Arrival (DOA) Processor Module - 
\mref{MG-mDOA}} \label{mod:DOA}

\subsection{Module}

This module processes audio data to estimate the direction of arrival of a sound 
source. This includes frequency domain analysis, signal processing, and 
direction estimation algorithms.

\subsection{Uses}

\begin{enumerate}
  \item \hyperref[mod:AudioGen]{Audio Generation Module}
  \item \hyperref[mod:AudioNormalizer]{Audio Normalizer Module} 
  \item \hyperref[mod:AudioSpectralLeakage]{Audio Spectral Leakage Prevention
    Module}
  \item \hyperref[mod:AudioAnomaly]{Audio Anomaly Detection Module}
\end{enumerate}

\subsection{Syntax}

\subsubsection{Exported Constants}

None

\subsubsection{Exported Access Programs}

\begin{center}
\begin{tabularx}{\linewidth}{l X X X}
\hline
\textbf{Name} & \textbf{In} & \textbf{Out} & \textbf{Exceptions} \\
\hline
calculateDirection & audioData[][]: [][\hyperref[generic:float32]{float32}] 
& direction: \hyperref[generic:float32]{float32} 
& AudioProcessingFailure \\
\hline
\end{tabularx}
\end{center}

\subsection{Semantics}

\subsubsection{State Variables}

None.

\subsubsection{Environment Variables}

None.

\subsubsection{Assumptions}

None.

\subsubsection{Access Routine Semantics}

\noindent calculateDirection():
\begin{itemize}
\item transition: None
\item output: direction
\item exception: AudioProcessingFailure
\end{itemize}

\subsubsection{Local Functions}

None.

\newpage

\section{MIS of Audio360 Engine \label{mod:Audio360Engine} - 
  \mref{MG-mAudio360Engine}}

\subsection{Module}

Orchestrates the overall audio processing by receiving raw input data and
managing \hyperref[mod:Audio360Engine_Uses]{module} communication.

\subsection{Uses}\label{mod:Audio360Engine_Uses}

\begin{enumerate}
  \item \hyperref[mod:Classification]{Classification Module}
  \item \hyperref[mod:DOA]{DOA Processor Module}
  \item \hyperref[mod:FaultManager]{Fault Manager Module} 
  \item \hyperref[mod:Logging]{Logging Module} 
\end{enumerate}

\subsection{Syntax}

\subsubsection{Exported Constants}

None

\subsubsection{Exported Access Programs}

\begin{center}
\begin{tabularx}{\linewidth}{l X X X}
\hline
\textbf{Name} & \textbf{In} & \textbf{Out} & \textbf{Exceptions} \\
\hline
runProgram & - & - & ProgramStartFailure, ProgramRunTimeFailure \\
\hline
\end{tabularx}
\end{center}

\subsection{Semantics}

\subsubsection{State Variables}

\begin{itemize}
  \item state [\hyperref[enum:Audio360State]{Audio360State}]:
    Determines the current state of the the Audio360 engine. Each state will run
    a specific module in \hyperref[mod:Audio360Engine_Uses]{used modules}.
\end{itemize}

\subsubsection{Environment Variables}

\begin{itemize}
  \item status [\hyperref[enum:Audio360Status]{Audio360Status}]:
    Status of the module. This encapsulates initialization, running, ready,
    or errored.
\end{itemize}

\subsubsection{Assumptions}

None

\subsubsection{Access Routine Semantics}

\noindent run():
\begin{itemize}
\item transition: The state machine in figure
  \ref{fig:audio360_engine_state_machine} outlines the transition of this
  module.

\begin{figure}[h!]
    \centering 
    \includegraphics[width=\textwidth]{diagrams/Audio360EngineStateMachine.png}
    \caption{Internal state machine of Audio360 Engine module.}
    \label{fig:audio360_engine_state_machine}
\end{figure}

\item output: None
\item exception: ProgramStartFailure, ProgramRunTimeFailure
\end{itemize}

\subsubsection{Local Functions}

None

\newpage

\section{MIS of Audio Sampling Module \label{mod:AudioSampling} -
  \mref{MG-mAudioSampling}}

\subsection{Module}
Provides sampling of audio data from a given source while preserving the order
of individual samples.


\subsection{Uses}

\begin{itemize}
  \item \hyperref[mod:MicInput]{Microphone Input Module}
  \item \hyperref[mod:MicDiag]{Microphone Diagnostic Module}
\end{itemize}

\subsection{Syntax}

\subsubsection{Exported Constants}

\begin{itemize}
  \item sampleWindowSize [\hyperref[generic:uint16]{uint16}]:
    \label{var:sampleWindowSize}
    The number of samples in a window. The value is a power of 2.
\end{itemize}

\subsubsection{Exported Access Programs}

\begin{center}
\begin{tabularx}{\linewidth}{l X X X}
\hline
\textbf{Name} & \textbf{In} & \textbf{Out} & \textbf{Exceptions} \\
\hline
sample & inputSource: \hyperref[generic:string]{string} & - & InputSouceError \\
getSampledData & - & \hyperref[var:sampleData]{sampleData} & - \\
\hline
\end{tabularx}
\end{center}

\subsection{Semantics}

\subsubsection{State Variables}

\begin{itemize}
  \item sampledData [\hyperref[generic:uint32]{uint32}[]]:
    \label{var:sampleData}  
    A cyclic array of size \hyperref[var:sampleWindowSize]{sampleWindowSize}
    that stored the sampled data.
  \item windowPosition [\hyperref[generic:uint32]{uint32}]:
    \label{var:windowPosition}  
    The current reference index position of the array storing sampled data.
    This denotes the starting point of the cyclic array. 
\end{itemize}

\subsubsection{Environment Variables}

None

\subsubsection{Assumptions}

None

\subsubsection{Access Routine Semantics}

\noindent sample():
\begin{itemize}
\item transition: 
  \begin{enumerate}
    \item $sampleData[windowPosition] := inputSource.newData$
    \item $((windowPosition + 1) \geq sampleWindowSize) \implies $
      $windowPosition:= 0~\vee \neg ((windowPosition + 1) \geq $
      $sampleWindowSize) \implies$ \\ $ windowPosition:= windowPosition + 1$
  \end{enumerate}

\item output: None
\item exception: InputSouceError
\end{itemize}

\noindent getSampledData():
\begin{itemize}
\item transition: None 
\item output: \hyperref[var:sampleData]{sampleData}
\item exception: None
\end{itemize}

\subsubsection{Local Functions}

None

\newpage
\section{MIS of Audio Spectral Leakage Prevention Module
  \label{mod:AudioSpectralLeakage} - \mref{MG-mAudioSpectralLeakage}}

\subsection{Module}

Applying windowing on audio signal to reduce
\hyperref[SRS-def:spectral_leakage]{spectral leakage} before
\hyperref[SRS-def:frequency_domain]{frequency domain} processing.

\subsection{Uses}

None

\subsection{Syntax}

\subsubsection{Exported Constants}

None

\subsubsection{Exported Access Programs}

\begin{center}
\begin{tabularx}{\linewidth}{l X X X}
\hline
\textbf{Name} & \textbf{In} & \textbf{Out} & \textbf{Exceptions} \\
\hline
applyFilter & sampleWindow:\hyperref[generic:float32]{float32[]} &
filteredWindow: \hyperref[generic:float32]{float32[]} & - \\
\hline
\end{tabularx}
\end{center}

\subsection{Semantics}
\subsubsection{State Variables}

None

\subsubsection{Environment Variables}

None

\subsubsection{Assumptions}

None

\subsubsection{Access Routine Semantics}

\noindent applyFilter():
\begin{itemize}
\item transition: None
\item output:
  \begin{enumerate}
    \item $(\forall~i~|~0 \leq i \leq N - 1:~filteredWindow[i] := $
    $sampleWindow[i] * \sin(\dfrac{\pi * i}{N - 1})^2)$ \\
    such that $N = $ size of sampleWindow.
  \end{enumerate}
\item exception: None
\end{itemize}


\subsubsection{Local Functions}

None


\newpage

\section{MIS of Audio Normalizer Module \label{mod:AudioNormalizer} -
  \mref{MG-mAudioNormalizer}}

\subsection{Module}

Processes audio signals to maintain consistent amplitude across different
sources.

\subsection{Uses}

None

\subsection{Syntax}
\subsubsection{Exported Constants}

None

\subsubsection{Exported Access Programs}

\begin{center}
\begin{tabularx}{\linewidth}{l X X X}
\hline
\textbf{Name} & \textbf{In} & \textbf{Out} & \textbf{Exceptions} \\
\hline
normalize & sampleWindow: \hyperref[generic:int32]{int32[]},
source: \hyperref[generic:string]{string} &
filteredWindow: \hyperref[generic:float32]{float32[]} & - \\
\hline
\end{tabularx}
\end{center}

\subsection{Semantics}

\subsubsection{State Variables}

None

\subsubsection{Environment Variables}

None

\subsubsection{Assumptions}

Normalization factor is pre-deteremined and known for all given sources.

\subsubsection{Access Routine Semantics}

\noindent normalize():
\begin{itemize}
\item transition: None
\item output:
  \begin{enumerate}
    \item $(\forall~i~|~0 \leq i \leq N - 1:~filteredWindow[i] := $
    $\dfrac{sampleWindow[i]}{F_{source}})$ \\
    such that $N = $ size of sampleWindow and $F_{source}$ is normalization
    factor of a the sampleWindow source.
  \end{enumerate}
\item exception: None
\end{itemize}

\subsubsection{Local Functions}

None

\newpage

\section{MIS of Audio Anomaly Detection Module \label{mod:AudioAnomaly} - 
  \mref{MG-mAudioAnomaly}}

\subsection{Module}
This module analyzes audio data to detect anomalies based on predefined
patterns.

\subsection{Uses}
\begin{itemize}
  \item \hyperref[mod:MicInput]{Microphone Input Module} 
\end{itemize}

\subsection{Syntax}
\subsubsection{Exported Constants}
\begin{itemize}
  \item AA\_NONE
  \item AA\_SILENT
  \item AA\_LOUD
  \item AA\_SIL\_AMP\_TIM\_THRESH = 10
  \item AA\_SIL\_AMP\_THRESH = 100
  \item AA\_LOUD\_AMP\_THRESH = 60000
\end{itemize}

\subsubsection{Exported Access Programs}
\begin{center}
\begin{tabularx}{\linewidth}{l X X X}
\hline
\textbf{Name} & \textbf{In} & \textbf{Out} & \textbf{Exceptions} \\
\hline
initAnomalyDetector & - & \hyperref[generic:uint8]{uint8} & AnomalyDetectorInitFailure \\
\hline
\end{tabularx}
\end{center}

\subsection{Semantics}
\subsubsection{State Variables}
\begin{itemize}
  \item aStat : \(\mathbb{N}\)
  \item saCounter : \(\mathbb{N}\)
\end{itemize}

\subsubsection{Environment Variables}
\begin{itemize}
  \item micInputCallback : \hyperref[ds:cb-mic-data]{CALLBACK\_MIC\_DATA}
\end{itemize}

\subsection{Assumptions}
None.

\subsubsection{Access Routine Semantics}
\noindent initAnomalyDetector():

\begin{itemize}
  \item transition: aStat = AA\_NONE \(\wedge\)\\ 
  
    micInputCallback := 
    cb\_function(data: [MIC\_INPUT\_BUFFER\_SIZE] of int16) \(\rightarrow\ \{0, 1\}\)\\
    \(\text{cb\_function(data)} \equiv\)
    \[
\begin{alignedat}{2}
&
\Big(
   (\text{silent\_function(data)} \equiv 1)
   \Rightarrow
   \Big(
      \text{saCounter} := \text{saCounter} + 1
      \;\wedge\allowbreak\\[-2pt]
   &\quad
      ((\text{saCounter} \equiv \text{AA\_SIL\_AMP\_TIM\_THRESH})
         \Rightarrow
         \text{aStat} := \text{AA\_SILENT})
   \Big)
\Big)
\;\vee\allowbreak\\[4pt]
&\Big(
   (\neg(\text{silent\_function(data)} \equiv 1))
   \Rightarrow
   (\text{saCounter} := 0)
\Big)
\;\vee\allowbreak\\[4pt]
&\Big(
   (\text{loud\_noise\_function(data)} \equiv 1)
   \Rightarrow
   (\text{aStat} := \text{AA\_LOUD})
\Big)
\;\vee\allowbreak\\[4pt]
&\Big(
   (\neg\,\text{silent\_function(data)} \wedge
    \neg\,\text{loud\_noise\_function(data)})
   \Rightarrow
   (\text{aStat} := 0 \wedge
    \text{saCounter} := 0)
\Big)
\end{alignedat}
\]

\end{itemize}


\subsection{Local Functions}
\begin{itemize}
  \item silent\_function: [MIC\_INPUT\_BUFFER\_SIZE] of int16 \(\rightarrow\ \{0, 1\}\)\\
  silent\_function(data) \(\equiv\) ((\(+ i \vert 0 \le i < \text{MIC\_INPUT\_BUFFER\_SIZE }
  \wedge \text{ data[i] } < \) \\AA\_SIL\_AMP\_THRESH : 1)
  \( \equiv \) MIC\_INPUT\_BUFFER\_SIZE)
  \item loud\_noise\_function: [MIC\_INPUT\_BUFFER\_SIZE] of int16 \(\rightarrow\ \{0, 1\}\)\\
  loud\_noise\_function(data) \(\equiv\) ((\(+ i \vert 0 \le i < \text{MIC\_INPUT\_BUFFER\_SIZE }
  \wedge \text{ data[i] } > \) AA\_LOUD\_AMP\_THRESH : 1)
  \( \equiv \) MIC\_INPUT
  \_BUFFER\_SIZE)

\end{itemize}

\newpage
\section{MIS of Fast Fourier Transform - \mref{MG-mFFT}} \label{mod:FFT}

\subsection{Module}

Computes the discrete Fourier transform of the input signal to 
obtain its \hyperref[SRS-def:frequency_domain]{frequency domain}

\subsection{Uses}

\begin{itemize}
  \item \hyperref[mod:AudioSpectralLeakage]{Audio Spectral Leakage Prevention
    Module}
\end{itemize}

\subsection{Syntax}

\subsubsection{Exported Constants}

None

\subsubsection{Exported Access Programs}

\begin{center}
\begin{tabularx}{\linewidth}{l X X X}
\hline
\textbf{Name} & \textbf{In} & \textbf{Out} & \textbf{Exceptions} \\
\hline
signalToFrequency & signal: \hyperref[generic:int32]{int32[]} & frequency \hyperref[generic:float32]{float32[]} & - \\
\hline
\end{tabularx}
\end{center}

\subsection{Semantics}

\subsubsection{State Variables}

None

\subsubsection{Environment Variables}

\begin{itemize}
  \item inputsize [uint32]: The size of the signal input, as in number of
     samples. This is required to interface with hardware acceleration methods
     on the microcontroller.
\end{itemize}

\subsubsection{Assumptions}

None

\subsubsection{Access Routine Semantics}

\noindent signalToFrequency():
\begin{itemize}
\item transition: None
\item output:
  \begin{enumerate}
    \item $(\forall~k~|~0 \leq k \leq N - 1:~frequency[k] := $
    $\sum_{n=0}^{N-1} x_n * e^{-2i \pi n \frac{k}{N}})$ \\
    such that $N = $ size of input signal.
  \end{enumerate}
\item exception: None
\end{itemize}


\subsubsection{Local Functions}

\begin{itemize}
  \item createOutput(): returns a
    \hyperref[struct:FrequencyDomain]{FrequencyDomain}. This function extract
    features from the FFT and stores it in the data structure.
\end{itemize}

\newpage

\section{MIS of Logging Module \label{mod:Logging} -
  \mref{MG-mLogging}}

\subsection{Module}

Provides centralized logging of data streams to designated output destinations
for debugging and monitoring.

\subsection{Uses}

\begin{itemize}
  \item \hyperref[mod:UsartComm]{USART Communication Module}
  \item \hyperref[mod:sdcard]{SD Card Interface Module}
\end{itemize}

\subsection{Syntax}
\subsubsection{Exported Constants}

None

\subsubsection{Exported Access Programs}

\begin{center}
\begin{tabularx}{\linewidth}{l X X X}
\hline
\textbf{Name} & \textbf{In} & \textbf{Out} & \textbf{Exceptions} \\
\hline
log & text: \hyperref[generic:string]{string} & - & - \\
\hline
\end{tabularx}
\end{center}

\subsection{Semantics}

\subsubsection{State Variables}

None

\subsubsection{Environment Variables}

None

\subsubsection{Assumptions}

None

\subsubsection{Access Routine Semantics}

\noindent log():
\begin{itemize}
\item transition: None
\item output: input text is logged to output source (SD card or USART port)
\item exception: None
\end{itemize}

\subsubsection{Local Functions}

None

\newpage 


\section{MIS of Fault Manager Module - \mref{MG-mFaultManager}
  \label{mod:FaultManager}} 

\subsection{Module}

Monitors system health and manages critical faults to maintain core
functionality.

\subsection{Uses}

\begin{enumerate}
  \item \hyperref[mod:MicDiag]{Microphone Diagnostic Module}
  \item \hyperref[mod:Classification]{Classification Module}
  \item \hyperref[mod:DOA]{DOA Processor Module}
  \item \hyperref[mod:AudioAnomaly]{Audio Anomaly Detection Module}
\end{enumerate}

\subsection{Syntax}

\subsubsection{Exported Constants}

None

\subsubsection{Exported Access Programs}

\begin{center}
\begin{tabularx}{\linewidth}{l X X X}
\hline
\textbf{Name} & \textbf{In} & \textbf{Out} & \textbf{Exceptions} \\
\hline
runFaultAnalysis & - & - & - \\
\hline
\end{tabularx}
\end{center}

\subsection{Semantics}

\subsubsection{State Variables}

\begin{itemize}
  \item faultState [\hyperref[enum:FaultState]{FaultState}]:
    \label{faultState}
    The current fault state of the overall software system.
\end{itemize}

\subsubsection{Environment Variables}

\begin{itemize}
  \item \hyperref[faultState]{faultState}: This variable is shared with the
    display, to notifiy user of the internal status of the software system.
\end{itemize}

\subsubsection{Assumptions}

Fault Manager module itself does not run into any critical faults.

\subsubsection{Access Routine Semantics}

\noindent faultState():
\begin{itemize}
\item transition: update the value of faultState depending on the overal health
  of the software system.
\item output: None
\item exception: None
\end{itemize}

\subsubsection{Local Functions}

None
\newpage


\section{MIS of Mel Filter Module \label{mod:MelFilter} - 
\mref{MG-mMelFilter}}

\subsection{Module}

Converts a Spectrogram into the equivalent Mel-Spectrogram by applying a bank 
of $n$ mel-scaled triangular filters to each frequency frame.

\subsection{Uses}

None

\subsection{Syntax}

\subsubsection{Exported Constants}

None

\subsubsection{Exported Access Programs}

\begin{center}
\begin{tabularx}{\linewidth}{l X X X}
\hline
\textbf{Name} & \textbf{In} & \textbf{Out} & \textbf{Exceptions} \\
\hline
applyMelFilterBank & spectrogram: 2D real array & mel-spectrogram: 2D 
real array & invalidSpectrogramDimension \\
\hline
\end{tabularx}
\end{center}

\subsection{Semantics}
\subsubsection{State Variables}

None

\subsubsection{Environment Variables}

None

\subsubsection{Assumptions}

\begin{itemize}
  \item The input spectrogram is the output of the 
  \hyperref[MG-mFFT]{FFT Module} across a constant buffer time. This is 
  arranged as a 2D matrix where the rows represent time frames and columns 
  represent frequency bins. This matrix can be represented as $S(t, k)$ where 
  $k$ is the number of frequency bins, and $t$ is the number of time frames. 
  \item The mel filterbank matrix is precomputed offline and stored as a read 
  only constant within the system. It is computed offline using pre-determined 
  information like Sampling Rate, FFT Size, maximum and minimum frequencies and 
  number of mel filters. This matrix can be represented as $H(m,k)$, where $m$ 
  is the number of mel-filters, $k$ is the number of frequency bins and $H(m,k)$ 
  is a weight that tells you how much frequency bin $k$ contributes to mel band 
  $m$.
\end{itemize}

\subsubsection{Access Routine Semantics}

\noindent applyMelFilterBank():
\begin{itemize}
\item \textbf{transition:}
  None (function doesn't have states)
\item \textbf{output:}
  Returns the equivalent mel-spectrogram, where each time frame is converted 
  from linear-frequency bin representation to mel-scaled frequency bins. This
  is done by:
  \begin{enumerate}
    \item Multiplying each spectrogram frame by the mel filterbank matrix. 
    \item Summing weighted energy contributions per mel filter. 
  \end{enumerate}
    \[
    E(t, m) = \sum_{k} S(t, k) \cdot H(m, k)
    \]
\item \textbf{exception:} 
  Invalid spectrogram dimension. If the dimension of the input spectrogram is 
  not $S(t,k)$ and perhaps something like $S(t,p)$ where $p \neq k$, then the 
  matrix multiplication would throw an invalid dimension error, which must be 
  exposed from the module as invalid spectrogram dimension as well. This would 
  only happens if the number of frequency bins in the computted FFT is not 
  equal to the number of frequency bins in the pre-computed mel filterbank 
  matrix. 
\end{itemize}

\subsubsection{Local Functions}

None

\newpage 

\section{MIS of Discrete Cosine Transform Module \label{mod:DCT} - 
\mref{MG-mDCT}}

\subsection{Module}

Converts a mel-spectrogram into a 2D array containing the sequence of Mel 
Frequency Cepstral Coefficients (MFCC) vectors by applying the Discrete Cosine Transform (DCT) to each time frame. This reduces overlap between 
frequency bands that carry similar information and ensures each coefficient is 
uncorrelated to the other one. Pre-processing step that makes it easier to 
perform Principle Component Analysis later on. 

\subsection{Uses}

None

\subsection{Syntax}

\subsubsection{Exported Constants}

None

\subsubsection{Exported Access Programs}

\begin{center}
\begin{tabularx}{\linewidth}{l X X X}
\hline
\textbf{Name} & \textbf{In} & \textbf{Out} & \textbf{Exceptions} \\
\hline
applyDCT & mel-spectrogram: 2D real array & mfcc: 2D 
real array & invalidDimensions \\
\hline
\end{tabularx}
\end{center}

\subsection{Semantics}

\subsubsection{State Variables}

None

\subsubsection{Environment Variables}

None

\subsubsection{Assumptions}

\begin{itemize}
  \item The input mel-spectrogram is the output of the 
  \hyperref[MG-mMelFilter]{Mel Filter Module}. This input will be of size 
  $E(t, m)$, where $t$ is the number of time frames, and $m$ is the number of 
  mel-frequency bins. 
  \item The DCT transform matrix $D(c,m)$ will be computed offline, where 
  $c$ is the desired number of MFCC coefficients for each time frame, and $m$ is 
  the number of mel-frequency bands. $D(c,m)$ represents the unique cosine wave 
  that is associated with the $c^{\text{th}}$ coefficient and $m^{\text{th}}$
  band. A combination of these waves will be used to uniquely represent the 
  spectrogram at time frame $t$. 

    \[
    D(c,m) = \cos\!\left(\frac{\pi c}{M}(m - 0.5)\right)
    \]
  
  \item Only 13 MFCC coefficients will be calculated for each time frame $t$. 
  
\end{itemize}

\subsubsection{Access Routine Semantics}

\noindent applyDCT():
\begin{itemize}
\item \textbf{transition:}
  None (function doesn't have states)
\item \textbf{output:}
  Returns a matrix \textit{mfcc(t,c)}, such that for each time frame $t$ and each 
  MFCC coefficient index $c$, the value of the matrix at $mfcc(t,c)$ can be
  computed as
      \[
      mfcc(t, c) = \sum_{m=1}^{M} E(t, m) \cdot DCT(c, m)
      \]

  Where $E(t, m)$ represents the mel-spectrogram at time frame $t$ and 
  mel-frequency $m$. And $DCT(c,m)$ represents the DCT transform matrix that 
  was computed offline. 
\item \textbf{exception:} 
  Invalid spectrogram dimension. If the dimension of the input spectrogram is 
  not $E(c,m)$ and perhaps something like $E(c,p)$ where $m \neq p$ then the 
  matrix multiplication would throw an invalid dimension error, which must be 
  exposed from the module as invalid spectrogram dimension as well. This would 
  only happens if for some reason the number of mel-frequency bands returned 
  by the \hyperref[MG-mMelFilter]{Mel Filter Module} is not the same as the 
  number of mel-frequency bands in the DCT matrix computation.
\end{itemize}

\subsubsection{Local Functions}

None

\newpage 

\section{MIS of Principle Component Analysis Module \label{mod:PCA} - 
\mref{MG-mPCA}}

\subsection{Module}

Principle Component Analysis (PCA) highlights the features with the greatest 
variance from an input feature matrix. In this case, uses the MFCC feature 
vector for each time frame to find the direction of greatest variance. This 
helps with eliminating noise and unimportant information.

\subsection{Uses}

Non

\subsection{Syntax}

\subsubsection{Exported Constants}

None

\subsubsection{Exported Access Programs}

\begin{center}
\begin{tabularx}{\linewidth}{l X X X}
\hline
\textbf{Name} & \textbf{In} & \textbf{Out} & \textbf{Exceptions} \\
\hline
applyPCA & mfcc: 2D real array & pcaFeatures: 2D 
real array & invalidDimensions \\
\hline
\end{tabularx}
\end{center}

\subsection{Semantics}

\subsubsection{State Variables}

None

\subsubsection{Environment Variables}

None

\subsubsection{Assumptions}

\begin{itemize}
  \item The input MFCC matrix is the output of the
  \hyperref[MG-mDCT]{Discrete Cosine Transform Module}. This input will be of 
  size $mfcc(t,c)$, where $t$ is the number of time frames, and $c$ is the 
  number of coefficients. 
  \item The PCA mean vector and PCA projection matrix will be derived offline 
  during the system training and stored as read-only constants. The PCA mean 
  vector will be of size $(1,c)$, and will be computed as the average MFCC 
  vectors across all training samples. 
    \[
    \bar{x} = \frac{1}{T} \sum_{t=1}^{T} x(t)
    \]

  The PCA projection matrix will be of size $(K, c)$, where $K$ is the number 
  of eigenvectors with the maximum variance and $c$ is the number of MFCC 
  coefficients. These eigenvectors are extracted from the following covariance 
  matrix. 

    \[
    C = \frac{1}{T} \sum_{t=1}^{T} 
    \left(x(t) - \bar{x}\right)
    \left(x(t) - \bar{x}\right)^{T}
    \]
   
  
\end{itemize}

\subsubsection{Access Routine Semantics}

\noindent applyPCA():
\begin{itemize}
\item \textbf{transition:}
  None (function doesn't have states)
\item \textbf{output:}
  Returns a matrix \textit{pca(t, K)}, where $t$ represents the number of time 
  frames and $K$ represents the number of PCA components retained (proportional 
  to the number of eignevectors). Essentially, each input mfcc matrix is 
  projected onto the vectors of maximum variance, therefore extracting only the 
  most important features of the MFCC based on training. This projection can be 
  computed using the following matrix multiplication. 

  \[
  \text{pca} = \left( mfcc - \mathbf{1}\bar{x} \right) P^{T}
  \]

  Where $P$ represents the projection matrix, $\bar{x}$ represents the PCA 
  mean vector, and $mfcc$ represents the full mfcc matrix. 
\item \textbf{exception:} 
  Invalid mfcc dimension. If the dimension of the input mfcc matrix is 
  not $(t,c)$ and perhaps something like $(t,p)$ where $c \neq p$ then the 
  matrix multiplication would throw an invalid dimension error, which must be 
  exposed from the module as invalid mfcc dimension as well. This would 
  only happens if for some reason the number of coefficients returned 
  by the \hyperref[MG-mDCT]{Discrete Cosine Transform} is not the same as the 
  number of coefficients in the projection matrix.
\end{itemize}

\subsubsection{Local Functions}

None

\newpage 

\section{MIS of Linear Discriminant Analysis Module \label{mod:LDA} - 
\mref{MG-mLDA}}

\subsection{Module}

Classifies given feature vector into one of the predefined set of
sound classes. This is module is optimized for class seperation. 
The feature vector in this case is already narrowed down to the 
features with the most variance using the \hyperref[MG-mPCA]{PCA} 
module. 

\subsection{Uses}

None

\subsection{Syntax}

\subsubsection{Exported Constants}

None

\subsubsection{Exported Access Programs}

\begin{center}
\begin{tabularx}{\linewidth}{l X X X}
\hline
\textbf{Name} & \textbf{In} & \textbf{Out} & \textbf{Exceptions} \\
\hline
applyLDA & pcaFeatures: 2D real array & classLabels: 1D 
real array & invalidDimensions \\
\hline
\end{tabularx}
\end{center}

\subsection{Semantics}

\subsubsection{State Variables}

None

\subsubsection{Environment Variables}

None

\subsubsection{Assumptions}

\begin{itemize}
  \item The input pcaFeatures is the output of the
  \hyperref[MG-mPCA]{Principle Component Analysis Module}. This input will be 
  of size $(t,K)$, where $t$ is the number of time frames, and $K$ is 
  number of features selected with the highest variances from PCA.
  \item The LDA projection matrix, and class weight parameters are 
  computed offline using labelled training data. They are stored 
  as read-only constants in the system. The LDA projection matrix is of size 
  $(K,C-1)$, where $C$ is the number of classes, and $K$ is the number of
  variance eigenvectors. In essence, this matrix maps the input feature space 
  into the discriminant subspace where class seperation is maximized.
  
\end{itemize}

\subsubsection{Access Routine Semantics}

\noindent applyLDA():
\begin{itemize}
\item \textbf{transition:}
  None (function doesn't have states)
\item \textbf{output:}

  Returns a vector of class labels, one per time frame. 

  Each feature vector from PCA is projected onto the discriminant space using 
  the LDA projection matrix using the following multiplication. 

  \[
  z(t,:) = pca(t,:) \cdot W_{\text{LDA}}
  \]

  Where $pca(t,:)$ represents the PCA eigenvectors at time $t$, and 
  $W_{\text{LDA}}$ represents the LDA projection matrix. 

  Classification is then performed using linear discriminant functions for 
  each class $j$, where $w_j$ and $b_j$ is the weight and bias matricies 
  that were pre-computed offline during training for class $j$. 

  \[
  g_j(t) = w_j^{T} z(t,:) + b_j
  \]

  The predicted class for that time frame is then assigned as the class with 
  the maximum value. 

  \[
  \text{classLabel}(t) = \arg\max_{j} \, g_j(t)
  \]

  The output of the module is then a vector of length t, representing the best 
  classification for all time frames. 


\item \textbf{exception:} 
  Invalid feature space dimension. If the dimension of the input feature matrix 
  is not $(t,K)$ and perhaps something like $(t,p)$ where $K \neq p$ then the 
  matrix multiplication would throw an invalid dimension error, which must be 
  exposed from the module as invalid feature dimension as well. This would 
  only happens if for some reason the number of eigenvectors returned 
  by the \hyperref[MG-mPCA]{PCA} module is not the same as the 
  number of eignvectors in the precomputed LDA matrix.
\end{itemize}

\subsubsection{Local Functions}

None

\newpage 

\section{MIS of Classification Module \label{mod:Classification} - 
\mref{MG-mClassification}}

\subsection{Module}

Orchestrates the overall audio classification flow by receiving signals in the 
frequency domain and encapsulating the complexity of the various modules 
involved. 

\subsection{Uses}

\begin{enumerate}
  \item \hyperref[mod:AudioSampling]{Audio Sampling Module}
  \item \hyperref[mod:AudioNormalizer]{Audio Normalizer Module}
  \item \hyperref[mod:FFT]{Fast Fourier Transform Module}
  \item \hyperref[mod:MelFilter]{Mel Filter Module}
  \item \hyperref[mod:DCT]{Discrete Cosine Transform Module}
  \item \hyperref[mod:PCA]{Principle Component Analysis Module}
  \item \hyperref[mod:LDA]{Linear Discrete Analysis Module}
  \item 
\end{enumerate}

\subsection{Syntax}

\subsubsection{Exported Constants}

None

\subsubsection{Exported Access Programs}

\begin{center}
\begin{tabularx}{\linewidth}{l X X X}
\hline
\textbf{Name} & \textbf{In} & \textbf{Out} & \textbf{Exceptions} \\
\hline
classify & spectrogram: 2D real array & - & ClassificationStartFailure, 
ClassificationRunTimeFailure \\
\hline
\end{tabularx}
\end{center}

\subsection{Semantics}

\subsubsection{State Variables}

\begin{itemize}
  \item state[classificationState]: Stores the current classification. 
  Will continuously be updated at runtime after recieving continuous audio 
  signals. 
  \item state[classificationStatus]: Stores the confidence of the 
  classification that was done. Values for status include high, medium or low 
  confidence. 
\end{itemize}

\subsubsection{Environment Variables}

\begin{itemize}
  \item state[status]: Encapsulates initialization, running, 
  ready or errored status of the module.  
\end{itemize}

\subsubsection{Assumptions}
None.

\subsubsection{Access Routine Semantics}

\noindent applyLDA():
\begin{itemize}
\item \textbf{transition:}
  \hyperref[fig:classification_state_machine]{State machine} below outlines the 
  transition of this module.

\begin{figure}[h!]
    \centering 
    \includegraphics[width=\textwidth]{diagrams/ClassificationStateMachine.png}
    \caption{Internal state machine of Classification module.}
    \label{fig:classification_state_machine}
\end{figure}

\item \textbf{output:} None
\item \textbf{exception:} ClassificationStartFailure, 
ClassificationRunTimeFailure

\end{itemize}

\subsubsection{Local Functions}

None

\newpage 

\section{MIS of Visualization Module \label{mod:Visualization} - 
\mref{MG-mVisualization}}

\subsection{Module}

This module formats and renders visual information on the smart glasses display. 
It receives classification labels and direction angles from the Audio360 Engine, 
formats this information according to display constraints and prioritization 
rules, and renders visual indicators on the smart glasses display. The module 
ensures non-intrusive presentation while maintaining a minimum update rate of 30 
frames per second. It also handles display of error states and safety feature 
failure alerts.

\subsection{Uses}

\begin{enumerate}
  \item \hyperref[mod:Audio360Engine]{Audio360 Engine Module}
  \item \hyperref[mod:FaultManager]{Fault Manager Module}
\end{enumerate}

\subsection{Syntax}

\subsubsection{Exported Constants}

\begin{itemize}
  \item minUpdateRate [\hyperref[generic:uint16]{uint16}]: 
    \label{var:minUpdateRate}
    The minimum display update rate in frames per second. The value is 30 FPS 
    as specified by \hyperref[SRS-NFR8_1]{NFR8.1}.
\end{itemize}

\subsubsection{Exported Access Programs}

\begin{center}
\begin{tabularx}{\linewidth}{l X X X}
\hline
\textbf{Name} & \textbf{In} & \textbf{Out} & \textbf{Exceptions} \\
\hline
updateDisplay & classificationLabel: 
  \hyperref[generic:string]{string}, 
  directionAngle: \hyperref[generic:float32]{float32}, 
  priority: \hyperref[generic:uint8]{uint8} & - & 
  DisplayUpdateFailure, InvalidInputError \\
\hline
displayError & errorType: 
  \hyperref[generic:string]{string}, 
  errorMessage: \hyperref[generic:string]{string} & - & 
  DisplayUpdateFailure \\
\hline
\end{tabularx}
\end{center}

\subsection{Semantics}

\subsubsection{State Variables}

\begin{itemize}
  \item currentDisplayState [\hyperref[ds:dict]{dict}]: 
    \label{var:currentDisplayState}
    Stores the current state of the display, including active visual indicators, 
    their positions, and priorities. This state is continuously updated at 
    runtime as new classification and direction data is received.
  \item displayQueue [\hyperref[ds:list]{list}]: 
    \label{var:displayQueue}
    A priority queue that stores pending display updates. Items are ordered by 
    priority to ensure safety-critical information is displayed first, as 
    required by \hyperref[SRS-NFR6_1]{NFR6.1}.
\end{itemize}

\subsubsection{Environment Variables}

\begin{itemize}
  \item display [Display Hardware]: 
    The smart glasses display hardware interface. This encapsulates the 
    physical display device and its driver interface, allowing the visualization 
    format to be modified independently of the underlying display technology.
\end{itemize}

\subsubsection{Assumptions}

\begin{itemize}
  \item The display hardware is properly initialized and functional.
  \item The Audio360 Engine provides valid classification labels and direction 
    angles in radians within the range [0, 2$\pi$).
  \item Priority values are non-negative integers, with lower values indicating 
    higher priority (safety-critical information has priority 0).
\end{itemize}

\subsubsection{Access Routine Semantics}

\noindent updateDisplay():
\begin{itemize}
\item \textbf{transition:} 
  The module receives classification and direction data, applies 
  prioritization logic to determine if the information should be displayed 
  (based on \hyperref[SRS-NFR6_1]{NFR6.1}), formats the information according 
  to display constraints and non-intrusive presentation requirements 
  (\hyperref[SRS-NFR6_2]{NFR6.2}), and updates the display state. The visual 
  indicator is rendered on the smart glasses display at the appropriate position 
  corresponding to the direction angle. If multiple sources are present, only 
  the highest priority source is displayed.

\item \textbf{output:} None

\item \textbf{exception:} 
  DisplayUpdateFailure if the display hardware fails to update, or 
  InvalidInputError if the direction angle is outside the valid range [0, 2$\pi$) 
  or if the classification label is empty.
\end{itemize}

\noindent displayError():
\begin{itemize}
\item \textbf{transition:} 
  The module formats and displays an error message on the smart glasses display 
  to alert users when core safety features such as direction determination or 
  classification fail, as required by \hyperref[SRS-FR6_2]{FR6.2}. Error 
  messages are displayed with high priority to ensure visibility.

\item \textbf{output:} None

\item \textbf{exception:} 
  DisplayUpdateFailure if the display hardware fails to update the error 
  message.
\end{itemize}

\subsubsection{Local Functions}

\begin{itemize}
  \item formatDisplayData(classificationLabel: 
    \hyperref[generic:string]{string}, directionAngle: 
    \hyperref[generic:float32]{float32}) $\rightarrow$ 
    \hyperref[ds:dict]{dict}: 
    Formats the classification label and direction angle into a display-ready 
    data structure, including position calculations for directional indicators 
    and text labels.
  \item applyPrioritization(displayQueue: 
    \hyperref[ds:list]{list}) $\rightarrow$ 
    \hyperref[ds:dict]{dict}: 
    Selects the highest priority item from the display queue to be rendered, 
    ensuring safety-critical information is displayed first as required by 
    \hyperref[SRS-NFR6_1]{NFR6.1}.
  \item renderNonIntrusive(displayData: 
    \hyperref[ds:dict]{dict}) $\rightarrow$ 
    \hyperref[generic:bool]{bool}: 
    Renders visual indicators in a non-intrusive manner, minimizing visual 
    obstruction so users can safely perform external activities, as required by 
    \hyperref[SRS-NFR6_2]{NFR6.2}.
\end{itemize}

\newpage 

\section{MIS of Independent Component Analysis (ICA) Module 
\label{mod:ICA} - \mref{MG-mICA}}

\subsection{Module}

This module performs blind source separation on multi-channel audio signals to 
separate mixed audio signals into individual independent source components. The 
module accepts multi-channel audio signals from the microphone array and outputs 
separated audio streams, each representing an individual sound source. This 
enables improved classification and direction estimation accuracy when multiple 
sound sources are present simultaneously. The module is conditionally invoked by 
the Audio360 Engine when multiple sources are detected, allowing the system to 
operate with or without this capability.

\subsection{Uses}

\begin{enumerate}
  \item \hyperref[mod:Audio360Engine]{Audio360 Engine Module}
  \item \hyperref[mod:AudioSampling]{Audio Sampling Module}
\end{enumerate}

\subsection{Syntax}

\subsubsection{Exported Constants}

None

\subsubsection{Exported Access Programs}

\begin{center}
\begin{tabularx}{\linewidth}{l X X X}
\hline
\textbf{Name} & \textbf{In} & \textbf{Out} & \textbf{Exceptions} \\
\hline
separateSources & mixedAudio: 
  [][\hyperref[generic:float32]{float32}] & 
  separatedSources: [][\hyperref[generic:float32]{float32}] & 
  ICASeparationFailure, InvalidInputError \\
\hline
\end{tabularx}
\end{center}

\subsection{Semantics}

\subsubsection{State Variables}

\begin{itemize}
  \item separationModel [\hyperref[ds:dict]{dict}]: 
    \label{var:separationModel}
    Stores the ICA separation model parameters and learned mixing matrix. This 
    state may be updated during the separation process to adapt to the input 
    signal characteristics.
\end{itemize}

\subsubsection{Environment Variables}

None

\subsubsection{Assumptions}

\begin{itemize}
  \item The input audio signals contain multiple independent sound sources that 
    can be separated using ICA.
  \item The number of microphones is at least equal to the number of sound 
    sources to be separated.
  \item The sound sources are statistically independent and non-Gaussian.
  \item The mixing process is linear and instantaneous (no significant delays 
    between sources reaching different microphones).
\end{itemize}

\subsubsection{Access Routine Semantics}

\noindent separateSources():
\begin{itemize}
\item \textbf{transition:} 
  The module applies Independent Component Analysis (e.g., FastICA algorithm) to 
  the input mixed audio signals from multiple microphones. The algorithm 
  estimates the mixing matrix and separates the mixed signals into independent 
  source components. The separation model state may be updated during this 
  process. Each output stream represents a separated audio source that can be 
  further processed by classification and direction estimation modules.

\item \textbf{output:} 
  An array of separated audio source streams, where each stream contains the 
  audio data for one independent sound source. The number of output streams 
  corresponds to the number of detected independent sources.

\item \textbf{exception:} 
  ICASeparationFailure if the ICA algorithm fails to converge or cannot 
  successfully separate the sources. InvalidInputError if the input audio data 
  is invalid, has incorrect dimensions, or does not meet the assumptions required 
  for ICA separation.
\end{itemize}

\subsubsection{Local Functions}

\begin{itemize}
  \item estimateMixingMatrix(mixedAudio: 
    [][\hyperref[generic:float32]{float32}]) $\rightarrow$ 
    \hyperref[ds:dict]{dict}: 
    Estimates the mixing matrix that describes how the independent sources are 
    combined in the mixed signals. This is a core step in the ICA algorithm.
  \item applySeparation(mixedAudio: 
    [][\hyperref[generic:float32]{float32}], mixingMatrix: 
    \hyperref[ds:dict]{dict}) $\rightarrow$ 
    [][\hyperref[generic:float32]{float32}]: 
    Applies the inverse of the estimated mixing matrix to separate the mixed 
    signals into independent source components.
  \item checkConvergence(previousModel: 
    \hyperref[ds:dict]{dict}, currentModel: 
    \hyperref[ds:dict]{dict}) $\rightarrow$ 
    \hyperref[generic:bool]{bool}: 
    Checks if the ICA algorithm has converged by comparing the current model 
    parameters with the previous iteration's parameters.
\end{itemize}

\newpage 

\bibliographystyle{IEEEtran}
\bibliography{../../../refs/References}


\newpage{}

\section*{Appendix --- Reflection}

\begin{enumerate}
  \item What went well while writing this deliverable? 
  
  \textbf{Sathurshan:} The system was decomposed into modules that was small
  enough. This allowed the team to easily split up the work without having
  much dependencies on each other. Furthermore, writing the MIS has helped the
  team further align with design decisions and formally document.

  \textbf{Kalp:} I think that developing the PoC (Proof of Concept) was a great
  way to get a better understanding of the system and the modules that are
  involved. Though maybe not ideal in other scenarios, getting a small headstart
  on the design implementation helped us really easily scope out what the 
  modules on the system should be.

  \textbf{Nirmal:} Since the classification POC was already created, I feel 
  like it was a lot easier to decompose the classification feature into 
  distinct features. Furthermore, since the POC was created earlier, a lot of 
  the research was already completed, which means I knew what equations I 
  needed for each module. 

  \textbf{Jay:} I think the part that went well was how aligned the team already 
  was before writing the document. Because we had been discussing the system and 
  reviewing each module ahead of time, putting everything together felt more 
  organized. The MIS ended up reinforcing our understanding and helped us
  clearly lay out how each part of the system will work.

  \textbf{Omar: } Our team knocked it out of the park by using formal
  math to define each module's functionality whenever possible. This was 
  especially helpful for reducing workload looking ahead. 

  \item What pain points did you experience during this deliverable, and how
    did you resolve them?

  \textbf{Omar: } The major pain point was the timing of this deliverable. 
  Although it is done well, it required too much time and effort while
  balancing other deliverables and courses. We tried to resolve this by
  delegating sections to each team member based on their expertise. 

  \textbf{Sathurshan:} The main pain point was the deliverable deadline as the
  team was asked to finish the first revision in a week while preparing for the
  proof of concept of demo. The team expressed the concerns with the professor,
  and fortunately received an extension allowing us to put more effort into this
  document.

  \textbf{Kalp:} I think that the main pain point was just timing. With the VnV 
  plan due shortly prior to this deliverable, and the team PoC implementation 
  being due a week after this deliverable, there was really not much time to 
  write out this document, especially given the length and detail that it goes 
  into. This was slightly mitigated by the provided extension and the PoC 
  implementation, but it was still a challenge to get everything done in time.

  \textbf{Nirmal:} Although I mentioned I had a good idea of which modules 
  are needed for classification, modularizing which equation goes in what 
  module and in what order was a difficult and time consuming task. For example 
  the POC was done as a python implementation, and as such the logic with 
  coming up with the LDA projection matrix was encapsulated into python 
  libraries. For this document, I had to reverse engineer how exactly this is 
  done using equations. 

  \textbf{Jay:} The biggest pain point for me was keeping everything consistent 
  across the sections, especially since multiple people were writing at the same 
  time while also balancing work on the PoC. It was a bit overwhelming to juggle 
  both. We resolved this by coordinating early and reviewing each other's 
  sections so the document stayed cohesive. The extension also gave us some 
  breathing room to fix issues we noticed.

  \item Which of your design decisions stemmed from speaking to your client(s)
  or a proxy (e.g. your peers, stakeholders, potential users)? For those that
  were not, why, and where did they come from?

  \textbf{Omar:} Many design decisions stemmed from past experience working
  on embedded projects and large system designs. The approach to effectively
  break down larger tasks such as audio classification into smaller modules
  (e.g., FFT, Mel Filter, DCT, PCA, LDA) was inspired by prior discussion 
  with our capstone supervisor. 

  \textbf{Sathurshan:} The supervisor mentioned the difficulty of getting
  access to hardware that will meet our software specification. As a result, we
  designed the system such that the software is portable and can take input
  from different sources. This ensures that the capstone project can still
  be successful in the case we are not able to find the right hardware within
  our budget.

  \textbf{Kalp:} A lot of the hardware and software design decisions that we 
  made was based from speaking to the supervisor and the team. Our supervisor 
  would be able to give better insight on what would and wouldn't work for 
  the scope of what we're trying to accomplish. We selected the pipeline 
  design of audio to processing to output with the specific hardware, but the 
  specific algorithm (ex. LDA for classification) or the specific hardware came
  through our supervisors insights. 

  \textbf{Nirmal:} The decision of using PCA and LDA for classification stems 
  from asking our supervisor (MVM) for input on an approach that has low 
  computation. Originally, I was using a SVM for classifying audio sources, 
  but our supervisor mentioned this isn't really necessary, and might be 
  overkill. 

  \textbf{Jay:} Most of the high-level decisions came from discussions with the 
  supervisor, especially around what is realistic for the hardware we have 
  access to. Hearing their perspective helped steer us toward a design that is
  flexible and doesn’t rely too heavily on one specific setup. Some of the 
  lower-level decisions, like how we break up certain processes, came from our
  own team discussions and figuring out what would make implementation more 
  manageable.

  \item While creating the design doc, what parts of your other documents (e.g.
  requirements, hazard analysis, etc), it any, needed to be changed, and why?

  \textbf{Omar:} While writing the design document, some parts of the SRS needed
  to be updated to add labels and to link module decomposition to the SRS 
  requirements. 

  \textbf{Sathurshan:} SRS needed to be updated. It was updated because part
  of writing this document required reviewing the SRS. From this, I have found
  parts of the requirements that can be improved based on recent better
  understanding of the system.

  \textbf{Kalp:} The SRS document was updated to reflect the 
  changes that we made to the system. A lot of the requirements changes revolved 
  around the hardware and software design decisions that we made (ex. choosing
  to separate a process into multiple modules leads to more specific 
  requirements). 
  
  \textbf{Nirmal:} For the classification specific modules, no parts of any 
  other documents needed to be changed, since the other documents never 
  mentioned any exact implementation details. Which is what this document 
  is focused on. 

  \textbf{Jay:} Similar to the others, I found that parts of the SRS needed 
  updates. While detailing the modules in this document, it became clear that a 
  few requirements could be phrased more precisely or split into smaller pieces 
  based on the updated understanding of the system. So the changes mostly came 
  from wanting the requirements to better match the actual design.

  \item What are the limitations of your solution?  Put another way, given
  unlimited resources, what could you do to make the project better?
  (LO\_ProbSolutions)

  \textbf{Omar:} The main limitation in our case is the hardware constraints.
  Given unlimited resources, we would be able to design custom audio hardware that
  meets all of our software requirements without any guess work. We would also
  be able to design PCBs that are optimized for reducing trace lengths between
  the microphones and the microcontroller, improving signal integrity. 
  
  \textbf{Sathurshan:} The limitation of the solution is compute and memory
  power. Given further time, the team would be able to design a system that
  is more low level and is optimized in accomplishing the tasks that are
  required.

  \textbf{Kalp:} I think the biggest limitation that we can see right now is the
  hardware limitations. The software that we're using for this project is good 
  enough to handle the tasks that we're trying to accomplish within the 
  software requirements, but the hardware limitations are what are holding us 
  back slightly. With just better hardware (at the expense of higher cost), a 
  lot of the issues we're dealing with could be mitigated.
  
  \textbf{Nirmal:} Given unlimited resources, we could have a more robust 
  classification approach. The current approach of using PCA and LDA comes from 
  not having computation power to use resources like SVM or neural networks 
  taking in the full audio input. 

  \textbf{Jay:} The biggest limitation I see is tied to performance and the 
  hardware we're working with. If we had more resources, we could definitely 
  explore more powerful hardware and push the system to handle more complex 
  processing or higher-resolution data. With more time and computing power, 
  we could also refine the software to be more efficient and include features 
  that aren't feasible right now.
  
  \item Give a brief overview of other design solutions you considered. What
  are the benefits and tradeoffs of those other designs compared with the chosen
  design?  From all the potential options, why did you select the documented
  design? (LO\_Explores)

  \textbf{Omar:} During the research phase, we considered using many different
  hardware platforms including Raspberry Pi, Texas Instruments and Arduino.
  We settled on the STM32 platform due to its inclusion of DSP libraries and audio
  peripherals. While the Texas Instruments controllers have great documentation
  and software support, they are specialized for motor control applications which
  means they lack the real-time audio peripherals required for the project. 
  On the software side, we considered using more complex machine
  learning algorithms for audio classification, but ultimately chose not to due to 
  simplicity and compute resource constraints

  \textbf{Sathurshan:} Personally, there was not a lot of time to think about
  other designs. The team has been implementing the software before this
  document was created since the proof of concept is one week after this
  document is due. Thus, the team already considered and analyzed high level
  designs months ago and is too far back to document them.

  \textbf{Kalp:} I know for the software side, specifically the audio 
  classification, we considered using a neural network for the classification.
  This was a potential solution, but we ultimately decided against it because
  of the hardware limitations. The neural network would require a lot of memory
  and processing power, which is not available on the hardware that we're using.
  For this reason, we decided to use the LDA for classification because it is a 
  simple algorithm that is easy to implement and understand, and a lot less 
  computationally expensive than a neural network.

  \textbf{Nirmal:} Another design solution we considered for classification was 
  using SVM. Although that seemed to work really well, compared to the PCA and 
  LDA approach, this approach would firstly take a lot of storage on the 
  microcontroller which we didn't have. And also involved running full 
  blackbox models. The approach now just involves doing matrix multiplications 
  which is what the microcontroller is optimized for. 

  \textbf{Jay:} One alternative we discussed early on was using more advanced 
  machine learning methods for classification, like a small neural network or 
  more complex models. They might have improved accuracy, but they also would 
  have required more memory and processing power than we realistically have. 
  The chosen design gives us a simpler, reliable system that fits within our 
  constraints, which made it the more practical option for the project. 

\end{enumerate}


\end{document}