\documentclass{article}

\usepackage{float}
\restylefloat{table}

\usepackage{booktabs}
\usepackage{cite}
\usepackage{xr}

\title{Team Contributions: POC\\\progname}

\author{\authname}

\date{}

\input{../Comments}
%% Common Parts

\newcommand{\progname}{Audio360} % PUT YOUR PROGRAM NAME HERE
\newcommand{\authname}{Team 6, SixSense
\\ Omar Alam
\\ Sathurshan Arulmohan
\\ Nirmal Chaudhari
\\ Kalp Shah
\\ Jay Sharma
} % AUTHOR NAMES                  

\usepackage{hyperref}
    \hypersetup{colorlinks=true, linkcolor=blue, citecolor=blue, filecolor=blue,
                urlcolor=blue, unicode=false}
    \urlstyle{same}
                                


\externaldocument[DevPlan-]{../DevelopmentPlan/DevelopmentPlan}

\begin{document}

\maketitle

\newpage

\tableofcontents

\newpage

This document provides a summary of each team member's contributions leading
up to the Proof of Concept (POC) demonstration. The reporting period covers all
work completed from the beginning of the term through November 2, 2025.

\section{Demo Plans}

The team will be demonstrating the core features of \progname working in a 
simulated environment. The core features include direction detection of a sound
source, where the system outputs the angle of arrival of the detected sound. 
The other feature is audio classification of atleast 3 distinct classifications.
Demonstrating that these features work at a high level  will allow the team to
get validation from stakeholders who are hard of hearing ensuring that the
solution effectively addresses their pain point of limited situational
awareness.

According to the SRS \cite{SRS}, these features are intended to be deployed on a
microcontroller integrated into a pair of smart glasses. However, the shipment
of the glasses is scheduled to arrive after the POC deadline. While the team
currently has access to the microcontrollers, the hardware setup and integration
process require significant development effort. As a result, the system will
initially be deployed in a simulated environment for the POC. In this setup,
pre-recorded microphone array audio data will be used as input for the core
algorithms. The team will employ pyroomacoustics \cite{pyroomacoustics}, a
Python based audio room processing library, to generate simulated audio input
based on the spatial configuration and relative positions of the microphones
and sound sources.

\section{Team Meeting Attendance}

Team meetings are held weekly to discuss progress updates, address blockers,
and plan upcoming tasks.

\begin{table}[H]
\centering
\begin{tabular}{ll}
\toprule
\textbf{Student} & \textbf{Meetings}\\
\midrule
Total & 7\\
Omar Alam & 6\\
Sathurshan Arulmohan & 6\\
Nirmal Chaudhari & 7\\
Kalp Shah & 6\\
Jay Sharma & 6\\
\bottomrule
\end{tabular}
\end{table}


\section{Supervisor Meeting Attendance}

Supervisor meetings were held only when expert guidance was required, minimizing\
the number of scheduled sessions.
\newline
\newline
\noindent \textbf{Supervisor's Name: } Dr. Martin v. Mohrenschildt

\begin{table}[H]
\centering
\begin{tabular}{ll}
\toprule
\textbf{Student} & \textbf{Meetings}\\
\midrule
Total & 2\\
Omar Alam & 2\\
Sathurshan Arulmohan & 2\\
Nirmal Chaudhari & 2\\
Kalp Shah & 2\\
Jay Sharma & 1\\
\bottomrule
\end{tabular}
\end{table}

\section{Lecture Attendance}

Lecture tracking began on September 15, 2025. Lectures prior to this date were
not documented, as the team and project charter had not yet been established.
The team agreed that at least one member would attend each lecture and take
notes, with Nirmal designated as the
\hyperref[DevPlan-role:note_taker]{Note Taker}. In cases where Nirmal
was unable to attend, another team member served as a substitute. The team
successfully ensured that at least one representative attended every lecture.

\begin{table}[H]
\centering
\begin{tabular}{ll}
\toprule
\textbf{Student} & \textbf{Lectures}\\
\midrule
Total & 7\\
Omar Alam & 4\\
Sathurshan Arulmohan & 5\\
Nirmal Chaudhari & 5\\
Kalp Shah & 2\\
Jay Sharma & 1\\
\bottomrule
\end{tabular}
\end{table}


\section{TA Document Discussion Attendance}


\noindent \textbf{TA's Name: } Rashad Bhuiyan

\begin{table}[H]
\centering
\begin{tabular}{ll}
\toprule
\textbf{Student} & \textbf{TA Meetings}\\
\midrule
Total & 3\\
Omar Alam & 3\\
Sathurshan Arulmohan & 3\\
Nirmal Chaudhari & 3\\
Kalp Shah & 3\\
Jay Sharma & 3\\
\bottomrule
\end{tabular}
\end{table}

\section{Commits}

\begin{table}[H]
\centering
\begin{tabular}{lll}
\toprule
\textbf{Student} & \textbf{Commits} & \textbf{Percent}\\
\midrule
Total & 283 & 100\% \\
Omar Alam & 33 & 11.7\% \\
Sathurshan Arulmohan & 142 & 50.2\% \\
Nirmal Chaudhari & 48 & 17.0\% \\
Kalp Shah & 31 & 10.9\% \\
Jay Sharma & 29 & 10.2\% \\
\bottomrule
\end{tabular}
\end{table}

Sathurshan's commit count is higher than that of other team members
because GitHub registers a commit each time a developer merges a pull request
related to documentation updates. This is caused by the continuous deployment
bot automatically generating a commit when building and commiting the PDF
documentation. Sathurshan has primarily been responsible for managing pull
request merges on days when documentation submissions are due.

Furthermore, Omar has focused on hardware-related tasks, which are not
reflected in the commit metrics. Similarly, Nirmal, Kalp, and Jay have
undertaken research tasks for feature development that does not directly
result in commits.

\section{Issue Tracker}

\begin{table}[H]
\centering
\begin{tabular}{lll}
\toprule
\textbf{Student} & \textbf{Authored (O+C)} & \textbf{Assigned (C only)}\\
\midrule
Omar Alam & 5 & 24 \\
Sathurshan Arulmohan & 132 & 56 \\
Nirmal Chaudhari & 7 & 32 \\
Kalp Shah & 13 & 21 \\
Jay Sharma & 0 & 24 \\
\bottomrule
\end{tabular}
\end{table}

Sathurshan has a higher number of authored issues, as it is the
\hyperref[DevPlan-role:leader]{Leader's} responsibility to create and assign
tasks to team members for each milestone. This role distribution was agreed
upon by the team, as outlined in the Development Plan document
\cite{DevelopmentPlan}. Additionally, Sathurshan has closed more issues than
other members, having resolved the majority of peer review items at the time
of the report (12 out of 20 closed issues).

\section{CI/CD}

Section \textbf{CI/CD} of the Development Document \cite{DevelopmentPlan}
describes the team's approach to implementing CI/CD for the project.
At a high level, the CI/CD pipeline is used to automatically execute all
team-developed tests, including unit and integration tests, and to build the
source code using the target compiler. This process ensures that new code
changes do not break or regress previously implemented functionality.

\section{Team Charter Trigger Items}

There have been no team charter trigger items to date, and the team has made
good progress on the project. However, there is room for improvement in overall
work ethics throughout the semester to ensure the capstone is completed on
schedule. To address this, a
\href{https://github.com/Team6-SixSense/audio360/issues/278}{GitHub issue} has
been created to revise the team charter and implement the necessary updates to
support the successful completion of the project.

\newpage
\bibliographystyle{IEEEtran}
\bibliography{../../refs/References}

\end{document}