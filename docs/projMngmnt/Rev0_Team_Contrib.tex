\documentclass{article}

\usepackage{float}
\restylefloat{table}

\usepackage{booktabs}
\usepackage{cite}
\usepackage{xr}

\externaldocument[DevPlan-]{../DevelopmentPlan/DevelopmentPlan}

\title{Team Contributions: Rev 0\\\progname}

\author{\authname}

\date{}

\input{../Comments}
%% Common Parts

\newcommand{\progname}{Audio360} % PUT YOUR PROGRAM NAME HERE
\newcommand{\authname}{Team 6, SixSense
\\ Omar Alam
\\ Sathurshan Arulmohan
\\ Nirmal Chaudhari
\\ Kalp Shah
\\ Jay Sharma
} % AUTHOR NAMES                  

\usepackage{hyperref}
    \hypersetup{colorlinks=true, linkcolor=blue, citecolor=blue, filecolor=blue,
                urlcolor=blue, unicode=false}
    \urlstyle{same}
                                


\begin{document}

\maketitle

\newpage

\tableofcontents

\newpage

This document summarizes the contributions of each team member for the Rev 0
Demo.  The time period of interest is the time between the PoC demo
(November 20, 2025) and the Rev 0 demo (January 28, 2026); the contributions
prior to the PoC are NOT included.

\section{Demo Plans}

The team will demonstrate the core features of \progname~running directly on the
microcontroller. This represents a major advancement from the proof of concept
(POC), where these features were previously implemented and tested using a
Python-based simulation environment, pyroomacoustics \cite{pyroomacoustics}.
The features being presented include Direction of Arrival (DoA) estimation and
audio classification. Both features now execute entirely on the microcontroller,
which required implementation in embedded languages along with careful attention
to memory usage and performance optimization.

The team has also upgraded the hardware with a new microphone array arrangement
that matches the physical dimensions of the glasses. In addition, several
improvements have been implemented that are not directly visible, including
dynamic memory allocation for microphone data to improve efficiency and the
development of serial communication protocols for transmitting direction and
classification data to the glasses.

In addition, the team has received Rokid glasses and a compatible data cable
that enables streaming output directly to the device. As a result, the team will
present real-time visualization on the glasses themselves. The visualization
receives direction and classification data from the microcontroller and displays
the results on the glasses frame for selected sounds.

\section{Team Meeting Attendance}

Team meetings are held weekly starting the Winter semester to discuss progress
updates, address blockers, and plan upcoming tasks. Team meetings were stopped
after POC until the new year so that the team can  focus on final exams.
Snow days in second semester has forced the team to cancel meetings, however,
teammates provided updates in our team channel.

\begin{table}[H]
\centering
\begin{tabular}{ll}
\toprule
\textbf{Student} & \textbf{Meetings}\\
\midrule
Total & 2\\
Omar Alam & 2\\
Sathurshan Arulmohan & 2\\
Nirmal Chaudhari & 2\\
Kalp Shah & 1\\
Jay Sharma & 1\\
\bottomrule
\end{tabular}
\end{table}

\section{Supervisor/Stakeholder Meeting Attendance}

Supervisor meetings were held only when expert guidance was required, minimizing
the number of scheduled sessions.
\newline
\newline
\noindent \textbf{Supervisor's Name: } Dr. Martin v. Mohrenschildt

\begin{table}[H]
\centering
\begin{tabular}{ll}
\toprule
\textbf{Student} & \textbf{Meetings}\\
\midrule
Total & 1\\
Omar Alam & 1\\
Sathurshan Arulmohan & 1\\
Nirmal Chaudhari & 1\\
Kalp Shah & 1\\
Jay Sharma & 1\\
\bottomrule
\end{tabular}
\end{table}

\section{Lecture Attendance}

The team agreed that at least one member would attend each lecture and take
notes, with Nirmal designated as the
\hyperref[DevPlan-role:note_taker]{Note Taker}. In cases where Nirmal
was unable to attend, another team member served as a substitute. The team
successfully ensured that at least one representative attended every lecture.

\begin{table}[H]
\centering
\begin{tabular}{ll}
\toprule
\textbf{Student} & \textbf{Lectures}\\
\midrule
Total & 1\\
Omar Alam & 1\\
Sathurshan Arulmohan & 0\\
Nirmal Chaudhari & 1\\
Kalp Shah & 0\\
Jay Sharma & 0\\
\bottomrule
\end{tabular}
\end{table}


\section{TA Document Discussion Attendance}

Due to the snow storm and McMaster closure, the only TA document discussion was
cancelled.

\noindent \textbf{TA's Name: } Rashad Bhuiyan

\begin{table}[H]
\centering
\begin{tabular}{ll}
\toprule
\textbf{Student} & \textbf{Lectures}\\
\midrule
Total & 0\\
Omar Alam & 0\\
Sathurshan Arulmohan & 0\\
Nirmal Chaudhari & 0\\
Kalp Shah & 0\\
Jay Sharma & 0\\
\bottomrule
\end{tabular}
\end{table}

\section{Commits}

Commit data metrics date ranges from Novemeber 20, 2025 to January 28, 2026.
The following command was used to extract this metric:\\
\textit{git shortlog -sne --since="20 Nov 2025" --before="28 Jan 2026"}

\begin{table}[H]
\centering
\begin{tabular}{lll}
\toprule
\textbf{Student} & \textbf{Commits} & \textbf{Percent}\\
\midrule
Total & 141 & 100\% \\
Omar Alam & 39 & 27.7\% \\
Sathurshan Arulmohan & 72 & 51\% \\
Nirmal Chaudhari & 26 & 18.5\% \\
Kalp Shah & 0 & 0\% \\
Jay Sharma & 4 & 2.8\% \\
\bottomrule
\end{tabular}
\end{table}

\begin{itemize}
    \item Kalp Shah was unable to contribute to development due to medical
    circumstances. Therefore, the number of commits attributed to him is zero.
    He communicated this situation to the team in advance, and Sathurshan
    implemented the Direction of Arrival feature in his place.
    
    \item Five of Sathurshan's commits correspond to work completed by Omar and
    Nirmal. These commits appear under Sathurshan's Github account because Omar
    and Nirmal performed debugging and testing on his laptop, which was already
    connected to the microcontroller.
\end{itemize}


\section{Issue Tracker}

Sathurshan has a higher number of authored issues, as it is the
\hyperref[DevPlan-role:leader]{Leader's} responsibility to create and assign
tasks to team members for each milestone. This role distribution was agreed
upon by the team, as outlined in the Development Plan document
\cite{DevelopmentPlan}.

\begin{table}[H]
\centering
\begin{tabular}{lll}
\toprule
\textbf{Student} & \textbf{Authored (O+C)} & \textbf{Assigned (C only)}\\
\midrule
Omar Alam & 1 & 19 \\
Sathurshan Arulmohan & 46 & 37 \\
Nirmal Chaudhari & 3 & 7 \\
Kalp Shah & 0 & 2 \\
Jay Sharma & 0 & 4 \\
\bottomrule
\end{tabular}
\end{table}


\section{CICD}

Section \textbf{CI/CD} of the Development Document \cite{DevelopmentPlan}
describes the team's approach to implementing CI/CD for the project.
At a high level, the CI/CD pipeline is used to automatically execute all
team-developed tests, including unit and integration tests, and to build the
source code using the target compiler. This process ensures that new code
changes do not break or regress previously implemented functionality.

The following are all the CI github actions the team has. 

\begin{enumerate}
    \item Build source code using taget compiler.
    \item Run unit and integration tests.
    \item Source code static analyzer.
    \item Build tex files upon changes.
    \item Build Android app.
\end{enumerate}

The only CD item is commiting pdf documents whenever there is a change. Code
deployment is not possible due to limited hardware setup for the team.


\section{Team Charter Trigger Items}

There have been no team charter trigger items to date, and the team has made
excellent progress on the project. The team is at a state where major features
of the project are completed. The workload was temporarily adjusted due to the
short-term absence of a team member, however, the member has since returned and
the team will ensure that responsibilities are evenly distributed moving
forward.

\newpage
\bibliographystyle{IEEEtran}
\bibliography{../../refs/References}


\end{document}