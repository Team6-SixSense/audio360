\documentclass[12pt, titlepage]{article}

\usepackage{booktabs}
\usepackage{tabularx}
\usepackage{hyperref}
\usepackage{cite}
\usepackage{xr}
\usepackage{graphicx}
\usepackage{float}
\usepackage{url}
\hypersetup{
    colorlinks,
    citecolor=black,
    filecolor=black,
    linkcolor=red,
    urlcolor=blue
}



\input{../Comments}
%% Common Parts

\newcommand{\progname}{Audio360} % PUT YOUR PROGRAM NAME HERE
\newcommand{\authname}{Team 6, SixSense
\\ Omar Alam
\\ Sathurshan Arulmohan
\\ Nirmal Chaudhari
\\ Kalp Shah
\\ Jay Sharma
} % AUTHOR NAMES                  

\usepackage{hyperref}
    \hypersetup{colorlinks=true, linkcolor=blue, citecolor=blue, filecolor=blue,
                urlcolor=blue, unicode=false}
    \urlstyle{same}
                                


\externaldocument[SRS-]{../SRS/SRS}

\begin{document}

\title{Verification and Validation Report: \progname} 
\author{\authname}
\date{\today}
	
\maketitle

\pagenumbering{roman}

\section{Revision History}

\begin{tabularx}{\textwidth}{p{3cm}p{2cm}X}
\toprule {\bf Date} & {\bf Version} & {\bf Notes}\\
\midrule
March 9, 2026 & 1.0 & Initial VNV report. \\
\bottomrule
\end{tabularx}

~\newpage

\section{Symbols, Abbreviations and Acronyms}

\begin{table}[h]
  \centering
  \begin{tabular}{|l|l|} 
    \toprule		
    \textbf{symbol} & \textbf{description}\\
    \midrule 
    \progname & 360 Audio analysis system on smart glasses\\ \hline
    DOA & Direction of Arrival \\ \hline
    FFT & Fast Fourier Transform \\ \hline
    FR & Functional Requirement\\ \hline
    HUD & Heads Up Display \\ \hline
    M & Module \\ \hline
    NFR & Non-Functional Requirement\\ \hline
    R & Requirement\\ \hline    
    SRS & Software Requirements Specification\\ \hline
    T & Test\\ \hline
    VnV & Verification and Validation \\ \hline
    \bottomrule

  \end{tabular}
  \caption{Table of abbreviations and acronyms used in \progname.}
\end{table}

See SRS Documentation at
\hyperref[SRS-sec:symbols]{Symbols, Abbreviations, and Acronyms}
for a complete table used in \progname.


\newpage

\tableofcontents

\listoftables %if appropriate

\listoffigures %if appropriate

\newpage

\pagenumbering{arabic}

This document ...

\section{Functional Requirements Evaluation}

\section{Nonfunctional Requirements Evaluation}

\subsection{Usability}
		
\subsection{Performance}

\subsection{etc.}
	
\section{Comparison to Existing Implementation}	

This section will not be appropriate for every project.

\section{Unit Testing}

\section{Changes Due to Testing}

\wss{This section should highlight how feedback from the users and from 
the supervisor (when one exists) shaped the final product.  In particular 
the feedback from the Rev 0 demo to the supervisor (or to potential users) 
should be highlighted.}

\section{Automated Testing}
		
\section{Trace to Requirements}
		
\section{Trace to Modules}		

\section{Code Coverage Metrics}

Table \ref{tab:code_coverage} presents the code coverage metrics summary as of
March 1, 2026. Code coverage analysis was performed using gcovr \cite{gcovr}, a
Python based tool that utilizes gcov to generate line, function, and branch
coverage reports. Given the safety critical nature of \progname, achieving
full test coverage was established as a target objective.
However, this objective was not fully achieved due to gcovr reporting
branch coverage for portions of the C++ standard library in addition to project
developed source code.
Despite this limitation, the results indicate greater than $90\%$ line and
function coverage and greater than $60\%$ branch coverage, which is considered
sufficient for the scope of this capstone project and demonstrates a high level
of verification.

These coverage metrics exclude manufacturer supplied libraries and hardware
interface components. Verification of manufacturer code is outside the project's
responsibility, and meaningful testing of hardware interfaces would require
extensive mocking, which was determined to be beyond the scope of the project
objectives and available resources.

A copy of the code coverage report is also saved to this repository in
\href{https://github.com/Team6-SixSense/audio360/blob/main/docs/VnVReport/coverage.html}{\path{docs/VnVReport/coverage.html}}.

\begin{table}[h]
  \centering
  \begin{tabular}{|l|l|l|} 
    \toprule		
    \textbf{Metric} & \textbf{Coverage (\%)} & \textbf{Executed / Total} \\
    \midrule 
    Lines & 91.4 & $604 / 661$\\ \hline
    Functions & 94.4 & $85 / 90$\\ \hline
    Branches & 64.1 & $261 / 407$\\ \hline
    \bottomrule
  \end{tabular}
  \caption{Code coverage mertrics summary.}
  \label{tab:code_coverage}
\end{table}

\newpage
\bibliographystyle{IEEEtran}
\bibliography{../../refs/References}

\newpage{}
\section*{Appendix --- Reflection}

The information in this section will be used to evaluate the team members on the
graduate attribute of Reflection.

\input{../Reflection.tex}

\begin{enumerate}
  \item What went well while writing this deliverable? 
  \item What pain points did you experience during this deliverable, and how
    did you resolve them?
  \item Which parts of this document stemmed from speaking to your client(s) or
  a proxy (e.g. your peers)? Which ones were not, and why?
  \item In what ways was the Verification and Validation (VnV) Plan different
  from the activities that were actually conducted for VnV?  If there were
  differences, what changes required the modification in the plan?  Why did
  these changes occur?  Would you be able to anticipate these changes in future
  projects?  If there weren't any differences, how was your team able to clearly
  predict a feasible amount of effort and the right tasks needed to build the
  evidence that demonstrates the required quality?  (It is expected that most
  teams will have had to deviate from their original VnV Plan.)
\end{enumerate}

\end{document}